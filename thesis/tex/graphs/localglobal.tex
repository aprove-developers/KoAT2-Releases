\begin{figure}
\centering
\begin{tikzpicture}[->,>=stealth',auto,node distance=5cm,
    thick,
    main node/.style={circle,draw,font=\sffamily\Large\bfseries},
    aligned edge/.style={align=left}]

  \node[main node] (0) {$\location_0$};
  \node[main node] (1) [right of=0] {$\location_1$};
  \node[main node] (2) [right of=1] {$\location_2$};

  \path[every node/.style={font=\sffamily\small}]
    (0) edge[aligned edge] node[above=0.2cm] {$t_0$} node[below=0.2cm] {$\update(x) = x + 1$} (1)
    (1) edge[aligned edge] node[above=0.2cm] {$t_1$} node[below=0.2cm] {$\update(x) = 2 \cdot x$} (2)
    ;
\end{tikzpicture}
\caption{Example for the difference between local and global effects}
\label{fig:localglobal}
\end{figure}
