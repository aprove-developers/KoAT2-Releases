\documentclass{scrartcl}

\usepackage{amsmath}
\usepackage{amssymb}
\usepackage{amsfonts}
\usepackage{todonotes}

\newcommand\abs[1]{\left|#1\right|}

\newtheorem{theorem}{Theorem}
\newtheorem{definition}{Definition}
\newtheorem{remark}{Remark}

\begin{document}

\section{Programs}

\begin{definition}[Program] 
	Let $V$ be the set of all occurring program variables.
	Let $L$ be the set of all locations of a program.
	Let $T = L \times F(V) \times L$ be the set of all transitions of the program where $F(V)$ is the set of all quantifier-free formulas over $V$.
	A program is a graph $P = (L, T)$ where $L$ is the set of program locations and $T$ is a subset of all possible transitions between them.
	We write $l_0$ for the unique program location with $l_0 \in L$ which has no entry transitions in the program.
	We write $l_e$ for a program location $l_e \in L$ which has no outgoing transitions in the program.
\end{definition}

\begin{definition}[Valuation] 
	A valuation is a function $\sigma: V \rightarrow \mathbb{Z}$ which assigns each program variable a value.
\end{definition}

\begin{definition}[Configuration] 
	A configuration is a tuple $(l, \sigma)$ where $l \in L$ and $\sigma: V \rightarrow \mathbb{Z}$.
\end{definition}

\begin{definition}[Evaluations] 
	An evaluation is a function which maps a configuration with a transition to a result configuration.
	We write $(l, \sigma) \rightarrow_t (l', \sigma')$ iff with $t = (l, \tau, l') \in T$ it holds that $\sigma \models \tau$ and $\sigma' \models \tau$.
	We omit the transition and write $(l, \sigma) \rightarrow (l', \sigma')$ iff there exists a transition $t \in T$ such that $(l, \sigma) \rightarrow_t (l', \sigma')$ holds.
	We write $(l, \sigma) \rightarrow^k (l', \sigma')$ iff there exists a sequence of transitions $t_1, \dots, t_k \in T$ such that $(l, \sigma) \rightarrow_{t_1} \dots \rightarrow_{t_k} (l', \sigma')$ holds.
	We write $(l, \sigma) \rightarrow^* (l', \sigma')$ iff there exists a $k \in \mathbb{N}$ such that $(l, \sigma) \rightarrow^k (l', \sigma')$ holds.
	We write $(l, \sigma) \rightarrow_{T'} (l', \sigma')$ iff there exists a transition $t \in T' \subseteq T$ such that $(l, \sigma) \rightarrow_t (l', \sigma')$ holds.
\end{definition}

\begin{definition}[Polynomial Ranking Function] 
	Let $\mathbb{Z}[V]$ be the polynomial ring in the variables $v \in V$ over the field $\mathbb{Z}$.
	We define $p(\sigma) = p[v_1/\sigma(v_1), \dots, v_n/\sigma(v_n)]$ as the substitution of all $v_1, \dots v_n \in V$ by the valuation $\sigma: V \rightarrow \mathbb{Z}$ in the polynomial $p \in \mathbb{Z}[V]$.
	We define $\mathit{Pol}: L \rightarrow \mathbb{Z}[V]$ as polynomial ranking function for a transition set $T$ iff there is a nonempty $T_{>} \subseteq T$ such that:
	\[ \text{For all } (l, \tau, l') \in T \text{ it holds that } \tau \Rightarrow \mathit{Pol}(l)(\sigma) \geq \mathit{Pol}(l)(\sigma') \]
	\[ \text{For all } (l, \tau, l') \in T_{>} \text{ it holds that } \tau \Rightarrow \mathit{Pol}(l)(\sigma) > \mathit{Pol}(l)(\sigma') \]
	\[ \text{For all } (l, \tau, l') \in T_{>} \text{ it holds that } \tau \Rightarrow \mathit{Pol}(l)(\sigma) > 1 \]
\end{definition}

\section{Runtime}

\begin{definition}[Worst-Case Runtime Complexity]
	\[ \text{rc}_{\text{worst}}(\sigma_0) = \mathit{sup} \lbrace k \in \mathbb{N} \mid \exists l', \sigma': (l_0, \sigma_0) \rightarrow^k (l', \sigma') \rbrace \]
\end{definition}

\begin{definition}[Best-Case Runtime Complexity]
	\[ \text{rc}_{\text{best}}(\sigma_0) = \mathit{inf} \lbrace k \in \mathbb{N} \mid \exists \sigma_e, l_e: (l_0, \sigma_0) \rightarrow^k (l_e, \sigma_e) \rbrace \]
\end{definition}

\begin{remark}[Worst-Case to Best-Case Relation]
	For all input valuations $\sigma_0 \in V \rightarrow \mathbb{Z}$ it holds that $\text{rc}_{\text{best}}(\sigma_0) \leq \text{rc}_{\text{worst}}(\sigma_0)$.
\end{remark}

\section{Bounds}

\begin{definition}[Polynomial set]
	The set $P$ of polynomials is the smallest set with
	\[ v \in P \text{ for all } v \in V \] 
	\[ k \in P \text{ for all } k \in \mathbb{N} \]
	\[ -p \in P \text{ for all } p \in P \]
	\[ p_1 + p_2 \in P \text{ for all } p_1, p_2 \in P \]
	\[ p_1 * p_2 \in P \text{ for all } p_1, p_2 \in P \]
\end{definition}

\begin{definition}[Bound set]
	The set $B$ of possible bounds is the smallest set with
	\[ P \subset B \] 
	\[ \omega \in B \]
	\[ \mathit{max}(b_1, b_2) \in B \text{ for all } b_1, b_2 \in B \]
	\[ \mathit{min}(b_1, b_2) \in B \text{ for all } b_1, b_2 \in B \]
	\[ k^b \in B \text{ for all } k \in \mathbb{N}, b \in B \]
\end{definition}

\subsection{Runtime Bounds}

\begin{definition}[Lower Runtime Bound]
	\[ \mathcal{R}_\sqcup(t)(m) \leq \mathit{inf} \lbrace k \in \mathbb{N} \mid \exists \sigma_0, \sigma_e, l_e: (l_0, \sigma_0) (\rightarrow^*_{T \setminus \lbrace t \rbrace} \circ \rightarrow^t \circ \rightarrow^*_{T \setminus \lbrace t \rbrace})^k (l_e, \sigma_e) \rbrace \]
\end{definition}

\begin{definition}[Upper Runtime Bound]
	\[ \mathcal{R}_\sqcap(t)(m) \geq \mathit{sup} \lbrace k \in \mathbb{N} \mid \exists \sigma_0, l', \sigma': (l_0, \sigma_0) (\rightarrow^* \circ \rightarrow^t)^k (l', \sigma') \rbrace \]
\end{definition}

\begin{remark}[Upper to Lower Bound Relation]
	For all $\sigma_0 \in V \rightarrow \mathbb{Z}$ and $t \in T$ it holds that $\mathcal{R}_\sqcap(t)(\sigma_0) \geq \mathcal{R}_\sqcup(t)(\sigma_0)$.
\end{remark}

\begin{definition}[Runtime Bound]
	\[ \mathcal{R}(t)(m) = 
		\left( \begin{array}{c}
			\mathcal{R}_\sqcap(t)(m)\\
			\mathcal{R}_\sqcup(t)(m)
		\end{array} \right) \]
\end{definition}

\begin{remark}[Approximating Runtime Complexity]
	Let $\mathcal{R}$ be a runtime approximation for $T$.
	Then it holds that 
	\[ \sum_{t \in T}\mathcal{R}_\sqcup(t) \leq \mathit{rc}_\mathit{best} \leq \mathit{rc}_\mathit{worst} \leq \sum_{t \in T}\mathcal{R}_\sqcap(t) \]
\end{remark}

\subsection{Size Bounds}

\begin{definition}[Size Bounds]
	Let 
	\[ S_{t, v, \sigma_0} = \lbrace \sigma'(v) \mid \exists l', \sigma': (l_0, \sigma_0) (\rightarrow^* \circ \rightarrow_t) (l', \sigma') \rbrace \]
	denote the set of all possible values of a variable $v \in V$ directly after a transition $t \in T$ if started with a valuation $\sigma_0: V \rightarrow \mathbb{Z}$.
	We define $S_\sqcup, S_\sqcap: T \times V \rightarrow B$ as lower and upper size bound for a variable $v \in V$ at a specific transition $t \in T$ iff
	\[ S_\sqcap(t, v)(\sigma_0) \geq \mathit{sup}(S_{t, v, \sigma_0}) \]
	and 
	\[ S_\sqcup(t, v)(\sigma_0) \leq \mathit{inf}(S_{t, v, \sigma_0}) \]
\end{definition}

\begin{definition}[Local Size Bounds]
	Let
	\[ S_{t, v, \sigma} = \lbrace \sigma'(v) \mid \exists l, l', \sigma': (l, \sigma) \rightarrow_t (l', \sigma') \rbrace \]
	denote the set of possible values of a variable $v \in V$ after a single transition $t \in T$ with the previous values $\sigma$.
	We define $\mathcal{S}_{l,\sqcap}, \mathcal{S}_{l,\sqcup}: T \times V \rightarrow B$ as local lower and upper size bounds for a variable $v \in V$ and a transition $t \in T$ iff
	\[ \mathcal{S}_{l,\sqcap}(t, v)(\sigma) \geq \mathit{sup}(S_{t, v, \sigma}) \]
	and
	\[ \mathcal{S}_{l,\sqcup}(t, v)(\sigma) \leq \mathit{inf}(S_{t, v, \sigma}) \]	
\end{definition}

\section{Computing Bounds}

\subsection{Computing Runtime Bounds}

\begin{definition}[Safe operator for polynomial ranking function]
	Let $S = (S_\sqcup, S_\sqcap)$ be a size approximation and $\mathit{Pol}$ a polynomial ranking function.
	Let $V_\sqcup = \lbrace v_\sqcup \mid v \in V \rbrace$ be the annotated variables denoting their lower bounds and $V_\sqcap$ their counterparts.
	Let 
	\[ \mathit{safe}: \mathbb{Z}[V] \rightarrow \mathbb{Z}[V_\sqcup \cup V_\sqcap] \]
	be a function which transforms a polynomial ranking function to a safe version iff for all locations $l \in L$, transitions $t \in T$ and according valuations $\sigma: V \rightarrow \mathbb{Z}$ it holds that
	\[ 
		\mathit{safe}(\mathit{Pol}(l))(\sigma
			\left[ \begin{array}{c}
				v_\sqcup \rightarrow S_\sqcup(t, v)\\
			    v_\sqcap \rightarrow S_\sqcap(t, v)
			\end{array} \right] )
		\geq
		\mathit{safe}(\mathit{Pol}(l))(\sigma
			\left[ \begin{array}{c}
			    v_\sqcup \rightarrow \sigma(v)\\
				v_\sqcap \rightarrow \sigma(v)
			\end{array} \right] )
	\]
\end{definition}

\begin{remark}
    \todo{Definition of the suitable $safe$-operator}
\end{remark}

\begin{theorem}[TimeBounds]
	Let $((R_\sqcup, R_\sqcap), (S_\sqcup, S_\sqcap))$ be a complexity approximation.
	Let $T' \subseteq T$ such that T' contains no initial transition.
	Let $\mathit{Pol}$ be a polynomial ranking function for $T'$.
	Let $\mathit{safe}$ be a safe operator for polynomial ranking functions.
	Let $T_{l_{\mathit{in}}} = \lbrace (l, \tau, l_{\mathit{in}}) \mid \exists l, \tau: (l, \tau, l_{\mathit{in}}) \in T \setminus T' \rbrace$ denote the set of all transitions leading to an $l_{\mathit{in}} \in T$.
	Let $\mathcal{E}_{T'} = \lbrace l_{\mathit{in}} \mid T_{l_{\mathit{in}}} \neq \emptyset \wedge \exists l': (l_{\mathit{in}}, \tau, l') \in T' \rbrace$ denote the set of all entry locations of $T'$.
	Let
	\[ \mathcal{R}'(t') = \sum_{l \in \mathcal{E}_{T'}} \sum_{t \in T_l} \mathcal{R}(t) *
		\mathit{safe}(\mathit{Pol}(l))		
			\left( \begin{array}{c}
				S_\sqcup(t, v_1), \dots, S_\sqcup(t, v_n),\\
				S_\sqcap(t, v_1), \dots, S_\sqcap(t, v_n)
			\end{array} \right)
	\]
	for all $t' \in T'_>$ and let $\mathcal{R}'(t')=\mathcal{R}(t')$ for all $t \in T \setminus T'_>$.
	Then $\mathit{TimeBounds}(\mathcal{R}, \mathcal{S}, T') = \mathcal{R}'$ is also a runtime approximation.
\end{theorem}

\subsection{Computing Size Bounds}

\begin{theorem}[SizeBounds for trivial SCCs]
	Let $((R_\sqcup, R_\sqcap), (S_\sqcup, S_\sqcap))$ be a complexity approximation.
	Let $S_{l} = (S_{l, \sqcup}, S_{l, \sqcap})$ be a local size bound.
	Let RVG be a result variable graph.
	Let $\lbrace \alpha \rbrace \subseteq T \times V$ be a trivial SCC of the RVG.
	We define $S'(\alpha') = S(\alpha')$ for all $\alpha' \neq \alpha$ and
	\[ \]
\end{theorem}

\subsection{Result Variable Graph}

\begin{definition}[Pre-Transitions] 
	We define $\mathit{pre}: T \rightarrow 2^T$ as $\mathit{pre}(t') = \lbrace t \in T \mid \exists \sigma_0, l, \sigma: (l_0, \sigma_0) \rightarrow^* \circ \rightarrow_t \circ \rightarrow_{t'} (l, \sigma) \rbrace$ to denote the set of all transitions that may precede $t'$ in an evaluation.	
\end{definition}

\begin{remark}[Trivial overapproximation for pre-transitions] 
	If $\mathit{pre}(t)$ represents the set of all transitions that may precede $t$ in an evaluation, then $\mathit{pre}_{\mathit{trivial}}((l_1, \tau_1, l_1')) = \lbrace (l_2, \tau_2, l_2') \in T \mid l_2' = l_1  \rbrace$ is a valid overapproximation.
\end{remark}

\begin{definition}[Active variables] 
	We define $\mathit{actV}: B \rightarrow 2^V$ as 
	\[ \mathit{actV}(f) = \lbrace v_i \in V \mid \exists m_1, \dots, m_n, m_i' \in \mathbb{Z}: f(m_1, \dots, m_i, \dots, m_n) \neq f(m_1, \dots, m_i', \dots, m_n) \rbrace \]
	to denote the set of active variables in $f$.
\end{definition}

\begin{definition}[Result Variable Graph]
	Let $\mathcal{S}_{l} = (\mathcal{S}_{l,\sqcup}, \mathcal{S}_{l,\sqcap})$ be a local size bound.
	We define 
	\[ RVG = (T \times V, \lbrace ((t, v), (t', v')) \mid t \in \mathit{pre}(t'), v \in \mathit{actV}(S_{l,\sqcup}(t,v)) \cup \mathit{actV}(S_{l,\sqcap}(t,v)) \rbrace) \]
	to denote the result variable graph.
\end{definition}

\end{document}
