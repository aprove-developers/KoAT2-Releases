To increase readability, we only regard the specialized case of approximated evaluations.
The general proof for approximated replacements is analog.

Let $K$ be the set $\bar{\mathbb{Z}}$.
We have to show that $\exacteval{\lsubst{b}{\lstate}{\ustate}}{\valuation} \leq \exacteval{b}{\valuation} \leq \exacteval{\usubst{b}{\lstate}{\ustate}}{\valuation}$ holds for all states $\valuation, \lstate, \ustate \in \Valuation$ with $\lstate \leq \valuation \leq \ustate$.
That is, if $\leval{b}{\lstate}{\ustate} \leq \exacteval{b}{\valuation} \leq \ueval{b}{\lstate}{\ustate}$ holds for all states $\valuation, \lstate, \ustate \in \Valuation$ with $\lstate \leq \valuation \leq \ustate$.
We show that the proposition holds by structural induction.

First, we show that the proposition holds for the minimal structures.
For $\infty$ it holds that $\leval{\infty}{\lstate}{\ustate} = \ueval{\infty}{\lstate}{\ustate} = \exacteval{\infty}{\valuation} = \infty$.
Also, for a $k \in \BoundSet(\VSet)$ it holds that $\leval{k}{\lstate}{\ustate} = \ueval{k}{\lstate}{\ustate} = \exacteval{k}{\valuation} = k$.
For a variable $v \in \BoundSet(\VSet)$ we have $\leval{v}{\lstate}{\ustate} = \exacteval{v}{\lstate} \leq \exacteval{v}{\valuation} \leq \exacteval{v}{\ustate} = \ueval{v}{\lstate}{\ustate}$, since $\exacteval{v}{\lstate} \leq \exacteval{v}{\valuation} \leq \exacteval{v}{\ustate}$ holds by definition of $\lstate$, $\ustate$ and $\valuation$.

Now, we show that it also holds for the immediate substructures.

First, we regard negation.
Lets assume that $\leval{b}{\lstate}{\ustate} \leq \exacteval{b}{\valuation} \leq \ueval{b}{\lstate}{\ustate}$ holds for a bound $b \in \BoundSet(\VSet)$.
Then, we can conclude $-\leval{b}{\lstate}{\ustate} \geq -\exacteval{b}{\valuation} \geq -\ueval{b}{\lstate}{\ustate}$.
Then, $\leval{-b}{\lstate}{\ustate} = -\ueval{b}{\lstate}{\ustate} \leq -\exacteval{b}{\valuation} = \exacteval{-b}{\valuation}$ and also $\ueval{-b}{\lstate}{\ustate} = -\leval{b}{\lstate}{\ustate} \geq -\exacteval{b}{\valuation} = \exacteval{-b}{\valuation}$.

Second, we regard addition.
Lets assume that $\leval{b}{\lstate}{\ustate} \leq \exacteval{b}{\valuation} \leq \ueval{b}{\lstate}{\ustate}$ holds for two bounds $b \in \braced{b_1,b_2} \subseteq \BoundSet(\VSet)$.
Then, $\ueval{b_1 + b_2}{\lstate}{\ustate} = \ueval{b_1}{\lstate}{\ustate} + \ueval{b_2}{\lstate}{\ustate} \geq \exacteval{b_1}{\valuation} + \exacteval{b_2}{\valuation} = \exacteval{b_1 + b_2}{\valuation}$ as well as $\leval{b_1 + b_2}{\lstate}{\ustate} = \leval{b_1}{\lstate}{\ustate} + \leval{b_2}{\lstate}{\ustate} \leq \exacteval{b_1}{\valuation} + \exacteval{b_2}{\valuation} = \exacteval{b_1 + b_2}{\valuation}$.

Third, we regard multiplication.
Lets assume that $\leval{b}{\lstate}{\ustate} \leq \exacteval{b}{\valuation} \leq \ueval{b}{\lstate}{\ustate}$ holds for two bounds $b \in \braced{b_1,b_2} \subseteq \BoundSet(\VSet)$.
Let $M$ denote the set of multiplication combinations of $b_1$ and $b_2$.
\[ M = \braced{
  \ueval{b_1}{\lstate}{\ustate} \cdot \ueval{b_2}{\lstate}{\ustate},
  \ueval{b_1}{\lstate}{\ustate} \cdot \leval{b_2}{\lstate}{\ustate},
  \leval{b_1}{\lstate}{\ustate} \cdot \ueval{b_2}{\lstate}{\ustate},
  \leval{b_1}{\lstate}{\ustate} \cdot \leval{b_2}{\lstate}{\ustate}
} \]
We have $\leval{b_1 \cdot b_2}{\lstate}{\ustate} = \min M$ and $\ueval{b_1 \cdot b_2}{\lstate}{\ustate} = \max M$.
We have to show that $\min M \leq \exacteval{b_1 \cdot b_2}{\valuation} \leq \max M$ holds.
For two integers $k_1, k_2 \in \mathbb{Z}$ it always holds that \[ \max \braced{k_1 \cdot k_2, -k_1 \cdot k_2, k_1 \cdot -k_2, -k_1 \cdot -k_2} \geq k_1 \cdot k_2 \geq \min \braced{k_1 \cdot k_2, -k_1 \cdot k_2, k_1 \cdot -k_2, -k_1 \cdot -k_2}. \]
Therefore, $\ueval{b_1 \cdot b_2}{\lstate}{\ustate} = \max M \geq \exacteval{b_1 \cdot b_2}{\valuation}$ as well as $\leval{b_1 \cdot b_2}{\lstate}{\ustate} = \min M \leq \exacteval{b_1 \cdot b_2}{\valuation}$.

Fourth, we regard the maximum operator.
Lets assume that $\leval{b}{\lstate}{\ustate} \leq \exacteval{b}{\valuation} \leq \ueval{b}{\lstate}{\ustate}$ holds for two bounds $b \in \braced{b_1,b_2} \subseteq \BoundSet(\VSet)$.
Then, $\ueval{\max(b_1,b_2)}{\lstate}{\ustate} = \max(\ueval{b_1}{\lstate}{\ustate}, \ueval{b_2}{\lstate}{\ustate}) \geq \max(\exacteval{b_1}{\valuation}, \exacteval{b_2}{\valuation}) = \exacteval{\max(b_1, b_2)}{\valuation}$ as well as $\leval{\max(b_1, b_2)}{\lstate}{\ustate} = \max(\leval{b_1}{\lstate}{\ustate}, \leval{b_2}{\lstate}{\ustate}) \leq \max(\exacteval{b_1}{\valuation}, \exacteval{b_2}{\valuation}) = \exacteval{\max(b_1, b_2)}{\valuation}$.

Fifth, we regard exponentiation.
Lets assume that $\leval{b}{\lstate}{\ustate} \leq \exacteval{b}{\valuation} \leq \ueval{b}{\lstate}{\ustate}$ holds for a bound $b \in \BoundSet(\VSet)$ and consider an integer $k > 0$.
Then, $\ueval{k^b}{\lstate}{\ustate} = k^{\ueval{b}{\lstate}{\ustate}} \geq k^{\exacteval{b}{\valuation}} = \exacteval{k^b}{\valuation}$ as well as $\leval{k^b}{\lstate}{\ustate} = k^{\leval{b}{\lstate}{\ustate}} \geq k^{\exacteval{b}{\valuation}} = \exacteval{k^b}{\valuation}$.
