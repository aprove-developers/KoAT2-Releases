Let $\rv = (t,v) \in \RV$ be a result variable.
We need to show that if there exists a lower local size bound $\LLSB(\rv) \in \BoundSet^\sqcup_l$, then also $b^\sqcup_\rv$ is a lower local size bound.
Also, we need to show that if there exists an upper local size bound $\ULSB(\rv) \in \BoundSet^\sqcap_l$, then also $b^\sqcap_\rv$ is an upper local size bound.
Lets first regard the lower case.
We assume there is a lower local size bound $\LLSB(\rv) \in \BoundSet^\sqcup_l$.
By definition the lower local size bound $\LLSB(\rv)$ is of the following form.
\[ \LLSB(\rv) = s \cdot \left(e + \sum_{v \in P_1} v - \sum_{v \in N_1} v - \sum_{v \in P_2} \maxO{-v} - \sum_{v \in N_2} \maxO{v} \right) \]
The trivial lower local size bound $b^\sqcup_l$ is of the following form.
\begin{align*}
  b^\sqcup_l = & s_t \cdot \left(e_t - \sum_{v \in \VSet} \maxO{-v} - \sum_{v \in \VSet} \maxO{v} \right) \\
  = & s_t \cdot \left(e_t - \sum_{v \in \VSet} \abs{v} \right)
\end{align*}
By definition the sets $P_1$, $P_2$, $N_1$ and $N_2$ are pairwise disjoint subsets of the variable set $\VSet$.
Therefore, each variable $v \in \VSet$ is at most element of one of the sets $P_1$, $P_2$, $N_1$ and $N_2$.
If $-\abs{v}$ is smaller than $v$, $-v$, $-\abs{-v}$ and $-\abs{v}$, then $\sum_{v \in \VSet} \abs{v}$ is also smaller than $\sum_{v \in P_1} v - \sum_{v \in N_1} v - \sum_{v \in P_2} \maxO{-v} - \sum_{v \in N_2} \maxO{v}$.
We can easily observe that $-\abs{v}$ is indeed smaller than $v$, $-v$, $-\abs{-v}$ and $-\abs{v}$.
Therefore, it is left to show, that $e_t$ is smaller than $e$ and $s_t$ is greater than $s$.
\todo{Can we show this with \constant?}{}
