We only regard the case where $\alpha \in \ScaledSum$ holds for all $\alpha \in \SCC$.
Then, for all result variables $\hat{\alpha} = (\hat{t}, \hat{v}) \in \RV$, we have to show that two statements hold for all $m \in \mathbb{N}^n$.
\[ \USize'(\hat{\alpha})(m) \geq \UpperSizeComplexity{\hat{t}}{\hat{v}} \]
\[ \LSize'(\hat{\alpha})(m) \leq \LowerSizeComplexity{\hat{t}}{\hat{v}} \]
To this end, we now fix $m$ to an arbitrary value and consider a fixed valuation $\abs{\sigma_0} \leq m$ and a fixed evaluation
\[ \configuration{0} \valuateto{\hat{t}} (\hat{\location}, \hat{\valuation}). \]
Our goal is to show that the defined sizebounds are correct lower and upper bounds for each variable $v \in \VSet$.
\begin{equation} \label{eq:sizebound_target}
  \USize'(\hat{t}, v)(m) \geq \hat{\valuation}(v) \geq \LSize'(\hat{t}, v)(m).
\end{equation}

For any transition $t \in \TSet$, let $k_t$ be the number of times that the transition t was used in the evaluation.
To simplify the remaining proof, we define the following values:
\[ d^\square = \prod_{t \in \TSet} (s^\square_t)^{k_t} \cdot \left( e + \sum_{t \in \TSet} \left( k_t \cdot \text{in}^\square_t \right) \right). \]

Then we prove the following claims.
\begin{equation} \label{eq:sizebound_hypothesis}
  \begin{split}
    d^\sqcap(m) & \geq \max(0, \hat{\valuation}(v)) \\
    d^\sqcup(m) & \geq \max(0, -\hat{\valuation}(v))
  \end{split}
\end{equation}
Note that the theorem follows from the claim, as $\UTime(t)(m) \geq k_t$ holds by definition of $k_t$ and $\max(0, \hat{\valuation}(v))$ is always greater or equal than $\hat{\valuation}(v)$ as well as $\max(0, -\hat{\valuation}(v))$ is always greater or equal than $-\hat{\valuation}(v)$.
Also note that by the definition of the local sizebounds $s^\square_t \geq 1$ holds for $\square \in \braced{\sqcup, \sqcap}$.
Therefore for all $p^+_t \geq 0$ and $p^-_t \leq 0$ it holds that
\[ \prod_{t \in \TSet} (s^\sqcap_t)^{\UTime(t)} \cdot \left( e + \sum_{t \in \TSet} \left( \UTime(t)(m) \cdot p^+_t \right) \right) \geq \prod_{t \in \TSet} (s^\square_t)^{k_t} \cdot \left( e + \sum_{t \in \TSet} \left( k_t \cdot p^+_t \right) \right) \]
\[ \prod_{t \in \TSet} (s^\sqcup_t)^{\UTime(t)} \cdot \left( e + \sum_{t \in \TSet} \left( \UTime(t)(m) \cdot p^-_t \right) \right) \leq \prod_{t \in \TSet} (s^\square_t)^{k_t} \cdot \left( e + \sum_{t \in \TSet} \left( k_t \cdot p^-_t \right) \right) \]
We now fix $v$ to an arbitrary variable.
The only interesting case is if $\hat{\alpha} = (\hat{t}, v) \in \SCC$.
Since \todo{Not trivial anymore}{$\text{in}^\sqcap_{\hat{t}}(m) \geq 0 \geq \text{in}^\sqcup_{\hat{t}}(m)$} we can substitute them for $p^+_t$ and $p^-_t$ and get
\begin{equation}
  \begin{split}
    {\mathcal{S}^\sqcap}'(\hat{\alpha})(m) & \geq d^\sqcap(m) \\
    {\mathcal{S}^\sqcup}'(\hat{\alpha})(m) & \leq -d^\sqcup(m)
  \end{split}
\end{equation}

We prove the claim by induction on the length of the evaluation.
Intuitively, we show that we correctly approximate the effect of the last transition step $\hat{t}$ on the size of the value obtained so far (which in turn is captured by the induction hypothesis (\ref{eq:sizebound_target})).

Note that $\hat{t}$ can not be an initial transition, as there are no transitions leading back to the initial location $\location_0$
(i.e., then $\hat{\alpha}$ would not be contained in a nontrivial SCC $\SCC$ of the RVG).
Thus, the reduction has the form
\[ \configuration{0} \valuateto{\tilde{t}} (\tilde{\location}, \tilde{\valuation}) \rightarrow_{\hat{t}} (\hat{\location}, \hat{\valuation}). \]
for some transition $\tilde{t} \in \pre(\hat{t})$.

\todo{Maybe obsolete here}{Note that the value of a non active variable does not affect the value of a local sizebound.}
\begin{equation} \label{eq:nonactiveequalzero}
  \forall v \notin \actV(\ULSB(\alpha)): \valuation(v) = 0
\end{equation}

We have to estimate the sizes of the input variables,
that is, we have to find a bound on the size of each $v \in \VSet$ after the application of the transition $\tilde{t}$ that precedes $\hat{t}$.

If $(\tilde{t}, v) \notin \SCC$, then $\valuation_0 \leq m$ implies that $\USize(\tilde{t}, v)(m) \geq \tilde{\valuation}(v) \geq \LSize(\tilde{t}, v)(m)$.
As $(\tilde{t}, v) \in \pre(\hat{t}, v) \setminus \SCC$, we have $e \geq \USize(\tilde{t}, v)(m)$ and $e \geq 0$ which implies that $d^\sqcap \geq e \geq \USize(\tilde{t}, v)(m) \geq \tilde{\valuation}(v)$.
We also have $e \geq -\LSize(\tilde{t}, v)(m)$ and $e \geq 0$ which implies that $d^\sqcup \geq e \geq -\LSize(\tilde{t}, v)(m) \geq -\tilde{\valuation}(v)$.

If $(\tilde{t}, v) \in \SCC$, then the induction hypothesis (\ref{eq:sizebound_target}) implies that $d^\sqcap(m) \geq \max(0, \tilde{\valuation}(v))$ and $d^\sqcup(m) \geq \max(0, -\tilde{\valuation}(v))$ also hold for the reduction from $\configuration{0}$ to $(\tilde{\location}, \tilde{\valuation})$.
Let
\begin{equation} \label{eq:induction_consequence}
  {d^\square}'(m) = (s^\square_{\hat{t}})^{k_{\hat{t}} - 1} \cdot \prod_{t \in \TSet \setminus \braced{\hat{t}}} (s^\square_t)^{k_t} \cdot \left( e + \left( k_{\hat{t}} - 1 \right) \cdot \text{in}^\square_{\hat{t}}(m) + \sum_{t \in \TSet \setminus \braced{\hat{t}}} \left( k_t \cdot \text{in}^\square_t(m) \right) \right)
\end{equation}
The reason for using $k_{\hat{t}} - 1$ in ${d^\square}'(m)$ is that the last application of the transition $\hat{t}$ in the evaluation is missing in the evaluation from $\configuration{0}$ to $(\tilde{\location}, \tilde{\valuation})$.
Then, the induction hypothesis (\ref{eq:sizebound_target}) implies that ${d^\sqcap}'(m) \geq \max(0, \tilde{\valuation}(v))$ and ${d^\sqcup}'(m) \geq \max(0, -\tilde{\valuation}(v))$.

We now combine these bounds to prove that the induction hypothesis (\ref{eq:sizebound_target}) indeed holds.

Note that the overall scaling factor can be rewritten.
\begin{equation} \label{eq:scaling_factor_reduction}
  \prod_{t \in \TSet} (s^\square_t)^{k_t} = (s^\square_r)^{k_r} \cdot \prod_{t \in \TSet \setminus \braced{r}} (s^\square_t)^{k_t} = s^\square_r \cdot (s^\square_r)^{k_r - 1} \cdot \prod_{t \in \TSet \setminus \braced{r}} (s^\square_t)^{k_t}
\end{equation}
Also the input values can be rewritten.
\begin{equation} \label{eq:input_reduction}
  \begin{split}
  \sum_{t \in \TSet} (k_t \cdot \text{in}^\square_t)
  & = k_{\hat{t}} \cdot \text{in}^\square_{\hat{t}} + \sum_{t \in \TSet \setminus \braced{\hat{t}}} (k_t \cdot \text{in}^\square_t) \\
  & = \text{in}^\square_{\hat{t}}
    + (k_{\hat{t}} - 1) \cdot \text{in}^\square_{\hat{t}}
    + \sum_{t \in \TSet \setminus \braced{\hat{t}}} (k_t \cdot \text{in}^\square_t)
  \end{split}
\end{equation}

For $v \in \VSet \setminus \VSet_{\hat{\alpha}}$, we have $(\tilde{t}, v) \notin \SCC$ (by definition of $\VSet_{\hat{\alpha}}$).
For such $v$, we can deduce the following:
\begin{equation} \label{eq:input_rv_var_reduction}
  \begin{split}
    \text{in}^\sqcap_{\hat{\alpha},v}(m) \geq \max (0, \USize(\tilde{t}, v)(m)) \geq \max (0, \tilde{\valuation}(v)) \\
    \text{in}^\sqcup_{\hat{\alpha},v}(m) \geq \max (0, -\LSize(\tilde{t}, v)(m)) \geq \max (0, -\tilde{\valuation}(v))
  \end{split}
\end{equation}

\todo{Negative valuation for lower bound}{}

We can now propagate this to $\text{in}^\sqcap_{\hat{\alpha}}(m)$.

\begin{IEEEeqnarray*}{rClr} \label{eq:input_rv_reduction}
  & & \text{in}^\sqcap_{\hat{\alpha}}(m) \\
  & = & \sum_{v \in P_{\hat{\alpha}}^\sqcap \setminus \VSet_{\hat{\alpha}}} \text{in}^\sqcap_{{\hat{\alpha}},v}(m)
    + \sum_{v \in N_{\hat{\alpha}}^\sqcap} \text{in}^\sqcup_{{\hat{\alpha}},v}(m) \\
  & = & \sum_{v \in P_{\hat{\alpha}}^\sqcap \setminus \VSet_{\hat{\alpha}}} \max (0, \USize(\tilde{t}, v)(m))
    + \sum_{v \in N_{\hat{\alpha}}^\sqcap} \max (0, -\LSize(\tilde{t}, v)(m)) \\
  & \geq & \sum_{v \in P_{\hat{\alpha}}^\sqcap \setminus \VSet_{\hat{\alpha}}} \max(0, \tilde{\valuation}(v))
    + \sum_{v \in N_{\hat{\alpha}}^\sqcap} \max(0, -\tilde{\valuation}(v)) \\
  & \geq & \sum_{v \in P_{\hat{\alpha}}^\sqcap} \max(0, \tilde{\valuation}(v))
    - \sum_{v \in \VSet_{\hat{\alpha}}} \max(0, \tilde{\valuation}(v))
    + \sum_{v \in N_{\hat{\alpha}}^\sqcap} \max(0, -\tilde{\valuation}(v))
\end{IEEEeqnarray*}

Thus, we have
{\allowdisplaybreaks
\begin{IEEEeqnarray*}{rClr}
  & & d^\square(m) \\
  & = & \prod_{t \in \TSet} (s^\square_t)^{k_t} \cdot \left( e + \sum_{t \in \TSet} \left( k_t \cdot \text{in}^\square_t(m) \right) \right) \\
  & = & \left( s^\square_{\hat{t}} \cdot (s^\square_{\hat{t}})^{k_{\hat{t}} - 1} \cdot \prod_{t \in \TSet \setminus \braced{\hat{t}}} (s^\square_t)^{k_t} \right) \cdot \\
    && \left( e + \text{in}^\square_{\hat{t}}(m) + \left( k_{\hat{t}} - 1 \right) \cdot \text{in}^\square_{\hat{t}}(m) + \sum_{t \in \TSet \setminus \braced{\hat{t}}} \left( k_t \cdot \text{in}^\square_t(m) \right) \right)
    & (\ref{eq:scaling_factor_reduction}), (\ref{eq:input_reduction}) \\
  & = & s^\square_{\hat{t}} \cdot \left( {d^\square}' + \left( (s^\square_{\hat{t}})^{k_{\hat{t}} - 1} \cdot \prod_{t \in \TSet \setminus \braced{\hat{t}}} (s^\square_t)^{k_t} \right) \cdot \text{in}^\square_{\hat{t}}(m) \right) & (\ref{eq:induction_consequence}) \\
  & \geq & s^\square_{\hat{t}} \cdot \left( {d^\square}' + \text{in}^\square_{\hat{t}}(m) \right) & \text{since } \text{in}^\square_{\hat{t}}(m) \geq 0 \\
  & \geq & s^\square_{\hat{t}} \cdot \left( {d^\square}' + \max \braced{\max(0, \pm e^{\square}_\alpha) + \text{in}^\square_\alpha(m) \mid \alpha \in \SCC_{\hat{t}} } \right) & \text{by definition} \\
  & \geq & s^\square_{\hat{t}} \cdot \left( {d^\square}' + \max(0, \pm e^{\square}_{\hat{\alpha}}) + \text{in}^\square_{\hat{\alpha}}(m) \right) & \text{since } {\hat{\alpha}} \in \SCC_{\hat{t}} \\
  & = & \maximum{s^\square_{\alpha} \mid \alpha \in \SCC_{\hat{t}}} \cdot \\
    && \maximum{\abs{\pre(\alpha) \cap \SCC} \mid \alpha \in \SCC_{\hat{t}}} \cdot \\
    && \left( {d^\square}' + \max(0, \pm e^{\square}_{\hat{\alpha}}) + \text{in}^\square_{\hat{\alpha}}(m) \right) & \text{substitute } s^\square_{\hat{t}} \\
  & \geq & s^\square_{\hat{\alpha}} \cdot \abs{\pre(\hat{\alpha}) \cap \SCC} \cdot \left( {d^\square}' + \max(0, \pm e^\square_{\hat{\alpha}}) + \text{in}^\square_{\hat{\alpha}}(m) \right)
    & \text{since } \hat{\alpha} \in \SCC_{\hat{t}} \\
  & \geq & s^\square_{\hat{\alpha}} \cdot \left( \abs{\pre(\hat{\alpha}) \cap \SCC} \cdot {d^\square}' + \max(0, \pm e^\square_{\hat{\alpha}}) + \text{in}^\square_{\hat{\alpha}}(m) \right) \\
  & \geq & s^\square_{\hat{\alpha}} \cdot \left( \abs{\pre(\hat{\alpha}) \cap \SCC} \cdot {d^\square}' \pm e^{\square}_{\hat{\alpha}} + \text{in}^\square_{\hat{\alpha}}(m) \right) \\
  & \geq & s^\square_{\hat{\alpha}} \cdot \left( \sum_{v \in \VSet_{\hat{\alpha}}} {d^\square}' \pm e^\square_{\hat{\alpha}} + \text{in}^\square_{\hat{\alpha}}(m) \right) \\
  & \geq & s^\square_{\hat{\alpha}} \cdot \left( \sum_{v \in \VSet_{\hat{\alpha}}} \max(0,\pm \tilde{\valuation}(v)) \pm e^\square_{\hat{\alpha}} + \text{in}^\square_{\hat{\alpha}}(m) \right) & (\ref{eq:sizebound_hypothesis}) \\
  & = & s^\square_{\hat{\alpha}} \cdot \left( \pm e^\square_{\hat{\alpha}} + \sum_{v \in P_{\hat{\alpha}}^\sqcap} \max(0, \tilde{\valuation}(v)) + \sum_{v \in N_{\hat{\alpha}}^\sqcap} \max(0, -\tilde{\valuation}(v)) \right)
    & (\ref{eq:input_rv_reduction}) \\
  & \geq & s^\square_{\hat{\alpha}} \cdot ( \pm e^\square_{\hat{\alpha}} + \sum_{v \in P_{\hat{\alpha},1}^\sqcap} \tilde{\valuation}(v) - \sum_{v \in N_{\hat{\alpha},1}^\sqcap} \tilde{\valuation}(v) \\
    && + \sum_{v \in P_{\hat{\alpha},2}^\sqcap} \max \braced{0, \tilde{\valuation}(v)} + \sum_{v \in N_{\hat{\alpha},2}^\sqcap} \max \braced{0, -\tilde{\valuation}(v)} ) \\
  & = & \mathcal{S}^\square_l({\hat{\alpha}})(\tilde{\valuation}) \\
  & \geq & \hat{\valuation}(v)
\end{IEEEeqnarray*}}
