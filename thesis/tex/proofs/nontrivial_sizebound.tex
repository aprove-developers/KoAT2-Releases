We only regard the case where every $\rv \in \SCC$ is bounded by an upper scaled sum and a lower scaled sum.
Then, for all result variables $\actrv = (\actt, v) \in \RV$, we have to show that two statements hold for all initial states $\valuation_0 \in \Valuation$ and variables $v \in \VSet$.
\[ \eval{\USize'(\actt, v)}{m} \geq \sup \braced{ \valuation(v) \mid \exists \location, \valuation: (\location_0, \valuation_0) (\rightarrow^* \circ \rightarrow_\actt) (\location, \valuation)} \]
\[ \eval{\LSize'(\actt, v)}{m} \leq \inf \braced{ \valuation(v) \mid \exists \location, \valuation: (\location_0, \valuation_0) (\rightarrow^* \circ \rightarrow_\actt) (\location, \valuation)} \]
To this end, we now \textbf{fix $\valuation_0$ to an arbitrary initial state} and consider a \textbf{fixed evaluation}.
\[ (\location_0, \valuation_0) (\rightarrow^* \circ \rightarrow_\actt) (\actl, \actstate) \]
Our goal is to show that the defined size bounds are correct lower and upper bounds.
\begin{equation} \label{eq:sizebound_target}
  \forall v \in \VSet: \eval{\USize'(\actt, v)}{m} \geq \actstate(v) \geq \eval{\LSize'(\actt, v)}{m}.
\end{equation}
Instead of proving (\ref{eq:sizebound_target}) directly, we proof a simplified hypothesis, which implies the target (\ref{eq:sizebound_target}).
For any transition $t \in \TSet$, let $k_t$ be the number of times that the transition t was used in the evaluation.
Then, we define an upper bound $d^\sqcap \in \BoundSet$ and a lower bound $d^\sqcup \in \BoundSet$ for the absolute value of any variable which can be reached in the evaluation.
\[ d^\square = \prod_{t \in \TSet_\SCC} (\scale^\square_t)^{k_t} \cdot \left( \start + \sum_{t \in \TSet_\SCC} \left( k_t \cdot \effect^\square_t \right) \right). \]
This enables us to define the hypothesis.
\begin{equation} \label{eq:sizebound_hypothesis}
  \forall v \in \VSet: \eval{d^\sqcap}{m} \geq \maxO{\actstate(v)} \wedge \eval{d^\sqcup}{m} \geq \maxO{-\actstate(v)}
\end{equation}
We now show that the goal (\ref{eq:sizebound_target}) indeed follows from the claim (\ref{eq:sizebound_hypothesis}).
Note that $\eval{\UTime(t)}{m} \geq k_t$ holds by definition of $k_t$.
Also $\maxO{\actstate(v)}$ is always greater or equal than $\actstate(v)$ as well as $\maxO{-\actstate(v)}$ is always greater or equal than $-\actstate(v)$.
Additionally we have $\scale^\sqcap_t \geq 1$ and $\scale^\sqcup_t \geq 1$ by definition of the local size bounds.
Furthermore $\eval{\effect^\sqcap_\actt}{m} \geq 0$, $\eval{\effect^\sqcup_\actt}{m} \geq 0$ and $\start \geq 0$ hold by definition of the theorem.
Because of the resulting monotonicity we can conclude that it is sufficient to proof the hypothesis (\ref{eq:sizebound_hypothesis}). 
\[ \eval{{\mathcal{S}^\sqcap}'(\actt,v)}{m} =
   \prod_{t \in \TSet_\SCC} (\scale^\sqcap_t)^{\eval{\UTime(t)}{m}} \cdot \left( \eval{\start}{m} + \sum_{t \in \TSet_\SCC} \left( \eval{\UTime(t)}{m} \cdot \eval{\effect^\sqcap_t}{m} \right) \right) \geq \eval{d^\sqcap}{m} \]
\[ \eval{{\mathcal{S}^\sqcup}'(\actt,v)}{m} =
   (-1) \cdot \prod_{t \in \TSet_\SCC} (\scale^\sqcup_t)^{\eval{\UTime(t)}{m}} \cdot \left( \eval{\start}{m} + \sum_{t \in \TSet_\SCC} \left( \eval{\UTime(t)}{m} \cdot \eval{\effect^\sqcup_t}{m} \right) \right) \leq \eval{d^\sqcup}{m} \]

We now \textbf{fix $v$ to an arbitrary variable} and define with $\actrv = (\actt, v)$ the result variable, for which we want to find an upper size bound $\USize'(\actrv)$ and a lower size bound $\LSize'(\actrv)$.

If $\actrv \notin \SCC$, then the size bounds stay the same $\USize'(\actrv) = \USize(\actrv)$ and $\LSize'(\actrv) = \LSize(\actrv)$.

We will consider from now a result variable $\actrv \in \SCC$.

We prove the claim (\ref{eq:sizebound_hypothesis}) by induction on the length of the evaluation, using the last transition step $\actt$ as base case.

Note that $\actt$ can not be an initial transition, as there are no transitions leading back to the initial location $\location_0$
(i.e., then $\actrv$ would not be contained in a nontrivial SCC $\SCC$ of the RVG).
Thus, for some transition $\pret \in \pre(\actt)$ we can expect the evaluation to have the following form.
\[ (\location_0, \valuation_0) (\rightarrow^* \circ \rightarrow_\pret) (\prel, \prestate) \rightarrow_\actt (\actl, \actstate). \]

We have to estimate the sizes of the input variables, that is, we have to find an upper bound $\USize(\pret, v)$ and a lower bound $\LSize(\pret, v)$ for each variable $v \in \VSet$ after the application of the transition $\pret$ that precedes $\actt$.

We will first consider the base case of the induction.

For $(\pret, v) \notin \SCC$, we have existing size bounds $\eval{\USize(\pret, v)}{m} \geq \prestate(v) \geq \eval{\LSize(\pret, v)}{m}$.
As $(\pret, v) \in \pre(\actt, v) \setminus \SCC$, we have $\start \geq \eval{\USize(\pret, v)}{m}$ and $\start \geq 0$.
Therefore it holds that $\eval{d^\sqcap}{m} \geq \start \geq \eval{\USize(\pret, v)}{m} \geq \prestate(v)$.
We also have $\start \geq -\eval{\LSize(\pret, v)}{m}$ and $\start \geq 0$.
Therefore it also holds that $\eval{d^\sqcup}{m} \geq \start \geq -\eval{\LSize(\pret, v)}{m} \geq -\prestate(v)$.
This proofs the hypothesis (\ref{eq:sizebound_hypothesis}) for all incoming $(\pret, v) \notin \SCC$.

If $(\pret, v) \in \SCC$, then the induction hypothesis implies that $\eval{d^\sqcap}{m} \geq \maxO{\prestate(v)}$ and $\eval{d^\sqcup}{m} \geq \maxO{-\prestate(v)}$ also hold for the evaluation from $(\location_0, \valuation_0)$ to $(\prel, \prestate)$.

Now we continue with the inductive step.
\begin{equation} \label{eq:induction_consequence}
  \eval{\dpre{\square}}{m} = (\scale^\square_\actt)^{k_\actt - 1} \cdot \prod_{t \in \TSet_\SCC \setminus \braced{\actt}} (\scale^\square_t)^{k_t} \cdot \left( \start + \left( k_\actt - 1 \right) \cdot \eval{\effect^\square_\actt}{m} + \sum_{t \in \TSet_\SCC \setminus \braced{\actt}} \left( k_t \cdot \eval{\effect^\square_t}{m} \right) \right)
\end{equation}
The reason for using $k_\actt - 1$ in $\eval{\dpre{\square}}{m}$ is that the last application of the transition $\actt$ in the evaluation is missing in the evaluation from $(\location_0, \valuation_0)$ to $(\prel, \prestate)$.
Then, the induction hypothesis implies that $\eval{\dpre{\sqcap}}{m} \geq \maxO{\prestate(v)}$ and $\eval{\dpre{\sqcap}}{m} \geq \maxO{-\prestate(v)}$.

We now combine these bounds to prove that the induction hypothesis indeed holds.

Note that the overall scaling factor can be rewritten.
\begin{equation} \label{eq:scaling_factor_reduction}
  \prod_{t \in \TSet_\SCC} (\scale^\square_t)^{k_t} = (\scale^\square_r)^{k_r} \cdot \prod_{t \in \TSet_\SCC \setminus \braced{r}} (\scale^\square_t)^{k_t} = \scale^\square_r \cdot (\scale^\square_r)^{k_r - 1} \cdot \prod_{t \in \TSet_\SCC \setminus \braced{r}} (\scale^\square_t)^{k_t}
\end{equation}
Also the input values can be rewritten.
\begin{equation} \label{eq:input_reduction}
  \begin{split}
  \sum_{t \in \TSet_\SCC} (k_t \cdot \effect^\square_t)
  & = k_\actt \cdot \effect^\square_\actt + \sum_{t \in \TSet_\SCC \setminus \braced{\actt}} (k_t \cdot \effect^\square_t) \\
  & = \effect^\square_\actt
    + (k_\actt - 1) \cdot \effect^\square_\actt
    + \sum_{t \in \TSet_\SCC \setminus \braced{\actt}} (k_t \cdot \effect^\square_t)
  \end{split}
\end{equation}
Note that for all $v \in \VSet \setminus \VSet_\actrv$ we have $\eval{\effect^\sqcap_\actrv(v)}{m} \geq \maxO{\eval{\USize(\pret, v)}{m}}$ and $\eval{\effect^\sqcup_\actrv(v)}{m} \geq \maxO{-\eval{\LSize(\pret, v)}{m}}$.
We can now propagate this to $\eval{\effect^\sqcap_\actrv}{m}$.

\begin{equation} \label{eq:input_rv_reduction_upper}
  \begin{split}
  & \eval{\effect^\sqcap_\actrv}{m} \\
  = & \maxO{e^{\sqcap}_\actrv}
    + \sum_{v \in P_\actrv^\sqcap \setminus \VSet^+_\actrv} \eval{\effect^\sqcap_\actrv(v)}{m}
    + \sum_{v \in N_\actrv^\sqcap \setminus \VSet^-_\actrv} \eval{\effect^\sqcup_\actrv(v)}{m} \\
  = & \maxO{e^{\sqcap}_\actrv}
    + \sum_{v \in P_\actrv^\sqcap \setminus \VSet^+_\actrv} \maxO{\eval{\USize(\pret, v)}{m}}
    + \sum_{v \in N_\actrv^\sqcap \setminus \VSet^-_\actrv} \maxO{-\eval{\LSize(\pret, v)}{m}} \\
  \geq & \maxO{e^{\sqcap}_\actrv}
    + \sum_{v \in P_\actrv^\sqcap \setminus \VSet^+_\actrv} \maxO{\prestate(v)}
    + \sum_{v \in N_\actrv^\sqcap \setminus \VSet^-_\actrv} \maxO{-\prestate(v)} \\
  \end{split}
\end{equation}

\begin{equation} \label{eq:d_upper_reduction}
  \begin{split} 
  & \sum_{v \in \VSet_\actrv} \dpre{\sqcap} \\
  = & \sum_{v \in \VSet^+_\actrv} \dpre{\sqcap} + \sum_{v \in \VSet^-_\actrv} \dpre{\sqcap} \\
  \geq & \sum_{v \in \VSet^+_\actrv} \maxO{\prestate(v)} + \sum_{v \in \VSet^-_\actrv} \maxO{-\prestate(v)} (\ref{eq:sizebound_hypothesis}) \\
  \end{split}      
\end{equation}

It is also possible to propagate the same approach to $\eval{\effect^\sqcup_\actrv}{m}$.

\begin{equation} \label{eq:input_rv_reduction_lower}
  \begin{split}
  & \eval{\effect^\sqcup_\actrv}{m} \\
  = & \maxO{-e^{\sqcup}_\actrv}
    + \sum_{v \in P_\actrv^\sqcup \setminus \VSet^+_\actrv} \eval{\effect^\sqcup_\actrv(v)}{m}
    + \sum_{v \in N_\actrv^\sqcup \setminus \VSet^-_\actrv} \eval{\effect^\sqcap_\actrv(v)}{m} \\
  = & \maxO{-e^{\sqcup}_\actrv}
    + \sum_{v \in P_\actrv^\sqcup \setminus \VSet^+_\actrv} \maxO{-\eval{\LSize(\pret, v)}{m}}
    + \sum_{v \in N_\actrv^\sqcup \setminus \VSet^-_\actrv} \maxO{\eval{\USize(\pret, v)}{m}} \\
  \geq & \maxO{-e^{\sqcup}_\actrv}
    + \sum_{v \in P_\actrv^\sqcup \setminus \VSet^+_\actrv} \maxO{-\prestate(v)}
    + \sum_{v \in N_\actrv^\sqcup \setminus \VSet^-_\actrv} \maxO{\prestate(v)} \\
  \end{split}
\end{equation}

\begin{equation} \label{eq:d_lower_reduction}
  \begin{split} 
  & \sum_{v \in \VSet_\actrv} \dpre{\sqcup} \\
  = & \sum_{v \in \VSet^+_\actrv} \dpre{\sqcup} + \sum_{v \in \VSet^-_\actrv} \dpre{\sqcup} \\
  \geq & \sum_{v \in \VSet^+_\actrv} \maxO{-\prestate(v)} + \sum_{v \in \VSet^-_\actrv} \maxO{\prestate(v)} (\ref{eq:sizebound_hypothesis}) \\
  \end{split}      
\end{equation}

Thus, we have
{\allowdisplaybreaks
\begin{IEEEeqnarray*}{rClr}
  & & \eval{d^\square}{m} \\
  & = & \prod_{t \in \TSet_\SCC} (\scale^\square_t)^{k_t} \cdot \left( \start + \sum_{t \in \TSet_\SCC} \left( k_t \cdot \eval{\effect^\square_t}{m} \right) \right) \\
  & = & \left( \scale^\square_\actt \cdot (\scale^\square_\actt)^{k_\actt - 1} \cdot \prod_{t \in \TSet_\SCC \setminus \braced{\actt}} (\scale^\square_t)^{k_t} \right) \cdot \\
    && \left( \start + \eval{\effect^\square_\actt}{m} + \left( k_\actt - 1 \right) \cdot \eval{\effect^\square_\actt}{m} + \sum_{t \in \TSet_\SCC \setminus \braced{\actt}} \left( k_t \cdot \eval{\effect^\square_t}{m} \right) \right)
    & (\ref{eq:scaling_factor_reduction}), (\ref{eq:input_reduction}) \\
  & = & \scale^\square_\actt \cdot \left( \dpre{\square} + \left( (\scale^\square_\actt)^{k_\actt - 1} \cdot \prod_{t \in \TSet_\SCC \setminus \braced{\actt}} (\scale^\square_t)^{k_t} \right) \cdot \eval{\effect^\square_\actt}{m} \right) & (\ref{eq:induction_consequence}) \\
  & \geq & \scale^\square_\actt \cdot \left( \dpre{\square} + \eval{\effect^\square_\actt}{m} \right) & \text{since } \eval{\effect^\square_\actt}{m} \geq 0 \\
  & \geq & \scale^\square_\actt \cdot \left( \dpre{\square} + \max \braced{\eval{\effect^\square_\rv}{m} \mid \rv \in \SCC_\actt } \right) & \text{by definition} \\
  & \geq & \scale^\square_\actt \cdot \left( \dpre{\square} + \eval{\effect^\square_\actrv}{m} \right) & \text{since } \actrv \in \SCC_\actt \\
  & = & \left( \maximum{s^\square_{\rv} \mid \rv \in \SCC_\actt} \cdot
    \maximum{\abs{\VSet_\rv} \mid \rv \in \SCC_\actt} \right) \cdot
    \left( \dpre{\square} + \eval{\effect^\square_\actrv}{m} \right) & \text{substitute } \scale^\square_\actt \\
  & \geq & s^\square_\actrv \cdot \abs{V_\actrv} \cdot \left( \dpre{\square} + \eval{\effect^\square_\actrv}{m} \right)
    & \text{since } \actrv \in \SCC_\actt \\
  & \geq & s^\square_\actrv \cdot \left( \abs{V_\actrv} \cdot \dpre{\square} + \eval{\effect^\square_\actrv}{m} \right) \\
  & \geq & s^\square_\actrv \cdot \left( \sum_{v \in \VSet_\actrv} \dpre{\square} + \eval{\effect^\square_\actrv}{m} \right) \\
  & \geq & s^\square_\actrv \cdot \left( \maxO{\pm e^\square_\actrv} + \sum_{v \in P_\actrv^\sqcap} \maxO{\prestate(v)} + \sum_{v \in N_\actrv^\sqcap} \maxO{-\prestate(v)} \right)
    & (\ref{eq:d_upper_reduction}, \ref{eq:input_rv_reduction_upper}) \\
  & \geq & s^\square_\actrv \cdot ( \pm e^\square_\actrv + \sum_{v \in P_{\actrv,1}^\sqcap} \prestate(v) - \sum_{v \in N_{\actrv,1}^\sqcap} \prestate(v) \\
    && + \sum_{v \in P_{\actrv,2}^\sqcap} \maxO{\prestate(v)} + \sum_{v \in N_{\actrv,2}^\sqcap} \maxO{-\prestate(v)} ) \\
  & = & \eval{\mathcal{S}^\square_l(\actrv)}{\prestate} \\
  & \geq & \actstate(v)
\end{IEEEeqnarray*}}
