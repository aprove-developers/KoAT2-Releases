We only regard the case where every $\alpha \in \SCC$ is bounded by an upper scaled sum and a lower scaled sum.
Then, for all result variables $\hat{\alpha} = (\hat{t}, \hat{v}) \in \RV$, we have to show that two statements hold for all \todo{Really m here?}{$m \in \Valuation$} with $m(v) \in \mathbb{N}$ for all $v \in \VSet$.
\[ \eval{\USize'(\hat{\alpha})}{m} \geq \sup \braced{ \valuation(\hat{v}) \mid \exists \valuation_0, \location, \valuation: \abs{\valuation_0} \leq m \wedge (\location_0, \valuation_0) (\rightarrow^* \circ \rightarrow_{\hat{t}}) (\location, \valuation)} \]
\[ \eval{\LSize'(\hat{\alpha})}{m} \leq \inf \braced{ \valuation(\hat{v}) \mid \exists \valuation_0, \location \valuation: \abs{\valuation_0} \leq m \wedge (\location_0, \valuation_0) (\rightarrow^* \circ \rightarrow_{\hat{t}}) (\location, \valuation)} \]
To this end, we now fix $m$ to an arbitrary value and consider a fixed state $\abs{\valuation_0} \leq m$ and a fixed evaluation.
\[ (\location_0, \valuation_0) (\rightarrow^* \circ \rightarrow_{\hat{t}}) (\hat{\location}, \hat{\valuation}) \]
Our goal is to show that the defined size bounds are correct lower and upper bounds.
\begin{equation} \label{eq:sizebound_target}
  \forall v \in \VSet: \eval{\USize'(\hat{t}, v)}{m} \geq \hat{\valuation}(v) \geq \eval{\LSize'(\hat{t}, v)}{m}.
\end{equation}
Instead of proving (\ref{eq:sizebound_target}) directly, we proof a simplified hypothesis, which implies the target (\ref{eq:sizebound_target}).
For any transition $t \in \TSet$, let $k_t$ be the number of times that the transition t was used in the evaluation.
Then, we define an upper bound $d^\sqcap \in \BoundSet$ and a lower bound $d^\sqcup \in \BoundSet$ for the absolute value of any variable which can be reached in the evaluation.
\[ d^\square = \prod_{t \in \TSet} (s^\square_t)^{k_t} \cdot \left( e + \sum_{t \in \TSet} \left( k_t \cdot \text{in}^\square_t \right) \right). \]
This enables us to define the hypothesis.
\begin{equation} \label{eq:sizebound_hypothesis}
  \forall v \in \VSet: \eval{d^\sqcap}{m} \geq \maxO{\hat{\valuation}(v)} \wedge \eval{d^\sqcup}{m} \geq \maxO{-\hat{\valuation}(v)}
\end{equation}
We now show that the goal (\ref{eq:sizebound_target}) indeed follows from the claim (\ref{eq:sizebound_hypothesis}).
Note that $\eval{\UTime(t)}{m} \geq k_t$ holds by definition of $k_t$.
Also $\maxO{\hat{\valuation}(v)}$ is always greater or equal than $\hat{\valuation}(v)$ as well as $\maxO{-\hat{\valuation}(v)}$ is always greater or equal than $-\hat{\valuation}(v)$.
Additionally we have $s^\sqcap_t \geq 1$ and $s^\sqcup_t \geq 1$ by definition of the local size bounds.
Furthermore $\eval{\text{in}^\sqcap_{\hat{t}}}{m} \geq 0$, $\eval{\text{in}^\sqcup_{\hat{t}}}{m} \geq 0$ and $e \geq 0$ hold by definition of the theorem.
Because of the resulting monotonicity we can conclude that it is sufficient to proof the hypothesis (\ref{eq:sizebound_hypothesis}). 
\[ \eval{{\mathcal{S}^\sqcap}'(\hat{t},v)}{m} =
   \prod_{t \in \TSet} (s^\sqcap_t)^{\eval{\UTime(t)}{m}} \cdot \left( \eval{e}{m} + \sum_{t \in \TSet} \left( \eval{\UTime(t)}{m} \cdot \eval{\text{in}^\sqcap_t}{m} \right) \right) \geq \eval{d^\sqcap}{m} \]
\[ \eval{{\mathcal{S}^\sqcup}'(\hat{t},v)}{m} =
   (-1) \cdot \prod_{t \in \TSet} (s^\sqcup_t)^{\eval{\UTime(t)}{m}} \cdot \left( \eval{e}{m} + \sum_{t \in \TSet} \left( \eval{\UTime(t)}{m} \cdot \eval{\text{in}^\sqcup_t}{m} \right) \right) \leq \eval{d^\sqcup}{m} \]

We now fix $v$ to an arbitrary variable and define with $\hat{\alpha} = (\hat{t}, v)$ the result variable, for which we want to find an upper size bound $\USize'(\hat{\alpha})$ and a lower size bound $\LSize'(\hat{\alpha})$.

If $\hat{\alpha} \notin \SCC$, then the size bounds stay the same $\USize'(\hat{\alpha}) = \USize(\hat{\alpha})$ and $\LSize'(\hat{\alpha}) = \LSize(\hat{\alpha})$.

We will consider from now a result variable $\hat{\alpha} \in \SCC$.

We prove the claim (\ref{eq:sizebound_hypothesis}) by induction on the length of the evaluation, using the last transition step $\hat{t}$ as base case.

Note that $\hat{t}$ can not be an initial transition, as there are no transitions leading back to the initial location $\location_0$
(i.e., then $\hat{\alpha}$ would not be contained in a nontrivial SCC $\SCC$ of the RVG).
Thus, for some transition $\tilde{t} \in \pre(\hat{t})$ we can expect the evaluation to have the following form.
\[ (\location_0, \valuation_0) (\rightarrow^* \circ \rightarrow_{\tilde{t}}) (\tilde{\location}, \tilde{\valuation}) \rightarrow_{\hat{t}} (\hat{\location}, \hat{\valuation}). \]

We have to estimate the sizes of the input variables, that is, we have to find an upper bound $\USize(\tilde{t}, v)$ and a lower bound $\LSize(\tilde{t}, v)$ for each variable $v \in \VSet$ after the application of the transition $\tilde{t}$ that precedes $\hat{t}$.

We will first consider the base case of the induction.

For $(\tilde{t}, v) \notin \SCC$ we have existing size bounds $\eval{\USize(\tilde{t}, v)}{m} \geq \tilde{\valuation}(v) \geq \eval{\LSize(\tilde{t}, v)}{m}$.
As $(\tilde{t}, v) \in \pre(\hat{t}, v) \setminus \SCC$, we have $e \geq \eval{\USize(\tilde{t}, v)}{m}$ and $e \geq 0$.
Therefore it holds that $\eval{d^\sqcap}{m} \geq e \geq \eval{\USize(\tilde{t}, v)}{m} \geq \tilde{\valuation}(v)$.
We also have $e \geq -\eval{\LSize(\tilde{t}, v)}{m}$ and $e \geq 0$.
Therefore it also holds that $\eval{d^\sqcup}{m} \geq e \geq -\eval{\LSize(\tilde{t}, v)}{m} \geq -\tilde{\valuation}(v)$.
This proofs the hypothesis (\ref{eq:sizebound_hypothesis}) for all incoming $(\tilde{t}, v) \notin \SCC$.

If $(\tilde{t}, v) \in \SCC$, then the induction hypothesis implies that $\eval{d^\sqcap}{m} \geq \maxO{\tilde{\valuation}(v)}$ and $\eval{d^\sqcup}{m} \geq \maxO{-\tilde{\valuation}(v)}$ also hold for the evaluation from $(\location_0, \valuation_0)$ to $(\tilde{\location}, \tilde{\valuation})$.

Now we continue with the inductive step.
\begin{equation} \label{eq:induction_consequence}
  \eval{\dpre{\square}}{m} = (s^\square_{\hat{t}})^{k_{\hat{t}} - 1} \cdot \prod_{t \in \TSet \setminus \braced{\hat{t}}} (s^\square_t)^{k_t} \cdot \left( e + \left( k_{\hat{t}} - 1 \right) \cdot \eval{\text{in}^\square_{\hat{t}}}{m} + \sum_{t \in \TSet \setminus \braced{\hat{t}}} \left( k_t \cdot \eval{\text{in}^\square_t}{m} \right) \right)
\end{equation}
The reason for using $k_{\hat{t}} - 1$ in $\eval{\dpre{\square}}{m}$ is that the last application of the transition $\hat{t}$ in the evaluation is missing in the evaluation from $(\location_0, \valuation_0)$ to $(\tilde{\location}, \tilde{\valuation})$.
Then, the induction hypothesis implies that $\eval{\dpre{\sqcap}}{m} \geq \maxO{\tilde{\valuation}(v)}$ and $\eval{\dpre{\sqcap}}{m} \geq \maxO{-\tilde{\valuation}(v)}$.

We now combine these bounds to prove that the induction hypothesis indeed holds.

Note that the overall scaling factor can be rewritten.
\begin{equation} \label{eq:scaling_factor_reduction}
  \prod_{t \in \TSet} (s^\square_t)^{k_t} = (s^\square_r)^{k_r} \cdot \prod_{t \in \TSet \setminus \braced{r}} (s^\square_t)^{k_t} = s^\square_r \cdot (s^\square_r)^{k_r - 1} \cdot \prod_{t \in \TSet \setminus \braced{r}} (s^\square_t)^{k_t}
\end{equation}
Also the input values can be rewritten.
\begin{equation} \label{eq:input_reduction}
  \begin{split}
  \sum_{t \in \TSet} (k_t \cdot \text{in}^\square_t)
  & = k_{\hat{t}} \cdot \text{in}^\square_{\hat{t}} + \sum_{t \in \TSet \setminus \braced{\hat{t}}} (k_t \cdot \text{in}^\square_t) \\
  & = \text{in}^\square_{\hat{t}}
    + (k_{\hat{t}} - 1) \cdot \text{in}^\square_{\hat{t}}
    + \sum_{t \in \TSet \setminus \braced{\hat{t}}} (k_t \cdot \text{in}^\square_t)
  \end{split}
\end{equation}
Note that for all $v \in \VSet \setminus \VSet_{\hat{\alpha}}$ we have $\eval{\text{in}^\sqcap_{\hat{\alpha},v}}{m} \geq \maxO{\eval{\USize(\tilde{t}, v)}{m}}$ and $\eval{\text{in}^\sqcup_{\hat{\alpha},v}}{m} \geq \maxO{-\eval{\LSize(\tilde{t}, v)}{m}}$.
We can now propagate this to $\eval{\text{in}^\sqcap_{\hat{\alpha}}}{m}$.

\begin{equation} \label{eq:input_rv_reduction_upper}
  \begin{split}
  & \eval{\text{in}^\sqcap_{\hat{\alpha}}}{m} \\
  = & \sum_{v \in P_{\hat{\alpha}}^\sqcap \setminus \VSet^+_{\hat{\alpha}}} \eval{\text{in}^\sqcap_{{\hat{\alpha}},v}}{m}
    + \sum_{v \in N_{\hat{\alpha}}^\sqcap \setminus \VSet^-_{\hat{\alpha}}} \eval{\text{in}^\sqcup_{{\hat{\alpha}},v}}{m} \\
  = & \sum_{v \in P_{\hat{\alpha}}^\sqcap \setminus \VSet^+_{\hat{\alpha}}} \maxO{\eval{\USize(\tilde{t}, v)}{m}}
    + \sum_{v \in N_{\hat{\alpha}}^\sqcap \setminus \VSet^-_{\hat{\alpha}}} \maxO{-\eval{\LSize(\tilde{t}, v)}{m}} \\
  \geq & \sum_{v \in P_{\hat{\alpha}}^\sqcap \setminus \VSet^+_{\hat{\alpha}}} \maxO{\tilde{\valuation}(v)}
    + \sum_{v \in N_{\hat{\alpha}}^\sqcap \setminus \VSet^-_{\hat{\alpha}}} \maxO{-\tilde{\valuation}(v)} \\
  \end{split}
\end{equation}

\begin{equation} \label{eq:d_upper_reduction}
  \begin{split} 
  & \sum_{v \in \VSet_{\hat{\alpha}}} \dpre{\sqcap} \\
  = & \sum_{v \in \VSet^+_{\hat{\alpha}}} \dpre{\sqcap} + \sum_{v \in \VSet^-_{\hat{\alpha}}} \dpre{\sqcap} \\
  \geq & \sum_{v \in \VSet^+_{\hat{\alpha}}} \maxO{\tilde{\valuation}(v)} + \sum_{v \in \VSet^-_{\hat{\alpha}}} \maxO{-\tilde{\valuation}(v)} (\ref{eq:sizebound_hypothesis}) \\
  \end{split}      
\end{equation}

It is also possible to propagate the same approach to $\eval{\text{in}^\sqcup_{\hat{\alpha}}}{m}$.

\begin{equation} \label{eq:input_rv_reduction_lower}
  \begin{split}
  & \eval{\text{in}^\sqcup_{\hat{\alpha}}}{m} \\
  = & \sum_{v \in P_{\hat{\alpha}}^\sqcup \setminus \VSet^+_{\hat{\alpha}}} \eval{\text{in}^\sqcup_{{\hat{\alpha}},v}}{m}
    + \sum_{v \in N_{\hat{\alpha}}^\sqcup \setminus \VSet^-_{\hat{\alpha}}} \eval{\text{in}^\sqcap_{{\hat{\alpha}},v}}{m} \\
  = & \sum_{v \in P_{\hat{\alpha}}^\sqcup \setminus \VSet^+_{\hat{\alpha}}} \maxO{-\eval{\LSize(\tilde{t}, v)}{m}}
    + \sum_{v \in N_{\hat{\alpha}}^\sqcup \setminus \VSet^-_{\hat{\alpha}}} \maxO{\eval{\USize(\tilde{t}, v)}{m}} \\
  \geq & \sum_{v \in P_{\hat{\alpha}}^\sqcup \setminus \VSet^+_{\hat{\alpha}}} \maxO{-\tilde{\valuation}(v)}
    + \sum_{v \in N_{\hat{\alpha}}^\sqcup \setminus \VSet^-_{\hat{\alpha}}} \maxO{\tilde{\valuation}(v)} \\
  \end{split}
\end{equation}

\begin{equation} \label{eq:d_lower_reduction}
  \begin{split} 
  & \sum_{v \in \VSet_{\hat{\alpha}}} \dpre{\sqcup} \\
  = & \sum_{v \in \VSet^+_{\hat{\alpha}}} \dpre{\sqcup} + \sum_{v \in \VSet^-_{\hat{\alpha}}} \dpre{\sqcup} \\
  \geq & \sum_{v \in \VSet^+_{\hat{\alpha}}} \maxO{-\tilde{\valuation}(v)} + \sum_{v \in \VSet^-_{\hat{\alpha}}} \maxO{\tilde{\valuation}(v)} (\ref{eq:sizebound_hypothesis}) \\
  \end{split}      
\end{equation}

Thus, we have
{\allowdisplaybreaks
\begin{IEEEeqnarray*}{rClr}
  & & \eval{d^\square}{m} \\
  & = & \prod_{t \in \TSet} (s^\square_t)^{k_t} \cdot \left( e + \sum_{t \in \TSet} \left( k_t \cdot \eval{\text{in}^\square_t}{m} \right) \right) \\
  & = & \left( s^\square_{\hat{t}} \cdot (s^\square_{\hat{t}})^{k_{\hat{t}} - 1} \cdot \prod_{t \in \TSet \setminus \braced{\hat{t}}} (s^\square_t)^{k_t} \right) \cdot \\
    && \left( e + \eval{\text{in}^\square_{\hat{t}}}{m} + \left( k_{\hat{t}} - 1 \right) \cdot \eval{\text{in}^\square_{\hat{t}}}{m} + \sum_{t \in \TSet \setminus \braced{\hat{t}}} \left( k_t \cdot \eval{\text{in}^\square_t}{m} \right) \right)
    & (\ref{eq:scaling_factor_reduction}), (\ref{eq:input_reduction}) \\
  & = & s^\square_{\hat{t}} \cdot \left( \dpre{\square} + \left( (s^\square_{\hat{t}})^{k_{\hat{t}} - 1} \cdot \prod_{t \in \TSet \setminus \braced{\hat{t}}} (s^\square_t)^{k_t} \right) \cdot \eval{\text{in}^\square_{\hat{t}}}{m} \right) & (\ref{eq:induction_consequence}) \\
  & \geq & s^\square_{\hat{t}} \cdot \left( \dpre{\square} + \eval{\text{in}^\square_{\hat{t}}}{m} \right) & \text{since } \eval{\text{in}^\square_{\hat{t}}}{m} \geq 0 \\
  & \geq & s^\square_{\hat{t}} \cdot \left( \dpre{\square} + \max \braced{\maxO{\pm e^{\square}_\alpha} + \eval{\text{in}^\square_\alpha}{m} \mid \alpha \in \SCC_{\hat{t}} } \right) & \text{by definition} \\
  & \geq & s^\square_{\hat{t}} \cdot \left( \dpre{\square} + \maxO{\pm e^{\square}_{\hat{\alpha}}} + \eval{\text{in}^\square_{\hat{\alpha}}}{m} \right) & \text{since } {\hat{\alpha}} \in \SCC_{\hat{t}} \\
  & = & \maximum{s^\square_{\alpha} \mid \alpha \in \SCC_{\hat{t}}} \cdot \\
    && \maximum{\abs{\pre(\alpha) \cap \SCC} \mid \alpha \in \SCC_{\hat{t}}} \cdot \\
    && \left( \dpre{\square} + \maxO{\pm e^{\square}_{\hat{\alpha}}} + \eval{\text{in}^\square_{\hat{\alpha}}}{m} \right) & \text{substitute } s^\square_{\hat{t}} \\
  & \geq & s^\square_{\hat{\alpha}} \cdot \abs{\pre(\hat{\alpha}) \cap \SCC} \cdot \left( \dpre{\square} + \maxO{\pm e^\square_{\hat{\alpha}}} + \eval{\text{in}^\square_{\hat{\alpha}}}{m} \right)
    & \text{since } \hat{\alpha} \in \SCC_{\hat{t}} \\
  & \geq & s^\square_{\hat{\alpha}} \cdot \left( \abs{\pre(\hat{\alpha}) \cap \SCC} \cdot \dpre{\square} + \maxO{\pm e^\square_{\hat{\alpha}}} + \eval{\text{in}^\square_{\hat{\alpha}}}{m} \right) \\
  & \geq & s^\square_{\hat{\alpha}} \cdot \left( \abs{\pre(\hat{\alpha}) \cap \SCC} \cdot \dpre{\square} \pm e^{\square}_{\hat{\alpha}} + \eval{\text{in}^\square_{\hat{\alpha}}}{m} \right) \\
  & \geq & s^\square_{\hat{\alpha}} \cdot \left( \sum_{v \in \VSet_{\hat{\alpha}}} \dpre{\square} \pm e^\square_{\hat{\alpha}} + \eval{\text{in}^\square_{\hat{\alpha}}}{m} \right) \\
  & \geq & s^\square_{\hat{\alpha}} \cdot \left( \pm e^\square_{\hat{\alpha}} + \sum_{v \in P_{\hat{\alpha}}^\sqcap} \maxO{\tilde{\valuation}(v)} + \sum_{v \in N_{\hat{\alpha}}^\sqcap} \maxO{-\tilde{\valuation}(v)} \right)
    & (\ref{eq:d_upper_reduction}, \ref{eq:input_rv_reduction_upper}) \\
  & \geq & s^\square_{\hat{\alpha}} \cdot ( \pm e^\square_{\hat{\alpha}} + \sum_{v \in P_{\hat{\alpha},1}^\sqcap} \tilde{\valuation}(v) - \sum_{v \in N_{\hat{\alpha},1}^\sqcap} \tilde{\valuation}(v) \\
    && + \sum_{v \in P_{\hat{\alpha},2}^\sqcap} \maxO{\tilde{\valuation}(v)} + \sum_{v \in N_{\hat{\alpha},2}^\sqcap} \maxO{-\tilde{\valuation}(v)} ) \\
  & = & \eval{\mathcal{S}^\square_l({\hat{\alpha}})}{\tilde{\valuation}} \\
  & \geq & \hat{\valuation}(v)
\end{IEEEeqnarray*}}
