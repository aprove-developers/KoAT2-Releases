We only regard the case where every result variable $\rv \in \SCC$ is bounded by an upper scaled sum and a lower scaled sum.
That is, if $\ULSB(\rv) \in \BoundSet^\sqcap_l$ and $\LLSB(\rv) \in \BoundSet^\sqcup_l$ for all result variables $\rv \in \SCC$.
Then, for all result variables $\actrv = (\actt, v) \in \RV$, we have to show that two statements hold for all states $\lstate, \ustate \in \Valuation$ and variables $v \in \VSet$.
\[ \ueval{\USize'(\actt, v)}{\lstate}{\ustate} \geq \usizeboundterm \]
\[ \leval{\LSize'(\actt, v)}{\lstate}{\ustate} \leq \lsizeboundterm \]
To this end, we now \textbf{fix $\lstate$, $\ustate$ and $\valuation_0$} to arbitrary initial states with $\lstate \leq \valuation_0 \leq \ustate$ and consider a \textbf{fixed evaluation}.
\[ (\location_0, \valuation_0) (\rightarrow^* \circ \rightarrow_\actt) (\actl, \actstate) \]
Our goal is to show that the defined size bounds are correct lower and upper bounds for each variable $v \in \PVSet$.
\begin{equation} \label{eq:sizebound_target}
  \ueval{\USize'(\actt, v)}{\lstate}{\ustate} \geq \actstate(v) \geq \leval{\LSize'(\actt, v)}{\lstate}{\ustate}
\end{equation}
Instead of proving (\ref{eq:sizebound_target}) directly, we proof a simplified hypothesis, which implies the target (\ref{eq:sizebound_target}).
For any transition $t \in \TSet$, let $k_t$ be the number of times that the transition t was used in the evaluation.
Then, we define an upper bound $d^\sqcap \in \BoundSet(\PVSet)$ and a lower bound $d^\sqcup \in \BoundSet(\PVSet)$ for the absolute value of any variable which can be reached in the evaluation.
For $\square \in \braced{\sqcup, \sqcap}$ we define
\[ d^\square = \prod_{t \in \TSet_\SCC} (\scale^\square_t)^{k_t} \cdot \left( \maxO{\start^\square} + \sum_{t \in \TSet_\SCC} \left( k_t \cdot \effect^\square_t \right) \right). \]
This enables us to define the hypothesis.
\begin{equation} \label{eq:sizebound_hypothesis}
  \ueval{d^\sqcap}{\lstate}{\ustate} \geq \actstate(v) \wedge \ueval{d^\sqcup}{\lstate}{\ustate} \geq -\actstate(v)
\end{equation}
We now show that the goal (\ref{eq:sizebound_target}) indeed follows from the claim (\ref{eq:sizebound_hypothesis}).
Note that $\ueval{\UTime(t)}{\lstate}{\ustate} \geq k_t$ holds by definition of $k_t$.
Additionally, we have $\scale^\sqcap_t \geq 1$ and $\scale^\sqcup_t \geq 1$ by definition of the local size bounds.
Furthermore, the transition effects are always positive (e.g. $\ueval{\effect^\sqcap_\actt}{\lstate}{\ustate} \geq 0$ and $\ueval{\effect^\sqcup_\actt}{\lstate}{\ustate} \geq 0$).
With those properties, we can conclude that it is sufficient to proof the hypothesis (\ref{eq:sizebound_hypothesis}).
First, we can show that $\ueval{{\mathcal{S}^\sqcap}'(\actt,v)}{\lstate}{\ustate} \geq \ueval{d^\sqcap}{\lstate}{\ustate}$ holds.
\begin{align*}
  \ueval{{\mathcal{S}^\sqcap}'(\actt,v)}{\lstate}{\ustate} = &
   \prod_{t \in \TSet_\SCC} (\scale^\sqcap_t)^{\ueval{\UTime(t)}{\lstate}{\ustate}} \cdot \left( \ueval{\maxO{\start^\sqcap}}{\lstate}{\ustate} + \sum_{t \in \TSet_\SCC} \left( \ueval{\UTime(t)}{\lstate}{\ustate} \cdot \ueval{\effect^\sqcap_t}{\lstate}{\ustate} \right) \right) \\
   \geq & \prod_{t \in \TSet_\SCC} (\scale^\sqcap_t)^{k_t} \cdot \left( \ueval{\maxO{\start^\sqcap}}{\lstate}{\ustate} + \sum_{t \in \TSet_\SCC} \left( k_t \cdot \ueval{\effect^\sqcap_t}{\lstate}{\ustate} \right) \right) \\
   = & \ueval{d^\sqcap}{\lstate}{\ustate}
\end{align*}
Second, we can also show that $\leval{{\mathcal{S}^\sqcup}'(\actt,v)}{\lstate}{\ustate} \leq -\ueval{d^\sqcup}{\lstate}{\ustate}$ holds.
\begin{align*}
  \leval{{\mathcal{S}^\sqcup}'(\actt,v)}{\lstate}{\ustate} = &
   \leval{(-1) \cdot \prod_{t \in \TSet_\SCC} (\scale^\sqcup_t)^{\UTime(t)} \cdot \left( \maxO{\start^\sqcup} + \sum_{t \in \TSet_\SCC} \left( \UTime(t) \cdot \effect^\sqcup_t \right) \right)}{\lstate}{\ustate} \\
   = & (-1) \cdot \ueval{\prod_{t \in \TSet_\SCC} (\scale^\sqcup_t)^{\UTime(t)} \cdot \left( \maxO{\start^\sqcup} + \sum_{t \in \TSet_\SCC} \left( \UTime(t) \cdot \effect^\sqcup_t \right) \right)}{\lstate}{\ustate} \\
   = & (-1) \cdot \prod_{t \in \TSet_\SCC} (\scale^\sqcup_t)^{\ueval{\UTime(t)}{\lstate}{\ustate}} \cdot \left( \ueval{\maxO{\start^\sqcup}}{\lstate}{\ustate} + \sum_{t \in \TSet_\SCC} \left( \ueval{\UTime(t)}{\lstate}{\ustate} \cdot \ueval{\effect^\sqcup_t}{\lstate}{\ustate} \right) \right) \\
   \leq & (-1) \cdot \prod_{t \in \TSet_\SCC} (\scale^\sqcup_t)^{k_t} \cdot \left( \ueval{\maxO{\start^\sqcup}}{\lstate}{\ustate} + \sum_{t \in \TSet_\SCC} \left( k_t \cdot \ueval{\effect^\sqcup_t}{\lstate}{\ustate} \right) \right) \\
   = & -\ueval{d^\sqcup}{\lstate}{\ustate}
\end{align*}   

We now \textbf{fix $v$ to an arbitrary variable} and define with $\actrv = (\actt, v)$ the result variable, for which we want to find an upper size bound $\USize'(\actrv)$ and a lower size bound $\LSize'(\actrv)$. If $\actrv \notin \SCC$, then the size bounds stay the same $\USize'(\actrv) = \USize(\actrv)$ and $\LSize'(\actrv) = \LSize(\actrv)$. We will consider from now on a result variable $\actrv \in \SCC$.

We prove the claim (\ref{eq:sizebound_hypothesis}) by induction on the length of the evaluation, using the last transition step $\actt$ as base case.

Note that $\actt$ can not be an initial transition, as there are no transitions leading back to the initial location $\location_0$
(i.e., then $\actrv$ would not be contained in a nontrivial SCC $\SCC$ of the RVG).
Thus, for some transition $\pret \in \pre(\actt)$ we can expect the evaluation to have the following form.
\[ (\location_0, \valuation_0) (\rightarrow^* \circ \rightarrow_\pret) (\prel, \prestate) \rightarrow_\actt (\actl, \actstate). \]

We have to estimate the sizes of the input variables of the transition $\actt$.
That is, we have to find an upper bound $\USize(\pret, v)$ and a lower bound $\LSize(\pret, v)$ for each variable $v \in \PVSet$ after the application of the transition $\pret$ that precedes $\actt$.

We will first consider the base case of the induction.

Consider $(\pret, v) \notin \SCC$.
We have existing size bounds $\ueval{\USize(\pret, v)}{\lstate}{\ustate} \geq \prestate(v) \geq \leval{\LSize(\pret, v)}{\lstate}{\ustate}$.
For the upper case, since $(\pret, v) \in \pre(\actt, v) \setminus \SCC$, we have $\ueval{\maxO{\start^\sqcap}}{\lstate}{\ustate} \geq \ueval{\USize(\pret, v)}{\lstate}{\ustate}$.
Therefore it holds that $\ueval{d^\sqcap}{\lstate}{\ustate} \geq \ueval{\maxO{\start^\sqcap}}{\lstate}{\ustate} \geq \ueval{\USize(\pret, v)}{\lstate}{\ustate} \geq \prestate(v)$.
For the lower case, we have $\ueval{\maxO{\start^\sqcup}}{\lstate}{\ustate} \geq -\leval{\LSize(\pret, v)}{\lstate}{\ustate}$.
Therefore it holds that $\ueval{d^\sqcup}{\lstate}{\ustate} \geq \ueval{\maxO{\start^\sqcup}}{\lstate}{\ustate} \geq -\leval{\LSize(\pret, v)}{\lstate}{\ustate} \geq -\prestate(v)$.
This proves the hypothesis (\ref{eq:sizebound_hypothesis}) for all incoming $(\pret, v) \notin \SCC$.

If $(\pret, v) \in \SCC$, then the induction hypothesis implies that $\ueval{d^\sqcap}{\lstate}{\ustate} \geq \prestate(v)$ and $\ueval{d^\sqcup}{\lstate}{\ustate} \geq -\prestate(v)$ also hold for the evaluation from $(\location_0, \valuation_0)$ to $(\prel, \prestate)$.

Now we continue with the inductive step.
\begin{equation} \label{eq:induction_consequence}
  \dpre{\square} = (\scale^\square_\actt)^{k_\actt - 1} \cdot \prod_{t \in \TSet_\SCC \setminus \braced{\actt}} (\scale^\square_t)^{k_t} \cdot \left( \maxO{\start^\square} + \left( k_\actt - 1 \right) \cdot \effect^\square_\actt + \sum_{t \in \TSet_\SCC \setminus \braced{\actt}} \left( k_t \cdot \effect^\square_t \right) \right)
\end{equation}
The reason for using $k_\actt - 1$ in $\dpre{\square}$ is that the last application of the transition $\actt$ in the evaluation is missing in the evaluation from $(\location_0, \valuation_0)$ to $(\prel, \prestate)$.
Then, the induction hypothesis implies that $\ueval{\dpre{\sqcap}}{\lstate}{\ustate} \geq \prestate(v)$ and $\ueval{\dpre{\sqcup}}{\lstate}{\ustate} \geq -\prestate(v)$.

We now combine these bounds to prove that the induction hypothesis indeed holds.
Note that the overall scaling factor can be rewritten.
\begin{equation} \label{eq:scaling_factor_reduction}
  \prod_{t \in \TSet_\SCC} (\scale^\square_t)^{k_t} = (\scale^\square_r)^{k_r} \cdot \prod_{t \in \TSet_\SCC \setminus \braced{r}} (\scale^\square_t)^{k_t} = \scale^\square_r \cdot (\scale^\square_r)^{k_r - 1} \cdot \prod_{t \in \TSet_\SCC \setminus \braced{r}} (\scale^\square_t)^{k_t}
\end{equation}
Also the loop effect can be rewritten.
\begin{equation} \label{eq:input_reduction}
  \begin{split}
  \sum_{t \in \TSet_\SCC} (k_t \cdot \effect^\square_t)
  & = k_\actt \cdot \effect^\square_\actt + \sum_{t \in \TSet_\SCC \setminus \braced{\actt}} (k_t \cdot \effect^\square_t) \\
  & = \effect^\square_\actt
    + (k_\actt - 1) \cdot \effect^\square_\actt
    + \sum_{t \in \TSet_\SCC \setminus \braced{\actt}} (k_t \cdot \effect^\square_t)
  \end{split}
\end{equation}
Note that by definition of the \textbf{incoming constants}, for all variables $v \in P^\square_\actrv \setminus \VSet_\actrv$ we have $\ueval{\incoming^\sqcap_\actrv(v)}{\lstate}{\ustate} \geq \ueval{\USize(\pret, v)}{\lstate}{\ustate}$.
Also, for all variables $v \in N^\square_\actrv$ we have $\ueval{\incoming^\sqcup_\actrv(v)}{\lstate}{\ustate} \geq \ueval{-\LSize(\pret, v)}{\lstate}{\ustate} = -\leval{\LSize(\pret, v)}{\lstate}{\ustate}$, since $\VSet^-_\actrv = \emptyset$ by definition of the theorem.
By definition of size bounds, we also have $\leval{\LSize(\pret, v)}{\lstate}{\ustate} \leq \prestate(v)$ and therefore $-\leval{\LSize(\pret, v)}{\lstate}{\ustate} \geq -\prestate(v)$.
We can now propagate this to $\ueval{\effect^\sqcap_\actrv}{\lstate}{\ustate}$.

\begin{equation} \label{eq:input_rv_reduction_upper}
  \begin{split}
  & \ueval{\effect^\sqcap_\actrv}{\lstate}{\ustate} \\
  = & \maxO{e^{\sqcap}_\actrv}
    + \sum_{v \in P_\actrv^\sqcap \setminus \VSet_\actrv} \maxO{\ueval{\incoming^\sqcap_\actrv(v)}{\lstate}{\ustate}}
    + \sum_{v \in N_\actrv^\sqcap} \maxO{\ueval{\incoming^\sqcup_\actrv(v)}{\lstate}{\ustate}} \\
  \geq & \maxO{e^{\sqcap}_\actrv}
    + \sum_{v \in P_\actrv^\sqcap \setminus \VSet_\actrv} \maxO{\ueval{\USize(\pret, v)}{\lstate}{\ustate}}
    + \sum_{v \in N_\actrv^\sqcap} \maxO{-\leval{\LSize(\pret, v)}{\lstate}{\ustate}} \\
  \geq & \maxO{e^{\sqcap}_\actrv}
    + \sum_{v \in P_\actrv^\sqcap \setminus \VSet_\actrv} \maxO{\prestate(v)}
    + \sum_{v \in N_\actrv^\sqcap} \maxO{-\prestate(v)} \\
  \end{split}
\end{equation}

It is also possible to propagate the same approach to $\ueval{\effect^\sqcup_\actrv}{\lstate}{\ustate}$.

\begin{equation} \label{eq:input_rv_reduction_lower}
  \begin{split}
  & \ueval{\effect^\sqcup_\actrv}{\lstate}{\ustate} \\
  = & \maxO{-e^{\sqcup}_\actrv}
    + \sum_{v \in P_\actrv^\sqcup \setminus \VSet_\actrv} \maxO{\ueval{\effect^\sqcup_\actrv(v)}{\lstate}{\ustate}}
    + \sum_{v \in N_\actrv^\sqcup} \maxO{\ueval{\effect^\sqcap_\actrv(v)}{\lstate}{\ustate}} \\
  \geq & \maxO{-e^{\sqcup}_\actrv}
    + \sum_{v \in P_\actrv^\sqcup \setminus \VSet_\actrv} \maxO{-\leval{\LSize(\pret, v)}{\lstate}{\ustate}}
    + \sum_{v \in N_\actrv^\sqcup} \maxO{\ueval{\USize(\pret, v)}{\lstate}{\ustate}} \\
  \geq & \maxO{-e^{\sqcup}_\actrv}
    + \sum_{v \in P_\actrv^\sqcup \setminus \VSet_\actrv} \maxO{-\prestate(v)}
    + \sum_{v \in N_\actrv^\sqcup} \maxO{\prestate(v)} \\
  \end{split}
\end{equation}

Thus, we have
{\allowdisplaybreaks
\begin{IEEEeqnarray*}{rClr}
  & & \ueval{d^\sqcap}{\lstate}{\ustate} \\
  & = & \prod_{t \in \TSet_\SCC} (\scale^\sqcap_t)^{k_t} \cdot \left( \ueval{\maxO{\start^\sqcap}}{\lstate}{\ustate} + \sum_{t \in \TSet_\SCC} \left( k_t \cdot \ueval{\effect^\sqcap_t}{\lstate}{\ustate} \right) \right) \\
  & = & \left( \scale^\sqcap_\actt \cdot (\scale^\sqcap_\actt)^{k_\actt - 1} \cdot \prod_{t \in \TSet_\SCC \setminus \braced{\actt}} (\scale^\sqcap_t)^{k_t} \right) \cdot \\
    && \left( \ueval{\maxO{\start^\sqcap}}{\lstate}{\ustate} + \ueval{\effect^\sqcap_\actt}{\lstate}{\ustate} + \left( k_\actt - 1 \right) \cdot \ueval{\effect^\sqcap_\actt}{\lstate}{\ustate} + \sum_{t \in \TSet_\SCC \setminus \braced{\actt}} \left( k_t \cdot \ueval{\effect^\sqcap_t}{\lstate}{\ustate} \right) \right)
    & (\ref{eq:scaling_factor_reduction}), (\ref{eq:input_reduction}) \\
  & = & \scale^\sqcap_\actt \cdot \left( \ueval{\dpre{\sqcap}}{\lstate}{\ustate} + \left( (\scale^\sqcap_\actt)^{k_\actt - 1} \cdot \prod_{t \in \TSet_\SCC \setminus \braced{\actt}} (\scale^\sqcap_t)^{k_t} \right) \cdot \ueval{\effect^\sqcap_\actt}{\lstate}{\ustate} \right) & (\ref{eq:induction_consequence}) \\
  & \geq & \scale^\sqcap_\actt \cdot \left( \ueval{\dpre{\sqcap}}{\lstate}{\ustate} + \ueval{\effect^\sqcap_\actt}{\lstate}{\ustate} \right) & \text{since } \ueval{\effect^\sqcap_\actt}{\lstate}{\ustate} \geq 0 \\
  & \geq & \scale^\sqcap_\actt \cdot \left( \ueval{\dpre{\sqcap}}{\lstate}{\ustate} + \max \braced{\ueval{\effect^\sqcap_\rv}{\lstate}{\ustate} \mid \rv \in \SCC_\actt } \right) & \text{by definition} \\
  & \geq & \scale^\sqcap_\actt \cdot \left( \ueval{\dpre{\sqcap}}{\lstate}{\ustate} + \ueval{\effect^\sqcap_\actrv}{\lstate}{\ustate} \right) & \text{since } \actrv \in \SCC_\actt \\
  & = & \left( \maximum{s^\sqcap_{\rv} \mid \rv \in \SCC_\actt} \cdot
    \maximum{\abs{\VSet_\rv} \mid \rv \in \SCC_\actt} \right) \cdot
    \left( \ueval{\dpre{\sqcap}}{\lstate}{\ustate} + \ueval{\effect^\sqcap_\actrv}{\lstate}{\ustate} \right) & \text{substitute } \scale^\sqcap_\actt \\
  & \geq & s^\sqcap_\actrv \cdot \abs{V_\actrv} \cdot \left( \ueval{\dpre{\sqcap}}{\lstate}{\ustate} + \ueval{\effect^\sqcap_\actrv}{\lstate}{\ustate} \right)
    & \text{since } \actrv \in \SCC_\actt \\
  & \geq & s^\sqcap_\actrv \cdot \left( \abs{V_\actrv} \cdot \ueval{\dpre{\sqcap}}{\lstate}{\ustate} + \ueval{\effect^\sqcap_\actrv}{\lstate}{\ustate} \right) \\
  & \geq & s^\sqcap_\actrv \cdot \left( \sum_{v \in \VSet_\actrv} \ueval{\dpre{\sqcap}}{\lstate}{\ustate} + \ueval{\effect^\sqcap_\actrv}{\lstate}{\ustate} \right) \\
  & \geq & s^\sqcap_\actrv \cdot \left( \sum_{v \in \VSet_\actrv} \ueval{\prestate(v)}{\lstate}{\ustate} + \ueval{\effect^\sqcap_\actrv}{\lstate}{\ustate} \right) \\
  & \geq & s^\sqcap_\actrv \cdot \left( \maxO{e^\sqcap_\actrv} + \sum_{v \in P_\actrv^\sqcap} \maxO{\prestate(v)} + \sum_{v \in N_\actrv^\sqcap} \maxO{-\prestate(v)} \right)
    & (\ref{eq:input_rv_reduction_upper}) \\
  & \geq & s^\sqcap_\actrv \cdot (e^\sqcap_\actrv + \sum_{v \in P_{\actrv,1}^\sqcap} \prestate(v) - \sum_{v \in N_{\actrv,1}^\sqcap} \prestate(v) \\
    && + \sum_{v \in P_{\actrv,2}^\sqcap} \maxO{\prestate(v)} + \sum_{v \in N_{\actrv,2}^\sqcap} \maxO{-\prestate(v)} ) \\
  & = & \exacteval{\mathcal{S}^\sqcap_l(\actrv)}{\prestate} \\
  & \geq & \actstate(v)
\end{IEEEeqnarray*}}

\todo{Copy and modify in the end for the lower case}{}
