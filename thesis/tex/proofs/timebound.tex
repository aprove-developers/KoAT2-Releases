We want to prove the following claim for all \todo{Exclude t0}{$t \in \TSet$} and all $m \in \mathbb{N}^n$.
\[ \eval{\UTime'(t)}{m} \geq \sup \braced{ k \mid \exists k \in \mathbb{N}, \valuation_0, \location, \valuation: (\location_0, \valuation_0) (\rightarrow^* \circ \rightarrow_t)^k (\location, \valuation) } \]

For $m \in \mathbb{N}^n$, let $\USize_m(t)$ be an assignment from each variable $v \in \VSet$ to the approximated value $\eval{\USize(t, v)}{m}$.
It suffices to show that for all $t \in \TSet'_>$ and $m \in \mathbb{N}^n$ the following claim holds.
\begin{IEEEeqnarray*}{rCl}
  \sum_{\location \in \mathcal{E}_{\TSet'}} \sum_{\tilde{t} \in \TSet_l} \eval{\UTime(\tilde{t})}{m} \cdot \maximum{0, \eval{\Pol(\location)_+}{\USize_m(\tilde{t})} - \eval{\Pol(\location)_-}{\LSize_m(\tilde{t})}} \\
  \geq \sup \braced{ k \mid \exists k \in \mathbb{N}, \valuation_0, \location, \valuation: (\location_0, \valuation_0) (\rightarrow^* \circ \rightarrow_t)^k (\location, \valuation) }
\end{IEEEeqnarray*}
To this end, let $\valuation_0$ be a state with $\abs{\valuation_0} \leq m$ and consider a (finite or infinite) evaluation starting with $\valuation_0$ where $k \in \mathbb{N} \cup \braced{\omega}$ steps are performed with the transition $t$.
The goal is now to show that
\begin{IEEEeqnarray*}{rCl}
  \sum_{\location \in \mathcal{E}_{\TSet'}} \sum_{\tilde{t} \in \TSet_l} \eval{\UTime(\tilde{t})}{m} \cdot \maximum{0, \eval{\Pol(\location)_+}{\USize_m(\tilde{t})} - \eval{\Pol(\location)_-}{\LSize_m(\tilde{t})}} \geq k
\end{IEEEeqnarray*}

\todo{Is it?}{This is trivial for $k = 0$.}
We now consider the case where $k > 0$.

Let $\rightarrow^+_\TSet$ denote the transitive closure of $\rightarrow_\TSet$.
Then, we can use the following representation for the considered evaluation.
\begin{IEEEeqnarray*}{lClC}
  (\location_0, \valuation_0) & \rightarrow^+_{\TSet \setminus \TSet'} & (\hat{\location}_1, \hat{\valuation}_1) & \rightarrow^+_{\TSet'} \\
  (\location_1, \valuation_1) & \rightarrow^+_{\TSet \setminus \TSet'} & (\hat{\location}_2, \hat{\valuation}_2) & \rightarrow^+_{\TSet'} \\
  (\location_2, \valuation_2) & \rightarrow^+_{\TSet \setminus \TSet'} & \dots
\end{IEEEeqnarray*}
In this evaluation, the outgoing transitions of all $(\hat{\location}_i, \hat{\valuation}_i)$ are transitions from $\TSet'$.
The outgoing transitions of all $(\location_1, \valuation_1)$ are transitions from $\TSet \setminus \TSet'$.

Now we have to investigate how often the transition $t$ is used in the evaluation.
Since $t \in \TSet'$, it can only be used in sequences of the following form.
\[ (\hat{\location}_i, \hat{\valuation}_i) \rightarrow^+_{\TSet'} (\location_i, \valuation_i) \]
\todo{And we probably should}{We can show that the following term is an upper bound on the number of times the transition $t$ is used in this sequence.}
\[ \maximum{0, \eval{\Pol(\location)_+}{\hat{\valuation}_i} - \eval{\Pol(\location)_-}{\hat{\valuation}_i}} \]

Let the transition in the evaluation reaching $(\hat{\location}_i, \hat{\valuation}_i)$ be $\tilde{t}_i$.
Thus, we have $\hat{\location}_i \in \mathcal{E}_{\TSet'}$ and $\tilde{t}_i \in \TSet_{\hat{\location}_i}$.
As \todo{Check, if that hasn't changed}{$(\location_o, \valuation_o) \rightarrow^*_\TSet \circ \rightarrow_{\tilde{t}_i} (\hat{\location}_i, \hat{\valuation}_i)$} and $\abs{\valuation_0} \leq m$, by definition of the size approximations we have $\eval{\USize(\tilde{t}_i, v)}{m} \geq \hat{\valuation}_i(v) \geq \eval{\LSize(\tilde{t}_i, v)}{m}$.
By the weak monotonicity of $\Pol(\hat{\location}_i)_+$ and $\Pol(\hat{\location}_i)_-$ we can conclude the following two statements.
\begin{IEEEeqnarray*}{lCl}
  \eval{\Pol(\hat{\location}_i)_+}{\USize_m(\tilde{t}_i)} \geq \eval{\Pol(\hat{\location}_i)_+}{\hat{\valuation}_i} \\
  \eval{\Pol(\hat{\location}_i)_-}{\LSize_m(\tilde{t}_i)} \leq \eval{\Pol(\hat{\location}_i)_-}{\hat{\valuation}_i}
\end{IEEEeqnarray*}
Thus, we can conclude an upper bound on the number of times the transition $t$ is used in the sequence.
\[ \maximum{0, \eval{\Pol(\location)_+}{\USize_m(\tilde{t})} - \eval{\Pol(\location)_-}{\LSize_m(\tilde{t})}} \]

It remains to examine how often such a sequence can occur in the full evaluation.
As observed earlier, the transition reaching $\hat{\valuation}_i$ in the evaluation is always some $\tilde{t}_i \in \TSet_{\hat{\location}_i}$.
Each $\tilde{t}_i$ can occur at most $\eval{\UTime(\tilde{t}_i}{m}$ times in evaluations.
As discussed previously, in every $\TSet'$-sequence the transition $t$ can be applied at most $\eval{\Pol(\location)_+}{\hat{\valuation}_i} - \eval{\Pol(\location)_-}{\hat{\valuation}_i}$ times.
Thus, we can infer that our initial statement holds.
\begin{IEEEeqnarray*}{rCl}
  \sum_{\location \in \mathcal{E}_{\TSet'}} \sum_{\tilde{t} \in \TSet_l} \eval{\UTime(\tilde{t})}{m} \cdot \maximum{0, \eval{\Pol(\location)_+}{\USize_m(\tilde{t})} - \eval{\Pol(\location)_-}{\LSize_m(\tilde{t})}} \geq k
\end{IEEEeqnarray*}
