We want to prove the following claim for all transitions $t \in \TSet$ and all initial states $\valuation_0 \in \Valuation$.
\[ \eval{\UTime'(t)}{\valuation_0} \geq \sup \braced{ k \in \mathbb{N} \mid \exists \location, \valuation: (\location_0, \valuation_0) (\rightarrow^* \circ \rightarrow_t)^k (\location, \valuation) } \]

This is trivial, if $t \notin \TSet'_>$.
Therefore, lets consider a transition $t \in \TSet'_>$.
For $\valuation_0 \in \Valuation$ and $\square \in \braced{\sqcup, \sqcap}$, let $\valuation^\square_t: \VSet \rightarrow \mathbb{Z}_\bot$ be a fixed upper or lower size bound $\eval{\GSize(t, v)}{\valuation_0}$ for each variable $v \in \VSet$ at the transition $t$.
It suffices to show that for all $t \in \TSet'_>$ and $\valuation_0: \VSet \rightarrow \mathbb{N}$ the following claim holds.
\begin{IEEEeqnarray*}{rCl}
  \sum_{\location \in \mathcal{E}_{\TSet'}} \sum_{\pret \in \TSet_l} \eval{\UTime(\pret)}{\valuation_0} \cdot \maxO{\subst{\Pol(\location)}{\valuation^\sqcup_\pret}{\valuation^\sqcap_\pret}} \\
  \geq \sup \braced{ k \in \mathbb{N} \mid \exists \location, \valuation: (\location_0, \valuation_0) (\rightarrow^* \circ \rightarrow_t)^k (\location, \valuation) }
\end{IEEEeqnarray*}
To this end, let $\valuation_0$ be an initial state and consider a (finite or infinite) evaluation starting with $\valuation_0$ where $k \in \mathbb{N} \cup \braced{\infty}$ steps are performed with the transition $t$.
The goal is now to show that
\begin{IEEEeqnarray*}{rCl}
  \sum_{\location \in \mathcal{E}_{\TSet'}} \sum_{\pret \in \TSet_l} \eval{\UTime(\pret)}{\valuation_0} \cdot \maxO{\subst{\Pol(\location)}{\valuation^\sqcup_\pret}{\valuation^\sqcap_\pret}} \geq k
\end{IEEEeqnarray*}

This is trivial for $k = 0$, since the operator $\maxO{.}$ ensures positivity and $\eval{\UTime(t)}{\valuation} \geq 0$ holds for any transition $t$ and any state $\valuation$.

We now consider the case $k > 0$.
Let $\rightarrow^+_\TSet$ denote the transitive closure of $\rightarrow_\TSet$.
Then, we can use the following representation for the considered evaluation.
\begin{IEEEeqnarray*}{lClC}
  (\prel_0, \prestate_0) & \rightarrow^+_{\TSet \setminus \TSet'} & (\actl_1, \actstate_1) & \rightarrow^+_{\TSet'} \\
  (\prel_1, \prestate_1) & \rightarrow^+_{\TSet \setminus \TSet'} & (\actl_2, \actstate_2) & \rightarrow^+_{\TSet'} \\
  (\prel_2, \prestate_2) & \rightarrow^+_{\TSet \setminus \TSet'} & \dots
\end{IEEEeqnarray*}
In this evaluation, the outgoing transitions of all $(\actl_i, \actstate_i)$ are transitions from $\TSet'$.
The outgoing transitions of all $(\prel_i, \prestate_i)$ are transitions from $\TSet \setminus \TSet'$.

Now we have to investigate how often the transition $t$ is used in this evaluation.
Since $t \in \TSet'$, it can only be used in sequences of the following form.
\[ (\actl_i, \actstate_i) \rightarrow^+_{\TSet'} (\prel_i, \prestate_i) \]
\todo{Show with Complexities from PRFs proof from old koat}{We can show that $\maxO{\eval{\Pol(\location_i)}{\actstate_i}{}}$ is an upper bound on the number of times the transition $t$ is used in this sequence.}

In this paragraph, lets fix this $i$ and consider the evaluation $(\actl_i, \actstate_i) \rightarrow^+_{\TSet'} (\prel_i, \prestate_i)$ as $(\actl, \actstate) \rightarrow^+_{\TSet'} (\prel, \prestate)$.
Let the transition in the evaluation reaching $(\actl, \actstate)$ be $\pret$.
Thus, we have $\actl \in \mathcal{E}_{\TSet'}$ and $\pret \in \TSet_{\actl}$.
As $(\location_0, \valuation_0) \rightarrow^*_\TSet \circ \rightarrow_{\pret} (\actl, \actstate)$, by definition of the size approximations we have $\valuation^\sqcap_\pret(v) \geq \actstate(v) \geq \valuation^\sqcup_\pret(v)$.
By definition of the substitution of bounds, we can conclude that $\subst{\Pol(\actl)}{\valuation^\sqcup_\pret}{\valuation^\sqcap_\pret} \geq \subst{\Pol(\actl)}{\actstate}{\actstate}$.
Thus, if we ensure positivity, we can conclude an upper bound $\maxO{\subst{\Pol(\location)}{\valuation^\sqcup_\pret}{\valuation^\sqcap_{\pret}}}$ on the number of times the transition $t$ is used in the sequence.

It remains to examine how often a sequence $(\actl_i, \actstate_i) \rightarrow^+_{\TSet'} (\prel_i, \prestate_i)$ can occur in the full evaluation.
As observed earlier, the transition reaching $(\actl_i, \actstate_i)$ in the evaluation is always some $\pret_i \in \TSet_{\actl_i}$.
Each $\pret_i$ can occur at most $\eval{\UTime(\pret_i)}{m}$ times in evaluations.
We found out, that in every $\TSet'$-sequence the transition $t$ can occur at most $\maxO{\subst{\Pol(\location)}{\valuation^\sqcup_\pret}{\valuation^\sqcap_\pret}}$ times.
Thus, we can infer that our initial statement holds.
\begin{IEEEeqnarray*}{rCl}
  \sum_{\location \in \mathcal{E}_{\TSet'}} \sum_{\pret \in \TSet_l} \eval{\UTime(\pret)}{\valuation_0} \cdot \maxO{\subst{\Pol(\location)}{\valuation^\sqcup_\pret}{\valuation^\sqcap_\pret}} \geq k
\end{IEEEeqnarray*}
