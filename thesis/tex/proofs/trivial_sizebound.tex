Let $\rv = (\actt, v) \in \RV$ be a trivial SCC.

We have to show that for all states $\lstate, \ustate \in \Valuation$ and for all variables $v \in \PVSet$ the following two statements hold.
\[ \ueval{\USize'(\actt, v)}{\lstate}{\ustate} \geq \usizeboundterm \]
\[ \leval{\LSize'(\actt, v)}{\lstate}{\ustate} \leq \lsizeboundterm \]
Consider a state $\valuation_0 \in \Valuation$ with $\lstate \leq \valuation_0 \leq \ustate$.
To this end, we consider an evaluation
\[ (\location_0, \valuation_0) (\rightarrow^* \circ \rightarrow_\actt) (\location, \valuation). \]
Now the goal is to show that $\ueval{\USize'(\actt, v)}{\lstate}{\ustate} \geq \valuation(v) \geq \leval{\LSize'(\actt, v)}{\lstate}{\ustate}$ holds.

If $\actt$ is an initial transition $\actt \in \TSet_0$, then the evaluation has the form $(\location_0, \valuation_0) \rightarrow_\actt (\location, \valuation)$, as by definition there are no transitions leading back to the start location $\location_0$.
Thus, for initial transitions, we obtain by the theorem of approximated evaluations $\ueval{\USize'(\actt, v)}{\lstate}{\ustate} = \ueval{\ULSB(\actt, v)}{\lstate}{\ustate} \geq \exacteval{\ULSB(\actt, v)}{\valuation_0} \geq \valuation(v)$ and $\leval{\LSize'(\actt, v)}{\lstate}{\ustate} = \leval{\LLSB(\actt, v)}{\lstate}{\ustate} \leq \exacteval{\LLSB(\actt, v)}{\valuation_0} \leq \valuation(v)$.

Lets now assume that $\actt \in \TSet \setminus \TSet_0$ is not an initial transition.
If $\ULSB(\actt,v) \notin \BoundSet^\sqcap_l$, then $\ueval{\USize'(\actt, v)}{\lstate}{\ustate} = \ueval{\USize(\actt, v)}{\lstate}{\ustate} \geq \exacteval{\USize(\actt, v)}{\valuation_0} \geq \valuation(v)$.
Similar, if $\LLSB(\actt,v) \notin \BoundSet^\sqcup_l$, then $\leval{\LSize'(\actt, v)}{\lstate}{\ustate} = \leval{\LSize(\actt, v)}{\lstate}{\ustate} \leq \exacteval{\LSize(\actt, v)}{\valuation_0} \leq \valuation(v)$.

Otherwise, for the upper case it holds that $\ULSB(\actt,v) \in \BoundSet^\sqcap_l$ and for the lower case it holds that $\LLSB(\actt,v) \in \BoundSet^\sqcup_l$.
Since $\actt \notin \TSet_0$ is not an initial transition, the evaluation has for some transition $\pret \in \pre(\actt)$ the form
\[ (\location_0, \valuation_0) (\rightarrow^* \circ \rightarrow_\pret) (\prel, \prestate) \rightarrow_\actt (\location, \valuation). \]
For all variables $v \in \PVSet$ we have $\ueval{\USize(\pret, v)}{\lstate}{\ustate} \geq \prestate(v) \geq \leval{\LSize(\pret, v)}{\lstate}{\ustate}$.
Thus, $\USize(\pret, v)$ and $\LSize(\pret, v)$ are bounds for the value of all variables $v \in \PVSet$ before the transition $\actt$ is applied.
Then, we can prove that $\ueval{\USize'(\actt, v)}{\lstate}{\ustate}$ is indeed an upper bound for the value of all variables $v \in \PVSet$ after the application of the transition $\actt$.
\begin{align*}
  \ueval{\USize'(\actt, v)}{\lstate}{\ustate} = &
  \maximum{ \ueval{\ULSB(\actt,v)}{\leval{\LSize(\tilde{t})}{\lstate}{\ustate}}{\ueval{\USize(\tilde{t})}{\lstate}{\ustate}} \mid \tilde{t} \in \pre(\actt)} \\
  \geq & \ueval{\ULSB(\actt,v)}{\leval{\LSize(\pret)}{\lstate}{\ustate}}{\ueval{\USize(\pret)}{\lstate}{\ustate}} \\
  \geq & \ueval{\ULSB(\actt,v)}{\prestate}{\prestate} \\
  \geq & \exacteval{\ULSB(\actt,v)}{\prestate} \\
  \geq & \valuation(v)
\end{align*}
Also, $\leval{\LSize'(\actt, v)}{\lstate}{\ustate}$ is a lower bound for the value of all variables $v \in \PVSet$ after the application of the transition $\actt$.
\begin{align*}
  \leval{\LSize'(\actt, v)}{\lstate}{\ustate} = &
  \minimum{ \leval{\LLSB(\actt,v)}{\leval{\LSize(\tilde{t})}{\lstate}{\ustate}}{\ueval{\USize(\tilde{t})}{\lstate}{\ustate}} \mid \tilde{t} \in \pre(\actt)} \\
  \leq & \leval{\LLSB(\actt,v)}{\leval{\LSize(\pret)}{\lstate}{\ustate}}{\ueval{\USize(\pret)}{\lstate}{\ustate}} \\
  \leq & \leval{\LLSB(\actt,v)}{\prestate}{\prestate} \\
  \leq & \exacteval{\LLSB(\actt,v)}{\prestate} \\
  \leq & \valuation(v)
\end{align*}
