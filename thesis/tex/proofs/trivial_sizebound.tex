Let $\rv = (t, v) \in \RV$ be the trivial SCC for which the processor was applied.

We have to show that for all states $\lstate, \ustate \in \Valuation$ and for all $v \in \VSet$ two statements hold.
\[ \exacteval{\USize'(t, v)}{\lstate}{\ustate} \geq \usizeboundterm \]
\[ \exacteval{\USize'(t, v)}{\lstate}{\ustate} \geq \lsizeboundterm \]

Consider a state $\valuation_0 \in \Valuation$ with $\lstate \leq \valuation_0 \leq \ustate$.
To this end, we consider an evaluation
\[ (\location_0, \valuation_0) (\rightarrow^* \circ \rightarrow_t) (\location, \valuation). \]
Now the goal is to show that $\exacteval{\USize'(t, v)}{\lstate}{\ustate} \geq \valuation(v) \geq \exacteval{\LSize'(t, v)}{\lstate}{\ustate}$ holds.

If $t$ is an initial transition, then the evaluation has the form $(\location_0, \valuation_0) \rightarrow_t (\location, \valuation)$, as by definition there are no transitions leading back to the start location $\location_0$ of $t$.

Thus, for initial transitions by the definition of size bounds we obtain
\[ \exacteval{\USize'(t, v)}{\lstate}{\ustate} = \exacteval{\ULSB(t, v)}{\lstate}{\ustate} \geq \valuation(v). \]
\[ \exacteval{\LSize'(t, v)}{\lstate}{\ustate} = \exacteval{\LLSB(t, v)}{\lstate}{\ustate} \leq \valuation(v). \]

In the case where t is not an initial transition, the evaluation has for some transition $\pret \in \pre(t)$ the form
\[ (\location_0, \valuation_0) (\rightarrow^* \circ \rightarrow_\pret) (\prel, \prestate) \rightarrow_t (\location, \valuation). \]

Lets assume that our initial assumption holds.
Then, we have for all $v \in \VSet$
\[ \exacteval{\USize(\pret, v)}{\lstate}{\ustate} \geq \prestate(v) \geq \exacteval{\LSize(\pret, v)}{\lstate}{\ustate}. \]
Thus, $\USize(\pret, v)$ and $\LSize(\pret, v)$ are bounds for the size of all variables $v \in \VSet$ before the transition $t$ is applied.

The only interesting case is when $\rv$ is bounded by a lower scaled sum and by an upper scaled sum.

We can easily see that the following inequations hold.
\[ \sum_{v \in P_{\rv,1}^\sqcap} \USize(\pret, v) \geq \sum_{v \in P_{\rv,1}^\sqcap} \prestate(v) \]
\[ - \sum_{v \in N_{\rv,1}^\sqcap} \LSize(\pret, v) \geq - \sum_{v \in N_{\rv,1}^\sqcap} \prestate(v) \]
\[ \sum_{v \in P_{\rv,2}^\sqcap} \maxO{\USize(\pret, v)} \geq \sum_{v \in P_{\rv,2}^\sqcap} \maxO{\prestate(v)} \]
\[ \sum_{v \in N_{\rv,2}^\sqcap} \maxO{-\LSize(\pret, v)} \geq \sum_{v \in N_{\rv,2}^\sqcap} \maxO{-\prestate(v)} \]
Since $s^\sqcap_\rv \geq 1$ it holds that $\exacteval{\USize(t, v)}{\lstate}{\ustate} \geq \valuation(v)$.

We now show that $\exacteval{\LSize(t, v)}{\lstate}{\ustate} \leq \valuation(v)$ also holds.
Again we can easily see that the following inequations hold.
\[ \sum_{v \in P_{\rv,1}^\sqcup} \LSize(\pret, v) \leq \sum_{v \in P_{\rv,1}^\sqcup} \prestate(v) \]
\[ - \sum_{v \in N_{\rv,1}^\sqcup} \USize(\pret, v) \leq - \sum_{v \in N_{\rv,1}^\sqcup} \prestate(v) \]
\[ - \sum_{v \in P_{\rv,2}^\sqcup} \maxO{-\LSize(\pret, v)} \leq - \sum_{v \in P_{\rv,2}^\sqcup} \maxO{-\prestate(v)} \]
\[ - \sum_{v \in N_{\rv,2}^\sqcup} \maxO{\USize(\pret, v)} \leq - \sum_{v \in N_{\rv,2}^\sqcup} \maxO{\prestate(v)} \]
Since $s^\sqcup_\rv \geq 1$ it holds that $\exacteval{\LSize(t, v)}{\lstate}{\ustate} \leq \valuation(v)$.
