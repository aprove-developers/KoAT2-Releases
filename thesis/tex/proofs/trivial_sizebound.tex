Let $\alpha = (t, v) \in \RV$ be the trivial SCC for which the processor was applied.

We have to show that for all \todo{Really m here?}{$m \in \Valuation$} with $m(v) \in \mathbb{N}$ for all $v \in \VSet$ two statements hold.
\[ \eval{\USize'(t, v)}{m} \geq \sup \braced{ \valuation(v) \mid \exists \valuation_0, \location, \valuation: \abs{\valuation_0} \leq m \wedge (\location_0, \valuation_0) (\rightarrow^* \circ \rightarrow_t) (\location, \valuation)} \]
\[ \eval{\USize'(t, v)}{m} \geq \inf \braced{ \valuation(v) \mid \exists \valuation_0, \location, \valuation: \abs{\valuation_0} \leq m \wedge (\location_0, \valuation_0) (\rightarrow^* \circ \rightarrow_t) (\location, \valuation)} \]

To this end, we consider a state $\abs{\valuation_0} \leq m$ and an evaluation
\[ (\location_0, \valuation_0) (\rightarrow^* \circ \rightarrow_t) (\location, \valuation). \]
Now the goal is to show that $\eval{\USize'(t, v)}{m} \geq \valuation(v) \geq \eval{\LSize'(t, v)}{m}$ holds.

If $t$ is an initial transition, then the evaluation has the form $\valuation_0 \rightarrow_t \valuation$, as by definition there are no transitions leading back to the start location $\location_0$ of $t$.

Thus, for initial transitions by the definition of size bounds we obtain
\[ \eval{\USize'(t, v)}{m} = \eval{\ULSB(t, v)}{m} \geq \valuation(v). \]
\[ \eval{\LSize'(t, v)}{m} = \eval{\LLSB(t, v)}{m} \leq \valuation(v). \]

In the case where t is not an initial transition, the evaluation has for some transition $\tilde{t} \in \pre(t)$ the form
\[ (\location_0, \valuation_0) (\rightarrow^* \circ \rightarrow_{\tilde{t}}) (\tilde{\location}, \tilde{\valuation}) \rightarrow_t (\location, \valuation). \]

Lets assume that our initial assumption holds.
\todo{We havent anymore}{As $\abs{\valuation_0} \leq m$}, we then have for all $v \in \VSet$
\[ \eval{\USize(\tilde{t}, v)}{m} \geq \tilde{\valuation}(v) \geq \eval{\LSize(\tilde{t}, v)}{m}. \]
Thus, $\USize(\tilde{t}, v)$ and $\LSize(\tilde{t}, v)$ are bounds for the size of all variables $v \in \VSet$ before the transition $t$ is applied.

The only interesting case is when $\alpha$ is bounded by a lower scaled sum and by an upper scaled sum.

We can easily see that the following inequations hold.
\[ \sum_{v \in P_{\alpha,1}^\sqcap} \USize(\tilde{t}, v) \geq \sum_{v \in P_{\alpha,1}^\sqcap} \tilde{\valuation}(v) \]
\[ - \sum_{v \in N_{\alpha,1}^\sqcap} \LSize(\tilde{t}, v) \geq - \sum_{v \in N_{\alpha,1}^\sqcap} \tilde{\valuation}(v) \]
\[ \sum_{v \in P_{\alpha,2}^\sqcap} \maxO{\USize(\tilde{t}, v)} \geq \sum_{v \in P_{\alpha,2}^\sqcap} \maxO{\tilde{\valuation}(v)} \]
\[ \sum_{v \in N_{\alpha,2}^\sqcap} \maxO{-\LSize(\tilde{t}, v)} \geq \sum_{v \in N_{\alpha,2}^\sqcap} \maxO{-\tilde{\valuation}(v)} \]
Since $s^\sqcap_\alpha \geq 1$ it holds that $\eval{\USize(t, v)}{m} \geq \valuation(v)$.

We now show that $\eval{\LSize(t, v)}{m} \leq \valuation(v)$ also holds.
Again we can easily see that the following inequations hold.
\[ \sum_{v \in P_{\alpha,1}^\sqcup} \LSize(\tilde{t}, v) \leq \sum_{v \in P_{\alpha,1}^\sqcup} \tilde{\valuation}(v) \]
\[ - \sum_{v \in N_{\alpha,1}^\sqcup} \USize(\tilde{t}, v) \leq - \sum_{v \in N_{\alpha,1}^\sqcup} \tilde{\valuation}(v) \]
\[ - \sum_{v \in P_{\alpha,2}^\sqcup} \maxO{-\LSize(\tilde{t}, v)} \leq - \sum_{v \in P_{\alpha,2}^\sqcup} \maxO{-\tilde{\valuation}(v)} \]
\[ - \sum_{v \in N_{\alpha,2}^\sqcup} \maxO{\USize(\tilde{t}, v)} \leq - \sum_{v \in N_{\alpha,2}^\sqcup} \maxO{\tilde{\valuation}(v)} \]
Since $s^\sqcup_\alpha \geq 1$ it holds that $\eval{\LSize(t, v)}{m} \leq \valuation(v)$.
