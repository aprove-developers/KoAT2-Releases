\section{Computing cost bounds}

In the last two chapters, we introduced algorithms for the computation of time bounds and size bounds.
The algorithm for the computation of time bounds uses the so far computed size bounds and the algorithm for the computation of size bounds uses the so far computed time bounds.
In this chapter, we present a cost bound algorithm, which runs after the consecutive executions of the time and size bound algorithms and uses the resulting time and size bounds for the computation of cost bounds.

A trivial approach for the computation of a cost bound $\UCost(t)$ of a transition $t \in \TSet$ is the multiplication of the time bound $\UTime(t)$ with the maximum of the substituted costs $\usubst{\cost(t)}{\LSize(\pret)}{\USize(\pret)}$ for a pre-transition $\pret \in \pre(t)$.
For the computation of a cost bound $\UCost(t)$ of an initial transition $t \in \TSet_0$ the substitution is not necessary, since an there are no pre-transitions of an initial transition $t \in \TSet_0$ and the variables of the costs $\cost(t)$ are already expressed in dependence on the input variables of the program.

\begin{definition}[Trivial Cost Bound]
  For each transition $t \in \TSet$ we define the cost bound $\UCost: \TSet \rightarrow \BoundSet(\PVSet)$ as trivial if and only if
  \[ \UCost(t) = \UTime(t) \cdot
  \begin{cases}
    \maxO{\cost(t)} & \text{if } t \in \TSet_0 \\
    \maxO{\maximum{\usubst{\cost(t)}{\LSize(\pret)}{\USize(\pret)} \mid \pret \in \pre(t)}} & \text{otherwise }
  \end{cases}
  \]
\end{definition}

The definition uses the $\maxO{\cdot}$ operator to ensure, that the given costs are actually greater than $0$.

\todo{Example necessary?}{}

Additionally to this trivial approach, we use ranking functions, to infer better cost bounds for single transitions.
With a few changes, the TimeBounds algorithm is extendable to the computation of cost bounds.

\begin{theorem}[CostBounds]
  Let $(\UTime, \Size)$ be a complexity approximation. \\
  Let $\UCost$ be a cost bound. \\
  Let $\TSet' \subseteq \TSet \setminus \TSet_0$ a subset of all transitions such that $\TSet'$ contains no initial transitions. \\
  Let $\TSet_{\location} = \braced{(\location', \update, \guard, \location) \mid \exists \location', \update, \guard: (\location', \update, \guard, \location) \in \TSet \setminus \TSet'}$ denote the set of all transitions outside of the subprogram $\TSet'$ leading to an $\location \in \LSet$. \\
  Let $\mathcal{E}_{\TSet'} = \braced{\location_{in} \mid \TSet_{\location_{in}} \neq \emptyset \wedge \exists \location': (\location_{in}, \update, \guard, \location') \in \TSet'}$ denote the set of all entry locations of $\TSet'$. \\
  Let $\costrank: \LSet \rightarrow \BoundSet(\PVSet)$ be a cost ranking function for $\TSet'$. \\
  For $t \in \TSet'_>$ let
  \[ \UCost'(t) = \sum_{\location \in \mathcal{E}_{\TSet'}} \sum_{\pret \in \TSet_\location} \UTime(\pret) \cdot \maxO{\usubst{\costrank(\location)}{\LSize(\pret)}{\USize(\pret)}} \]
  Let $\UCost'(t) = \UCost(t)$ for $t \in \TSet \setminus \TSet'_>$. \\
  Then, $\text{CostBounds}(\UTime, (\USize, \LSize), \TSet') = \UCost'$ is also a cost approximation.
\end{theorem}


With this definition, it is possible to start with a trivial cost bound, search for cost ranking functions and consecutively refine the cost bounds of single transitions.
