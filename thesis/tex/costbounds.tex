\section{Computing cost bounds}

In the last two chapters, we introduced algorithms for the computation of time bounds and size bounds.
The algorithm for the computation of time bounds uses the so far computed size bounds and the algorithm for the computation of size bounds uses the so far computed time bounds.
In this chapter, we present a cost bound algorithm, which runs after the consecutive executions of the time and size bound algorithms and uses the resulting time and size bounds for the computation of cost bounds.

A trivial approach for the computation of a cost bound $\UCost(t)$ of a transition $t \in \TSet$ is the multiplication of the time bound $\UTime(t)$ with the maximum of the substituted costs $\usubst{\cost(t)}{\LSize(\pret)}{\USize(\pret)}$ for a pre-transition $\pret \in \pre(t)$.
For the computation of a cost bound $\UCost(t)$ of an initial transition $t \in \TSet_0$ the substitution is not necessary, since an there are no pre-transitions of an initial transition $t \in \TSet_0$ and the variables of the costs $\cost(t)$ are already expressed in dependence on the input variables of the program.

\begin{definition}[Trivial Cost Bound]
  For each transition $t \in \TSet$ we define the cost bound $\UCost: \TSet \rightarrow \BoundSet(\PVSet)$ as trivial if and only if
  \[ \UCost(t) = \UTime(t) \cdot
  \begin{cases}
    \maxO{\cost(t)} & \text{if } t \in \TSet_0 \\
    \maxO{\maximum{\usubst{\cost(t)}{\LSize(\pret)}{\USize(\pret)} \mid \pret \in \pre(t)}} & \text{otherwise }
  \end{cases}
  \]
\end{definition}

The definition uses the $\maxO{\cdot}$ operator to ensure, that the given costs are actually greater than $0$.

\todo{Example necessary?}{}

Additionally to this trivial approach, we use ranking functions, to infer better cost bounds for single transitions.
With a few changes, the TimeBounds algorithm is extendable to the computation of cost bounds.

We want to prove the following claim for all transitions $t \in \TSet$ and all states $\lstate, \ustate \in \Valuation$.
\begin{align*}
  \ueval{\UCost'(t)}{\lstate}{\ustate} \geq \sup & \braced{ \sum_{1 \leq i \leq k} \exacteval{\cost(t)}{\valuation_i} \mid \exists \valuation_0, k \geq 1: \lstate \leq \valuation_0 \leq \ustate \\
    & \wedge (\location_0, \valuation_0) \rightarrow^* (\location_1, \valuation_1) (\rightarrow_t \circ \rightarrow^*) \dots (\location_k, \valuation_k) (\rightarrow_t \circ \rightarrow^*) (\location_{k+1}, \valuation_{k+1}) }
\end{align*}

This is trivial, if $t \notin \TSet'_>$.
Therefore, lets consider a transition $t \in \TSet'_>$.
It suffices to show that for all transitions $t \in \TSet'_>$ and all states $\lstate \leq \ustate$ the following claim holds.
\begin{align*}
  & \sum_{\location \in \mathcal{E}_{\TSet'}} \sum_{\pret \in \TSet_\location} \ueval{\UTime(\pret)}{\lstate}{\ustate} \cdot \maxO{\usubst{\costrank(\location)}{\leval{\LSize(\pret)}{\lstate}{\ustate}}{\ueval{\USize(\pret)}{\lstate}{\ustate}}} \\
  \geq \sup & \braced{ \sum_{1 \leq i \leq k} \exacteval{\cost(t)}{\valuation_i} \mid \exists \valuation_0, k \geq 1: \lstate \leq \valuation_0 \leq \ustate \\
    & \wedge (\location_0, \valuation_0) \rightarrow^* (\location_1, \valuation_1) (\rightarrow_t \circ \rightarrow^*) \dots (\location_k, \valuation_k) (\rightarrow_t \circ \rightarrow^*) (\location_{k+1}, \valuation_{k+1}) }
\end{align*}
To this end, let $\valuation_0$ be an initial state with $\lstate \leq \valuation_0 \leq \ustate$ and consider a (finite or infinite) evaluation starting with $\valuation_0$ where $k \in \mathbb{N} \cup \braced{\infty}$ steps are performed with the transition $t$.
The goal is now to show that
\[
  \sum_{\location \in \mathcal{E}_{\TSet'}} \sum_{\pret \in \TSet_\location} \ueval{\UTime(\pret)}{\lstate}{\ustate} \cdot \maxO{\usubst{\costrank(\location)}{\leval{\LSize(\pret)}{\lstate}{\ustate}}{\ueval{\USize(\pret)}{\lstate}{\ustate}}} \geq \sum_{1 \leq i \leq k} \exacteval{\cost(t)}{\valuation_i}
\]

This is trivial for $k = 0$.
Then it holds that $\sum_{1 \leq i \leq k} \exacteval{\cost(t)}{\valuation_i} = 0$.
Since the operator $\maxO{\cdot}$ ensures positivity and $\exacteval{\UTime(t)}{\valuation_0} \geq 0$ holds for any transition $t$, the statement holds.

We now consider the case $k > 0$.
Then, we can use the following representation for the considered evaluation.
\begin{IEEEeqnarray*}{lClClClCl}
  (\prel_0, \prestate_0) & \rightarrow^{\tilde{k}_0}_{\TSet \setminus \TSet'} & (\actl_{1,1}, \actstate_{1,1}) & \rightarrow_{\TSet'} & \dots & \rightarrow_{\TSet'} & (\actl_{1,k_1}, \actstate_{1,k_1}) & \rightarrow_{\TSet'} \\
  (\prel_1, \prestate_1) & \rightarrow^{\tilde{k}_1}_{\TSet \setminus \TSet'} & (\actl_{2,1}, \actstate_{2,1}) & \rightarrow_{\TSet'} & \dots & \rightarrow_{\TSet'} & (\actl_{2,k_2}, \actstate_{2,k_2}) & \rightarrow_{\TSet'} \\
  (\prel_2, \prestate_2) & \rightarrow^{\tilde{k}_2}_{\TSet \setminus \TSet'} & \dots
\end{IEEEeqnarray*}
In this evaluation, the outgoing transitions of all $(\actl_{i,h}, \actstate_{i,h})$ are transitions from $\TSet'$.
The outgoing transitions of all $(\prel_i, \prestate_i)$ are transitions from $\TSet \setminus \TSet'$.

Now we have to investigate how often the transition $t$ is used in this evaluation.
Since $t \in \TSet'$, it can only be used in sequences of the following form.
\[ (\actl_{i,1}, \actstate_{i,1}) \rightarrow_{\TSet'} \dots \rightarrow_{\TSet'} (\actl_{i,k_i}, \actstate_{i,k_i}) \rightarrow_{\TSet'} (\prel_i, \prestate_i) \]

We can show that $\maxO{\exacteval{\costrank(\location_{i,1})}{\actstate_{i,1}}} \geq \sum_{1 \leq h \leq {k_i}} \exacteval{\cost(t)}{\valuation_{i,h}}$.
For $k_i = 0$ this is obvious, since then $\sum_{1 \leq h \leq {k_i}} \exacteval{\cost(t)}{\valuation_{i,h}} = 0$ holds and the operator $\maxO{\cdot}$ ensures positivity.
Lets consider the case $k_i > 0$.
Since $\costrank$ is a cost ranking function for $\TSet'$, for all $j$ it holds that $\exacteval{\costrank(\location_{i,j})}{\actstate_{i,j}} \geq \exacteval{\costrank(\location_{i,j+1})}{\actstate_{i,j+1}}$.
Let $j_1 < j_2 < \dots$ be the indices where $t_j = t$.
Then, for all $j \in \braced{j_1, j_2, \dots}$, $t \in \TSet_>$ implies that $\exacteval{\costrank(\location_{i,j})}{\actstate_{i,j}} \geq \exacteval{\costrank(\location_{i,j+1})}{\actstate_{i,j+1}} + \exacteval{\cost(t)}{\actstate_{i,j}}$ and $\exacteval{\costrank(\location_{i,j})}{\actstate_{i,j}} \geq \exacteval{\cost(t)}{\actstate_{i,j}}$.
Thus, we obtain
\begin{align*}
  & \exacteval{\costrank(\location_{i,1})}{\actstate_{i,1}} \\
  \geq & \exacteval{\costrank(\location_{i,j_1})}{\actstate_{i,j_1}} \\
  \geq & \exacteval{\costrank(\location_{i,j_2})}{\actstate_{i,j_2}} + \exacteval{\cost(t)}{\actstate_{i,j_1}} \\
  \geq & \dots \\
  \geq & \exacteval{\costrank(\location_{i,j_{k_i}})}{\actstate_{i,j_{k_i}}} + \exacteval{\cost(t)}{\actstate_{i,j_{k_i-1}}} \\
  \geq & \exacteval{\cost(t)}{\actstate_{i,j_{k_i}}}
\end{align*}
Therefore, $\maxO{\exacteval{\costrank(\location_{i,1})}{\actstate_{i,1}}} \geq \sum_{1 \leq h \leq {k_i}} \exacteval{\cost(t)}{\valuation_{i,h}}$ is an upper bound on the costs of the transition $t$ in the sequence.

Let $\pret_i$ be the transition reaching $(\actl_{i,1}, \actstate_{i,1})$ in the evaluation.
Thus, we have $\actl_{i,1} \in \mathcal{E}_{\TSet'}$ and $\pret_i \in \TSet_{\actl_{i,1}}$.
As $(\location_0, \valuation_0) \rightarrow^*_\TSet \circ \rightarrow_{\pret_i} (\actl_{i,1}, \actstate_{i,1})$ and $\lstate \leq \valuation_0 \leq \ustate$, we have by definition of size bounds
\[ \ueval{\USize(\pret_i, v)}{\lstate}{\ustate} \geq \exacteval{\USize(\pret_i, v)}{\valuation_0} \geq \actstate_{i,1}(v) \geq \exacteval{\LSize(\pret_i, v)}{\valuation_0} \geq \leval{\LSize(\pret_i, v)}{\lstate}{\ustate}. \]
We can conclude
\begin{align*}
   & \maxO{\ueval{\costrank(\actl_{i,1})}{\leval{\LSize(\pret_i)}{\lstate}{\ustate}}{\ueval{\USize(\pret_i)}{\lstate}{\ustate}}} \\
   \geq & \maxO{\ueval{\costrank(\actl_{i,1})}{\exacteval{\LSize(\pret_i)}{\valuation_0}}{\exacteval{\USize(\pret_i)}{\valuation_0}}} \\
   \geq & \maxO{\ueval{\costrank(\actl_{i,1})}{\valuation_{i,1}}{\valuation_{i,1}}} \\
   \geq & \maxO{\exacteval{\costrank(\actl_{i,1})}{\valuation_{i,1}}} \\
   \geq & \sum_{1 \leq h \leq {k_i}} \exacteval{\cost(t)}{\valuation_{i,h}} \\
\end{align*}
Thus, $\maxO{\usubst{\costrank(\actl_{i,1})}{\leval{\LSize(\pret_i)}{\lstate}{\ustate}}{\ueval{\USize(\pret_i)}{\lstate}{\ustate}}}$ is a bound on the cost $\sum_{1 \leq h \leq {k_i}} \exacteval{\cost(t)}{\valuation_{i,h}}$ of the transition $t$ in the sequence.

It remains to examine how often a sequence $(\actl_{i,1}, \actstate_{i,1}) \rightarrow^*_{\TSet'} (\prel_i, \prestate_i)$ can occur in the full evaluation.
As observed earlier, the transition reaching $(\actl_{i,1}, \actstate_{i,1})$ in the evaluation is always some $\pret_i \in \TSet_{\actl_{i,1}}$.
Each $\pret_i$ can occur at most $\ueval{\UTime(\pret_i)}{\lstate}{\ustate}$ times in evaluations.
We found out, that in every $\TSet'$-sequence the transition $t$ has costs of at most $\maxO{\ueval{\costrank(\actl_{i,1})}{\leval{\LSize(\pret_i)}{\lstate}{\ustate}}{\ueval{\USize(\pret_i)}{\lstate}{\ustate}}}$.
Thus, we can infer that our initial statement holds.
\begin{IEEEeqnarray*}{rCl}
  \sum_{\location \in \mathcal{E}_{\TSet'}} \sum_{\pret \in \TSet_\location} \ueval{\UTime(\pret)}{\lstate}{\ustate} \cdot \maxO{\usubst{\costrank(\location)}{\leval{\LSize(\pret)}{\lstate}{\ustate}}{\ueval{\USize(\pret)}{\lstate}{\ustate}}} \geq \sum_{1 \leq i \leq k} \exacteval{\cost(t)}{\valuation_i}
\end{IEEEeqnarray*}


With this definition, it is possible to start with a trivial cost bound, search for cost ranking functions and consecutively refine the cost bounds of single transitions.
