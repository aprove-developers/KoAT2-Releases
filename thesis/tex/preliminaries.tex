\section{Preliminaries}


\subsection{Programs}

\begin{definition}[Program] 
	Let $\VSet$ be the set of all occurring program variables.
	Let $\LSet$ be the set of all locations of a program.
	Let $\TSet = \LSet \times F(\VSet) \times \LSet$ be the set of all transitions of the program where $F(\VSet)$ is the set of all quantifier-free formulas over $\VSet$.
	A program is a graph $P = (\LSet, \TSet)$ where $\LSet$ is the set of program locations and $\TSet$ is a subset of all possible transitions between them.
	We write $\location_0$ for the unique program location with $\location_0 \in \LSet$ which has no entry transitions in the program.
	We write $\location_e$ for a program location $\location_e \in \LSet$ which has no outgoing transitions in the program.
\end{definition}

\begin{definition}[State] 
	A state is a function $\valuation: \VSet \rightarrow \mathbb{Z}$ which assigns each program variable a value.
\end{definition}

\begin{definition}[Evaluation] 
	An evaluation is a function which maps a state with a transition to a result configuration.
	We write $\valuation \rightarrow_t \valuation'$ iff with $t = (\location, \tau, \location') \in \TSet$ it holds that $\valuation \models \tau$ and $\valuation' \models \tau$.
	\todo{Does it make sense without location?}{We omit the transition and write $\valuation \rightarrow \valuation'$ iff there exists a transition $t \in \TSet$ such that $\valuation \rightarrow_t \valuation'$ holds.}
	We write $\valuation \rightarrow^k \valuation'$ iff there exists a sequence of transitions $t_1, \dots, t_k \in \TSet$ such that $\valuation \rightarrow_{t_1} \dots \rightarrow_{t_k} \valuation'$ holds.
	We write $\valuation \rightarrow^* \valuation'$ iff there exists a $k \in \mathbb{N}$ such that $\valuation \rightarrow^k \valuation'$ holds.
	We write $\valuation \rightarrow_{\TSet'} \valuation'$ iff there exists a transition $t \in \TSet' \subseteq \TSet$ such that $\valuation \rightarrow_t \valuation'$ holds.
\end{definition}


\subsection{Result Variable Graph}

\begin{definition}[Pre-Transitions] 
  We define $\pre: \TSet \rightarrow 2^\TSet$ as
  \[\pre(t) = \braced{\tilde{t} \in \TSet \mid \exists \valuation_0, \valuation: \valuation_0 (\rightarrow_{t_0} \circ \rightarrow^* \circ \rightarrow_{\tilde{t}} \circ \rightarrow_{t}) \valuation}\]
  to denote the set of all transitions that may precede $t$ in an evaluation.	
\end{definition}

If $\pre(t)$ represents the set of all transitions that may precede $t$ in an evaluation, then $\pre_{\mathit{trivial}}((\location_1, \tau_1, \location_1')) = \braced{(\location_2, \tau_2, \location_2') \in \TSet \mid \location_2' = \location_1}$ is a valid overapproximation.

\begin{definition}[Active variables] 
	We define $\actV: B \rightarrow 2^\VSet$ as 
	\[ \actV(f) = \braced{ v_i \in \VSet \mid \exists m_1, \dots, m_n, m_i' \in \mathbb{Z}: f(m_1, \dots, m_i, \dots, m_n) \neq f(m_1, \dots, m_i', \dots, m_n)} \]
	to denote the set of active variables in $f$.
\end{definition}

\begin{definition}[Result Variable Graph]
	Let $\LSB$ be a local size bound.
	We define 
	\[ RVG = (\TSet \times \VSet, \braced{((t, v), (t', v')) \mid t \in \pre(t'), v \in \actV(\LLSB(t,v)) \cup \actV(\ULSB(t,v))}) \]
	to denote the result variable graph.
\end{definition}


\subsection{Ranking functions}

\begin{definition}[Polynomial Ranking Function] 
	Let $\mathbb{Z}[\VSet]$ be the polynomial ring in the variables $v \in \VSet$ over the field $\mathbb{Z}$.
	We define $p(\valuation) = p[v_1/\valuation(v_1), \dots, v_n/\valuation(v_n)]$ as the substitution of all $v_1, \dots v_n \in \VSet$ by the state $\valuation: \VSet \rightarrow \mathbb{Z}$ in the polynomial $p \in \mathbb{Z}[\VSet]$.
	We define $\mathit{Pol}: \LSet \rightarrow \mathbb{Z}[\VSet]$ as polynomial ranking function for a transition set $\TSet$ iff there is a nonempty $\TSet_{>} \subseteq \TSet$ such that:
	\[ \text{For all } (\location, \tau, \location') \in \TSet \text{ it holds that } \tau \Rightarrow \mathit{Pol}(\location)(\valuation) \geq \mathit{Pol}(\location)(\valuation') \]
	\[ \text{For all } (\location, \tau, \location') \in \TSet_{>} \text{ it holds that } \tau \Rightarrow \mathit{Pol}(\location)(\valuation) > \mathit{Pol}(\location)(\valuation') \]
	\[ \text{For all } (\location, \tau, \location') \in \TSet_{>} \text{ it holds that } \tau \Rightarrow \mathit{Pol}(\location)(\valuation) > 1 \]
\end{definition}


\subsection{Complexity}

\subsubsection{Runtime Complexity}

\begin{definition}[Worst-Case Runtime Complexity]
	\[ \text{rc}(m) = \sup \braced{ k + 1 \mid \exists k \in \mathbb{N}, \valuation_0, \valuation: \abs{\valuation_0} \leq m \wedge \valuation_0 (\rightarrow_{t_0} \circ \rightarrow^k) \valuation } \]
\end{definition}

\begin{definition}[Upper Runtime Bound]
	\[ \UTime(t)(m) \geq \sup \braced{ k \mid \exists k \in \mathbb{N}, \valuation_0, \valuation: \abs{\valuation_0} \leq m \wedge \valuation_0 (\rightarrow_{t_0} \circ (\rightarrow^* \circ \rightarrow_t)^k) \valuation } \]
\end{definition}

\begin{theorem}[Approximating Runtime Complexity]
	Let $\UTime$ be a runtime approximation for $\TSet$.
	Then it holds that 
	\[ \mathit{rc} \leq \sum_{t \in \TSet}\UTime(t) \]
\end{theorem}

\subsubsection{Cost Complexity}

\begin{definition}[Worst-Case Cost Complexity]
  \[ \text{cc}(m) = \sup \braced{ \sum_{0 \leq i \leq k} c(t_i)(\valuation_i) \mid \exists t_1, \cdots, t_k, \valuation_0, \cdots, \valuation_k: \abs{\valuation_0} \leq m \wedge
    \valuation_0 \rightarrow_{t_0} \valuation_1 \rightarrow_{t_1} \dots \rightarrow_{t_k} \valuation_k }
  \]
\end{definition}

\begin{definition}[Upper Cost Bound]
  \[ \UCost(t)(m) =
  \begin{cases}
    \UTime(t) \cdot \maxO{c(t)} & \text{if } t \text{ is an initial transition} \\
    \UTime(t) \cdot \maxO{\maximum{(c(t)_+)(\USize(\tilde{t})) - (c(t)_-)(\LSize(\tilde{t})) \mid \tilde{t} \in \pre(t)}} & \text{otherwise}
  \end{cases}
  \]
\end{definition}

\begin{theorem}[Approximating Cost Complexity]
	Let $\UCost$ be a cost approximation for $\TSet$.
	Then it holds that 
	\[ \mathit{cc} \leq \sum_{t \in \TSet} \UCost(t) \]
\end{theorem}

\todo{Define size complexity}{}
