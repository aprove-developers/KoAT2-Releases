\section{Preliminaries}


\subsection{Programs}

\begin{definition}[Program] 
  Let $\VSet$ be the set of all occurring program variables.
  Let $\LSet$ be the set of all locations of a program.
  Let $\TSet = \LSet \times F(\VSet) \times \LSet$ be the set of all transitions of the program where \todo{Only allow conjunctions here for PRF?}{$F(\VSet)$} is the set of all quantifier-free formulas over $\VSet$.
  A program is a graph $P = (\LSet, \TSet)$ where $\LSet$ is the set of program locations and $\TSet$ is a subset of all possible transitions between them.
  We write $\location_0$ for the unique program location with $\location_0 \in \LSet$ which has no entry transitions in the program.
  We write $t_0$ for an initial transition. That is a transition $(\location_0, \tau, \location) \in \TSet$ where the start location is the initial location $\location_0$.
  We write $\TSet_0 \subseteq \TSet$ to denote the set of all initial transitions.
\end{definition}

\begin{definition}[State] 
  A state is a total function $\valuation: \VSet \rightarrow \mathbb{Z}_\bot$ which assigns each program variable $v \in \VSet$ a value $\valuation(v) \in \mathbb{Z}_\bot = \mathbb{Z} \cup \braced{\bot}$.
  We denote with $\valuation(v) = \bot$ that the state does not define a value for the variable $v$.
  We write $\Sigma = \braced{ \valuation \mid \valuation: \VSet \rightarrow \mathbb{Z}_\bot}$ to denote the set of all states.
\end{definition}

\begin{definition}[Evaluation] 
  For a transition $t \in \TSet$ an evaluation step $\rightarrow_t \in (\LSet \times \Sigma) \times (\LSet \times \Sigma)$ is a relation between a previous state at a location and a resulting state at a following location.
  We write $(\location, \valuation) \rightarrow_t (\location', \valuation')$ iff with $t = (\location_t, \tau, \location_t') \in \TSet$ it holds that $\valuation \models \tau$ and $\valuation' \models \tau$ as well as $\location = \location_t$ and $\location' = \location_t'$.
  We omit the transition and write $(\location, \valuation) \rightarrow (\location', \valuation')$ iff there exists a transition $t \in \TSet$ such that $(\location, \valuation) \rightarrow_t (\location', \valuation')$ holds.
  We write $(\location, \valuation) \rightarrow^k (\location', \valuation')$ iff there exists a sequence of transitions $t_1, \dots, t_k \in \TSet$ such that $(\location, \valuation) \rightarrow_{t_1} \dots \rightarrow_{t_k} (\location', \valuation')$ holds.
  We write $(\location, \valuation) \rightarrow^* (\location', \valuation')$ iff there exists a $k \in \mathbb{N}$ such that $(\location, \valuation) \rightarrow^k (\location', \valuation')$ holds.
  We write $(\location, \valuation) \rightarrow_{\TSet'} (\location', \valuation')$ iff there exists a transition $t \in \TSet' \subseteq \TSet$ such that $(\location, \valuation) \rightarrow_t (\location', \valuation')$ holds.
\end{definition}


\subsection{Result Variable Graph}

\begin{definition}[Pre-Transitions] 
  We define $\pre: \TSet \rightarrow 2^\TSet$ as
  \[\pre(t) = \braced{\tilde{t} \in \TSet \mid \exists \valuation_0, \location, \valuation: (\location_0, \valuation_0) (\rightarrow^* \circ \rightarrow_{\tilde{t}} \circ \rightarrow_{t}) (\location, \valuation)}\]
  to denote the set of all transitions that may precede $t$ in an evaluation.	
\end{definition}

If $\pre(t)$ represents the set of all transitions that may precede $t$ in an evaluation, then $\pre_{\mathit{trivial}}((\location_1, \tau_1, \location_1')) = \braced{(\location_2, \tau_2, \location_2') \in \TSet \mid \location_2' = \location_1}$ is a valid overapproximation.

\begin{definition}[Active variables] 
	We define $\actV: B \rightarrow 2^\VSet$ as 
	\[ \actV(f) = \braced{ v_i \in \VSet \mid \exists m_1, \dots, m_n, m_i' \in \mathbb{Z}: f(m_1, \dots, m_i, \dots, m_n) \neq f(m_1, \dots, m_i', \dots, m_n)} \]
	to denote the set of active variables in $f$.
\end{definition}

\begin{definition}[Result Variable Graph]
	Let $\LSB$ be a local sizebound.
	We define 
	\[ RVG = (\TSet \times \VSet, \braced{((t, v), (t', v')) \mid t \in \pre(t'), v \in \actV(\LLSB(t,v)) \cup \actV(\ULSB(t,v))}) \]
	to denote the result variable graph.
\end{definition}


\subsection{Ranking functions}

\begin{definition}[Polynomial Ranking Function] 
	Let $\mathbb{Z}[\VSet]$ be the polynomial ring in the variables $v \in \VSet$ over the field $\mathbb{Z}$.
	We define $p(\valuation) = p[v_1/\valuation(v_1), \dots, v_n/\valuation(v_n)]$ as the substitution of all $v_1, \dots v_n \in \VSet$ by the state $\valuation: \VSet \rightarrow \mathbb{Z}_\bot$ in the polynomial $p \in \mathbb{Z}[\VSet]$.
	We define $\mathit{Pol}: \LSet \rightarrow \mathbb{Z}[\VSet]$ as polynomial ranking function for a transition set $\TSet$ iff there is a nonempty $\TSet_{>} \subseteq \TSet$ such that:
	\[ \text{For all } (\location, \tau, \location') \in \TSet \text{ it holds that } \tau \Rightarrow \mathit{Pol}(\location)(\valuation) \geq \mathit{Pol}(\location)(\valuation') \]
	\[ \text{For all } (\location, \tau, \location') \in \TSet_{>} \text{ it holds that } \tau \Rightarrow \mathit{Pol}(\location)(\valuation) > \mathit{Pol}(\location)(\valuation') \]
	\[ \text{For all } (\location, \tau, \location') \in \TSet_{>} \text{ it holds that } \tau \Rightarrow \mathit{Pol}(\location)(\valuation) > 1 \]
\end{definition}


\subsection{Complexity}

\subsubsection{Runtime Complexity}

\begin{definition}[Worst-Case Runtime Complexity]
  \[ \text{rc}(m) = \sup \braced{ k \in \mathbb{N} \mid \exists \valuation_0, \location, \valuation: \abs{\valuation_0} \leq m \wedge (\location_0, \valuation_0) \rightarrow^k (\location, \valuation) } \]
\end{definition}

\begin{definition}[Upper Runtime Bound]
  \[ \UTime(t)(m) \geq \sup \braced{ k \mid \exists k \in \mathbb{N}, \valuation_0, \location, \valuation: (\location_0, \valuation_0) (\rightarrow^* \circ \rightarrow_t)^k (\location, \valuation) } \]
\end{definition}

\begin{theorem}[Approximating Runtime Complexity]
	Let $\UTime$ be a runtime approximation for $\TSet$.
	Then it holds that 
	\[ \mathit{rc} \leq \sum_{t \in \TSet}\UTime(t) \]
\end{theorem}

\subsubsection{Cost Complexity}

\begin{definition}[Worst-Case Cost Complexity]
\[ \text{cc}(m) = \sup \braced{ \sum_{0 \leq i \leq k} c(t_i)(\valuation_i) \mid \exists k \geq 1, \valuation_0, \location, \valuation: \abs{\valuation_0} \leq m \wedge
  (\location_0, \valuation_0) \rightarrow_{t_0} (\location_1, \valuation_1) \rightarrow_{t_1} \dots \rightarrow_{t_k} (\location_k, \valuation_k) } \]
\end{definition}

\begin{definition}[Upper Cost Bound]
  \[ \UCost(t)(m) =
  \begin{cases}
    \UTime(t) \cdot \maxO{c(t)} & \text{if } t \text{ is an initial transition} \\
    \UTime(t) \cdot \maxO{\maximum{(c(t)_+)(\USize(\tilde{t})) - (c(t)_-)(\LSize(\tilde{t})) \mid \tilde{t} \in \pre(t)}} & \text{otherwise}
  \end{cases}
  \]
\end{definition}

\begin{theorem}[Approximating Cost Complexity]
	Let $\UCost$ be a cost approximation for $\TSet$.
	Then it holds that 
	\[ \mathit{cc} \leq \sum_{t \in \TSet} \UCost(t) \]
\end{theorem}

\subsubsection{Size complexity}

\begin{definition}[Bound set]
	The set $\BoundSet$ of possible bounds is the smallest set with
	\[ \omega \in \BoundSet \]
	\[ k \in \BoundSet \text{ for all } k \in \mathbb{N} \] 
	\[ v \in \BoundSet \text{ for all } v \in \VSet \] 
	\[ -b \in \BoundSet \text{ for all } b \in \BoundSet \] 
	\[ b_1 + b_2 \in \BoundSet \text{ for all } b_1, b_2 \in \BoundSet \] 
	\[ b_1 \cdot b_2 \in \BoundSet \text{ for all } b_1, b_2 \in \BoundSet \] 
	\[ \max(b_1, b_2) \in \BoundSet \text{ for all } b_1, b_2 \in \BoundSet \]
	\[ k^b \in \BoundSet \text{ for all } k \in \mathbb{N}, b \in \BoundSet \]
        For a bound $b \in \BoundSet$ and a state $\valuation \in \VSet \rightarrow \mathbb{Z}_\bot$ we write $b(\valuation)$ to denote the evaluation of the bound $b$ to a value in $\mathbb{Z}_\bot$.
        For a bound $b \in \BoundSet$ and a bound assignment $\valuation \in \VSet \rightarrow \BoundSet$ we write $b(\valuation)$ to denote the substitution of every variable $v$ of the bound $b$ with the appropiate value $\valuation(v) \in BoundSet$.
\end{definition}

\todo{Define size complexity}{}

\begin{definition}[Worst-Case Size Complexity]
  We call $\USize: \RV \rightarrow \BoundSet$ an \textbf{upper} sizebound if and only if for all $(t, v) \in \RV$ and all \todo{Correct, not to use m here?}{$\valuation \in \Valuation$} it holds that
  \[ \USize(t, v)(\valuation_0) \geq \sup \braced{\valuation(v) \mid \exists \location, \valuation: (\location_0, \valuation_0) (\rightarrow^* \circ \rightarrow_t) (\location, \valuation)}. \]
  Futhermore, we call $\LLSB: \RV \rightarrow \BoundSet$ a \textbf{lower} sizebound if and only if for all $(t, v) \in \RV$ and all $\valuation$ it holds that
  \[ \LSize(t, v)(\valuation_0) \geq \inf \braced{\valuation(v) \mid \exists \location, \valuation: (\location_0, \valuation_0) (\rightarrow^* \circ \rightarrow_t) (\location, \valuation)}. \]
  We call $\Size$ a sizebound.
\end{definition}

\begin{definition}[Local Sizebound]
  We call $\ULSB: \RV \rightarrow \BoundSet$ an \textbf{upper} local sizebound if and only if for all $(t, v) \in \RV$ and all \todo{Correct, not to use m here?}{$\valuation$} it holds that
  \[ \ULSB(t, v)(\valuation) \geq \sup \braced{\valuation'(v) \mid \exists \valuation': \valuation \rightarrow_t \valuation'}. \]
  Futhermore, we call $\LLSB: \RV \rightarrow \BoundSet$ a \textbf{lower} local sizebound if and only if for all $(t, v') \in \RV$ and all $\valuation$ it holds that
  \[ \LLSB(t, v)(\valuation) \geq \inf \braced{\valuation'(v) \mid \exists \valuation': \valuation \rightarrow_t \valuation'}. \]
  We call $\LSB$ a local sizebound.
\end{definition}

