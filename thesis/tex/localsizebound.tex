This section presents an algorithm for the computation of local size bounds.
In a first step, we define the possible local size bounds as a subset of all bounds $\BoundSet(\PVSet)$.
This way we are able to use specific characteristics of local size bounds later in the definition of the method for global size bounds.
We define two subsets $\BoundSet^\sqcup_l, \BoundSet^\sqcap_l \subseteq \BoundSet(\PVSet)$, one for lower local size bounds and one for upper local size bounds.

\begin{definition}[Local bound sets]
  Let $B$ denote the set of tuples $(s,e,P_1,P_2,N_1,N_2) \in \mathbb{N} \times \mathbb{Z} \times 2^\PVSet \times 2^\PVSet \times 2^\PVSet \times 2^\PVSet$ with $s \geq 1$, $P_1 \cap P_2 = \emptyset$ and $N_1 \cap N_2 = \emptyset$.
  Then, we define $\BoundSet^\sqcup_l \in \BoundSet(\PVSet)$ as 
  \[ \BoundSet^\sqcup_l = \left\{
    s \cdot \left(
        e
      + \sum_{v \in P_1} v
      - \sum_{v \in N_1} v
      - \sum_{v \in P_2} \maxO{-v}
      - \sum_{v \in N_2} \maxO{v}
      \right) \mid (s,e,P_1,P_2,N_1,N_2) \in B \right\} \]
  Also, we define $\BoundSet^\sqcap_l \in \BoundSet(\PVSet)$ as 
  \[ \BoundSet^\sqcap_l = \left\{
    s \cdot \left(
        e
      + \sum_{v \in P_1} v
      - \sum_{v \in N_1} v
      + \sum_{v \in P_2} \maxO{v}
      + \sum_{v \in N_2} \maxO{-v}
      \right) \mid (s,e,P_1,P_2,N_1,N_2) \in B \right\} \]
\end{definition}

Remember, that a local size bound $\LSB$ is an approximation of the values of variables after a single evaluation step with a transition $t$.
A pair $(t,v)$ of a transition and a variable is also writable as a result variable $\rv = (t,v)$.
We now use the defined local bound sets for the definition of a concrete local size bound, which we call a scaled sum.

\begin{definition}[Scaled sum]
  We say that $\rv \in \RV$ is bounded by an \textbf{upper scaled sum} if and only if there is an upper local size bound $\ULSB(\rv) \in \BoundSet^\sqcap_l$.
  We then denote with $s^\sqcap_\rv$ the scaling factor $s$, with $e^\sqcap_\rv$ the constant $e$ and with the sets $P_{\rv,1}^\sqcap, P_{\rv,2}^\sqcap, N_{\rv,1}^\sqcap, N_{\rv,2}^\sqcap$ the four variable sets $P_1$, $P_2$, $N_1$ and $N_2$.
  We say that $\rv \in \RV$ is bounded by a \textbf{lower scaled sum} if and only if there is a lower local size bound $\LLSB(\rv) \in \BoundSet^\sqcup_l$.
  We then denote with $s^\sqcup_\rv$ the scaling factor $s$, with $e^\sqcup_\rv$ the constant $e$ and with the sets $P_{\rv,1}^\sqcup, P_{\rv,2}^\sqcup, N_{\rv,1}^\sqcup, N_{\rv,2}^\sqcup$ the four variable sets $P_1$, $P_2$, $N_1$ and $N_2$.
\end{definition}

To clarify the possible local size bounds we introduce some examples.

\begin{example}[Upper scaled sum]
  \begin{figure}
\centering

\begin{tikzpicture}[->,>=stealth',auto,node distance=7cm,
    thick,
    main node/.style={circle,draw,font=\sffamily\Large\bfseries},
    aligned edge/.style={align=left}]

  \node[main node] (0) {$\location$};
  \node[main node] (1) [right of=0] {$\location'$};

  \path[every node/.style={font=\sffamily\small}]
    (0) edge[aligned edge] node[above] {$t_1$, $\update(x) = x$} node[below] {$\ULSB(t_1,x) = 1 \cdot (0 + x) \in \BoundSet^\sqcap_l$} (1)
    ;
\end{tikzpicture}

\begin{tikzpicture}[->,>=stealth',auto,node distance=7cm,
    thick,
    main node/.style={circle,draw,font=\sffamily\Large\bfseries},
    aligned edge/.style={align=left}]

  \node[main node] (0) {$\location$};
  \node[main node] (1) [right of=0] {$\location'$};

  \path[every node/.style={font=\sffamily\small}]
    (0) edge[aligned edge] node[above] {$t_2$, $\update(x) = -x$} node[below] {$\ULSB(t_2,x) = 1 \cdot (0 - x) \in \BoundSet^\sqcap_l$} (1)
    ;
\end{tikzpicture}

\begin{tikzpicture}[->,>=stealth',auto,node distance=7cm,
    thick,
    main node/.style={circle,draw,font=\sffamily\Large\bfseries},
    aligned edge/.style={align=left}]

  \node[main node] (0) {$\location$};
  \node[main node] (1) [right of=0] {$\location'$};

  \path[every node/.style={font=\sffamily\small}]
    (0) edge[aligned edge] node[above] {$t_3$, $\update(x) = 2 \cdot x + 2 \cdot y + 4$} node[below] {$\ULSB(t_3,x) = 2 \cdot (2 + x + y) \in \BoundSet^\sqcap_l$} (1)
    ;
\end{tikzpicture}

\begin{tikzpicture}[->,>=stealth',auto,node distance=7cm,
    thick,
    main node/.style={circle,draw,font=\sffamily\Large\bfseries},
    aligned edge/.style={align=left}]

  \node[main node] (0) {$\location$};
  \node[main node] (1) [right of=0] {$\location'$};

  \path[every node/.style={font=\sffamily\small}]
    (0) edge[aligned edge] node[above] {$t_4$, $\update(x) = 2 \cdot x + 3 \cdot y$} node[below] {$\ULSB(t_4,x) = 3 \cdot (0 + \maxO{x} + y) \in \BoundSet^\sqcap_l$} (1)
    ;
\end{tikzpicture}

\begin{tikzpicture}[->,>=stealth',auto,node distance=7cm,
    thick,
    main node/.style={circle,draw,font=\sffamily\Large\bfseries},
    aligned edge/.style={align=left}]

  \node[main node] (0) {$\location$};
  \node[main node] (1) [right of=0] {$\location'$};

  \path[every node/.style={font=\sffamily\small}]
    (0) edge[aligned edge] node[above] {$t_5$, $\update(x) = -2 \cdot x - 3 \cdot y$} node[below] {$\ULSB(t_5,x) = 3 \cdot (0 + \maxO{-x} - y) \in \BoundSet^\sqcap_l$} (1)
    ;
\end{tikzpicture}

\caption{Single transitions and their constructed scaled sum}
\label{fig:scaled_sum}
\end{figure}


  Consider the five transitions in Figure \ref{fig:scaled_sum}.
  For the transition $t_1$ the update is $\update(x) = x$.
  We can represent an upper scaled sum for $x$ in $t_1$ with $\ULSB(t_1,x) = 1 \cdot (0 + x) = x \in \BoundSet^\sqcap_l$, where $x$ is in $P_1$.
  For the transition $t_2$ the update is $\update(x) = -x$.
  We can represent an upper scaled sum for $x$ in $t_2$ with $\ULSB(t_2,x) = 1 \cdot (0 - x) = -x \in \BoundSet^\sqcap_l$, where $x$ is in $N_1$.
  If $N_1$ was not part of the local bound set $\BoundSet^\sqcap_l$, this would be impossible.
  For the transition $t_3$ the update is $\update(x) = 2 \cdot x + 2 \cdot y + 4$.
  We can represent an upper scaled sum for $x$ in $t_3$ with $\ULSB(t_3,x) = 2 \cdot (2 + x + y) \in \BoundSet^\sqcap_l$, where $x$ and $y$ are in $P_1$.
  
  In the previous three examples, it was sufficient to have $P_2 = \emptyset$ and $N_2 = \emptyset$.
  For the transition $t_4$ the update is $\update(x) = 2 \cdot x + 3 \cdot y$.
  One could suggest the upper scaled sum $b = 3 \cdot (x + y)$.
  But for a state $\valuation$ with $\exacteval{x}{\valuation} < 0$ and $\exacteval{y}{\valuation} > 0$ this bound is not sound, since with $\exacteval{b}{\valuation}$ we would underapproximate the actual value $\exacteval{2 \cdot x + 3 \cdot y}{\valuation}$.
  For these cases, scaled sums provide the sets $P_2$ and $N_2$.
  We can represent an upper scaled sum for $x$ in $t_4$ with $\ULSB(t_4,x) = 3 \cdot (\maxO{x} + y) \in \BoundSet^\sqcap_l$, where $x$ is in $P_2$ and $y$ is in $P_1$.
  This way the upper scaled sum is also sound for negative values of $x$.
  
  For the previous case, we used the set $P_2$.
  The set $N_2$ is provided for the negative counterpart.
  For the transition $t_5$ the update is $\update(x) = -2 \cdot x - 3 \cdot y$.
  We can represent an upper scaled sum for $x$ in $t_5$ with $\ULSB(t_5,x) = 3 \cdot (\maxO{-x} - y) \in \BoundSet^\sqcap_l$, where $x$ is in $N_2$ and $y$ is in $N_1$.
  With this construction, we only consider negative values of $x$ and therefore ensure a sound overapproximation.

\end{example}
  
The construction of a lower scaled sum is very similar to the construction of an upper scaled sum.
The components $s$, $e$, $P_1$ and $N_1$ have the same purpose for lower scaled sums and upper scaled sums.

\begin{example}[Lower scaled sum]
  Consider again the transition $t_1$ with an $\update(x) = x$.
  We can represent a lower scaled sum the same way as an upper scaled sum with $\LLSB(t_1,x) = 1 \cdot (0 + x) = x \in \BoundSet^\sqcup_l$.
  The difference is in the role of the sets $P_2$ and $N_2$.
  While $\sum_{v \in P_2} \maxO{v}$ and $\sum_{v \in N_2} \maxO{-v}$ are added in upper scaled sums, their lower counterparts $\sum_{v \in P_2} \maxO{-v}$ and $\sum_{v \in N_2} \maxO{v}$ are subtracted in lower scaled sums.
  Consider again the transition $t_4$ with the $\update(x) = 2 \cdot x + 3 \cdot y$.
  We can represent a lower scaled sum with $\LLSB(t_4,x) = 3 \cdot (0 - \maxO{-x} + y) \in \BoundSet^\sqcup_l$, where $x$ is in $P_2$ and $y$ is in $P_1$.  
  This way the effect of an unscaled variable is always overapproximated in the upper case and always underapproximated in the lower case.
\end{example}

We will now present an algorithm for the computation of sound, but minimal local size bounds $b^\sqcap \in \BoundSet^\sqcap_l$ and $b^\sqcup \in \BoundSet^\sqcup_l$.
We first consider the upper case.
The main idea is to choose a candidate for an upper scaled sum and then use an SMT-Solver to prove its validity.
For a selected upper scaled sum $b \in \BoundSet^\sqcap_l$, the guard must imply for every state $\valuation$ that the upper scaled sum bounds the updated value of the variable.
\[ \forall \valuation \in \Valuation: \exacteval{\guard}{\valuation} \Rightarrow \exacteval{\update(v)}{\valuation} \leq \exacteval{b}{\valuation} \]
In general, an SMT-Solver is only able to find models for existential formulas $\exists v_0: \exists v_1: \dots \exists v_k: \phi(v_0, \dots, v_k)$. \cite{smt} \\
Therefore, we use the SMT-Solver to prove the unsatisfiability of the negation of the formula.
\[ \exists \valuation \in \Valuation: \exacteval{\guard}{\valuation} \wedge \exacteval{\update(v)}{\valuation} > \exacteval{b}{\valuation} \]
If the SMT-Solver proves unsatisfiability, the upper scaled sum is a sound overapproximation of the actual value of the variable $v$.

It remains to be shown, how to choose a candidate for an upper scaled sum.
We say that $b_1 \in \BoundSet^\sqcap_l$ is a better bound than $b_2 \in \BoundSet^\sqcap_l$ iff for all states $\valuation$ it holds that $\exacteval{b_1}{\valuation} \leq \exacteval{b_2}{\valuation}$.
We then write $b_1 \leq b_2$.
We want to find a bound $b \in \BoundSet^\sqcap_l$ such that for all bounds $b' \in \BoundSet^\sqcap_l$ it holds that $b \leq b'$.
We present an algorithm for finding such an optimal bound $b$.
The idea of this algorithm is to start with a trivial bound, that is always valid.
Then, we take multiple steps for a component-wise improvement of this bound, always keeping a valid bound.

\subsection{Trivial local size bounds}

In a first step, we need to define trivial local size bounds as the starting bounds for the algorithm.
We need to find integers $s$ and $e$ as well as variable sets $P_1$, $P_2$, $N_1$ and $N_2$ to define such a trivial local size bound.
For the definition of the factor $s$ and the constant $e$ we define an overapproximation of the occurring constants in a bound.

\begin{definition}[Overapproximation of occurring constants]
  We define $\constant: \BoundSet_p(\VSet) \rightarrow \mathbb{N}$ as the following function.
  \[ \constant(k) = k \text{ for all } k \in \mathbb{N} \subset \BoundSet_p(\VSet) \] 
  \[ \constant(v) = 1 \text{ for all } v \in \VSet \subset \BoundSet_p(\VSet) \] 
  \[ \constant(-b) = \constant(b) \text{ for all } b \in \BoundSet_p(\VSet) \] 
  \[ \constant(b_1 + b_2) = \constant(b_1) + \constant(b_2) \text{ for all } b_1, b_2 \in \BoundSet_p(\VSet) \] 
  \[ \constant(b_1 \cdot b_2) = \constant(b_1) \cdot \constant(b_2) \text{ for all } b_1, b_2 \in \BoundSet_p(\VSet) \] 
\end{definition}

\todo{Prove}{Since we consider polynomial updates and guards, the product of all overapproximation of occurring constants of the polynomials of the guard and the update forms an overapproximation of the factor $s$ and the constant $e$.}
Now, we can define the trivial lower local size bound and the trivial upper local size bound.

\begin{definition}[Trivial local size bounds]
  Let $\rv = ((\location,\update,\guard,\location'),v) \in \RV$ be a result variable, for which we want to find a lower local size bound and an upper local size bound.
  Let $\PVSet$ be the set of all program variables.
  Let \[ s = e = \constant(\update(v)) \cdot \prod_{(l \bowtie r) \in \guard} \left( \constant(l) \cdot \constant(r) \right) \] be an overapproximation of the actual scaling factor and the actual constant.
  Then, the upper local size bound $s \cdot (e + \sum_{v \in \VSet} \maxO{v} + \sum_{v \in \VSet} \maxO{-v}) \in \BoundSet^\sqcap_l$ is a valid upper local size bound $\ULSB(t,v)$.
  Also, the lower local size bound $s \cdot (e - \sum_{v \in \VSet} \maxO{-v} - \sum_{v \in \VSet} \maxO{v}) \in \BoundSet^\sqcup_l$ is a valid lower local size bound $\LLSB(t,v)$.
\end{definition}

Let $\rv = (t,v) \in \RV$ be a result variable.
We need to show that $\exacteval{b^\sqcup}{\valuation} \leq \exacteval{\LLSB(\rv)}{\valuation}$ and $\exacteval{\ULSB(\rv)}{\valuation} \leq \exacteval{b^\sqcap}{\valuation}$ hold for any evaluation $(\location_0, \valuation_0) \rightarrow^* (\location, \valuation) \rightarrow_t (\location', \valuation') \rightarrow^* \dots$.
\todo{Can we assume to have the update function here?}{}


We show the construction of a trivial local size bound with an example of a trivial upper local size bound.

\begin{example}[Trivial local size bounds]
  Let $\rv = ((\location,\update,\text{true},\location'),x) \in \RV$ be a result variable with the update $\update(x) = 4 + 2 \cdot y - 3 \cdot z$.
  Thus, the overapproximation of the occurring constants is $\constant(\update(x)) = 4 + 2 \cdot 1 + 3 \cdot 1 = 9$
  Then, by definition of the trivial upper local size bound we can obtain $b^\sqcap = 9 * (9 + \maxO{y} + \maxO{z} + \maxO{-y} + \maxO{-z})$.
  Note that for all variables $v \in \VSet$, we have $\maxO{v} + \maxO{-v} = \abs{v}$, where $\abs{v}$ represents the absolute value of the variable $v$.
  Then, it is easy to see, that $b^\sqcap = 9 * (9 + \abs{y} + \abs{z})$ is a valid trivial upper local size bound.
\end{example}

\subsection{Algorithm for scaled sums}

With given trivial local size bounds we can now search the local bound sets $\BoundSet^\sqcup_l$ and $\BoundSet^\sqcap_l$ for better bounds.

\begin{algorithm}
\caption{Inferring a scaled sum}\label{ulsb_algorithm}
\begin{algorithmic}[1]
  \State Input: A result variable $t = ((\location,\update,\guard,\location'),v) \in \RV$
  \State Set $\bowtie$ to $\geq$ for the lower case and to $\leq$ for the upper case
  \State Set $f$ to a function $f(x)=x+1$ for the lower case and to $f(x)=x-1$ for the upper case
  \State Construct a trivial local size bound $b$ with the parameters $s$, $e$, $P_1$, $N_1$, $P_2$ and $N_2$
  \State Binary search $1 \dots s$ to find an $s$ such that $\bigwedge \guard \Rightarrow \update(v) \bowtie b$ is valid with this $s$, but not valid with $s-1$.
  \State Binary search $-e \dots e$ to find an $e$ such that $\bigwedge \guard \Rightarrow \update(v) \bowtie b$ is valid with this $e$, but not valid with $f(e)$.
  \ForAll {variables $v \in \VSet$}
    \State Set $P_2 := P_2 \setminus \braced{v}$
    \If {$\bigwedge \guard \Rightarrow \update(v) \bowtie b$ is not valid}
      \State Set $P_1 := P_1 \cup \braced{v}$
      \If {$\bigwedge \guard \Rightarrow \update(v) \bowtie b$ is not valid}
        \State Revert to $P_2 := P_2 \cup \braced{v}$ and $P_1 := P_1 \setminus \braced{v}$
      \EndIf
    \EndIf
    \State Set $N_2 := N_2 \setminus \braced{v}$
    \If {$\bigwedge \guard \Rightarrow \update(v) \bowtie b$ is not valid}
      \State Set $N_1 := N_1 \cup \braced{v}$
      \If {$\bigwedge \guard \Rightarrow \update(v) \bowtie b$ is not valid}
        \State Revert to $N_2 := N_2 \cup \braced{v}$ and $N_1 := N_1 \setminus \braced{v}$
      \EndIf
    \EndIf
  \EndFor
\end{algorithmic}
\end{algorithm}

The presented algorithm consists of three steps.
In the first step the scaling factor $s$ is minimized.
Since we chose the constant $e$ to be big enough, it is ensured, that the constant $e$ is not the reason for $\bigwedge \guard \Rightarrow \update(v) \leq b$ not being valid with $s-1$.
Instead there exists a variable, which does not allow further minimization of the scaling factor $s$.
Then, the next step is to also minimize the constant $e$.
With the minimized factor $s$ and the minimized constant $e$, we can now filter the variables, which are not required for the validity of the bound.
Assume a variable $v \in \VSet$.
In a trivial local size bound the variable $v$ is both in $P_2$ and $N_2$.
If the variable is obsolete for the bound, it is removed from both sets, otherwise it is removed from at least one set.
If the variable can not be removed from both sets, it might still be possible to move it to $P_1$ and $N_1$, respectively.
Note that this might be possible, since for each variable $v \in \VSet$ and each valuation $\valuation \in \Valuation$ it holds that $\exacteval{v}{\valuation} \leq \exacteval{\maxO{v}}{\valuation}$ and also $\exacteval{-v}{\valuation} \leq \exacteval{\maxO{-v}}{\valuation}$.

The algorithm does not define an order for the iteration of the variables.
Note that the order affects the resulting bound, but not its optimality.
Consider the update $\update(x) = y + z$ and the guard $\guard = \braced{y = z}$.
Depending on the order it is both possible to receive the bounds $2 \cdot y$ and $2 \cdot z$.
But since $2 \cdot y \leq 2 \cdot z$ and $2 \cdot z \leq 2 \cdot y$ hold, they are both optimal.
