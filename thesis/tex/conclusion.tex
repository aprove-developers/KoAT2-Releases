\section{Conclusion}

This thesis presents a method for the computation of non-monotonic time, size and cost bounds of integer programs.
It is based on the method of the KoAT paper \cite{koat} and extends it in several ways.

The main extension regards the computation of ranking functions for time bounds.
While the presented method uses the same approach to determine a ranking function as \cite{koat}, \cite{koat} applies an operator which transforms a non-monotonic rank into a monotonic rank.
The presented method uses an approximated substitution which substitutes each variable in regard to the monotonicity of the rank in this variable.
This way, a non-monotonic rank does not need to be transformed into a monotonic rank.

This change alone does not yield any benefit if the size bound used for the substitution of a variable is itself an upper size bound and its negation a lower size bound.
With this property of a size bound, an approximated substitution also transforms each non-monotonic rank into a monotonic rank.
Therefore, the presented method distinguishes between lower size bounds and upper size bounds.
It is able to use the difference between the positive and the negative effect of a variable.
The positive effect determines the upper size bound and the negative effect determines the lower size bound.
The opposite effect can be eliminated, respectively.
This way, the method yields more precise size bounds than \cite{koat}.
These are then used to infer more precise time bounds.

Another improvement concerns the computation of cost bounds.
This thesis presents an approach, where ranking functions are also used in the context of the costs of a program.
While time ranking functions yield a measure on the maximal number of steps possible from a specific location, cost ranking functions yield a measure on the maximal costs possible from a specific location.
With cost ranking functions, the presented method is able to infer more precise cost bounds.

The main benefit of the presented method is a more precise approximation of the coefficients of time and cost bounds.
But the method is also able to yield a better asymptotic complexity class.
This is, for example, the case if the lower size bound of a variable is approximated to be $0$ instead of the negation of the size bound computed by the method of \cite{koat}.

The described improvements and extensions of the presented method and the resulting performance in the evaluation against other comparable methods make the presented method a powerful tool for the computation of time and cost bounds of integer programs.
