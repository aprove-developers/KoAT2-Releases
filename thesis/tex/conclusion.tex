\section{Conclusion}

In this thesis, we presented a method for the computation of time and cost bounds of integer programs.
It is based on the method of the KoAT paper \cite{koat} and extends it in several ways.

The main extension regards the computation of ranking functions for time bounds.
While the presented method uses the same approach for determining a ranking function as the KoAT paper \cite{koat}, the KoAT paper uses an operator which transforms a possibly non-monotonic rank into a monotonic rank.
The presented method uses an approximated substitution which substitutes each variable in regard to the monotonicity of the rank in this variable.
This way a non-monotonic rank does not need to be transformed to a monotonic rank.

This change alone does not yield any benefit, if the size bound used for the substitution of a variable is itself an upper size bound and its negation a lower size bound.
With this property of a size bound, an approximated substitution also transforms each rank into a monotonic rank.
Therefore, the presented method distinguishes between lower size bounds and upper size bounds.
It is able to use the difference of the positive and the negative effect of a variable.
The positive effect determines the upper size bound and the negative effect determines the lower size bound and the opposite effect can be eliminated, respectively.
This way the method yields better size bounds than the KoAT paper \cite{koat}, which are then used to infer better time bounds.

Another improvement regards the computation of cost bounds.
This thesis presents an approach, where ranking functions are also used in the context of the costs of a program.
While time ranking functions yield a measure on the maximal number of steps possible from a specific location, cost ranking functions yield a measure on the maximal costs possible from a specific location.
This way the presented method is able to infer a better cost bound than the trivial cost bound resulting from a multiplication of the time bound of a transition with the cost of a transition.

The main benefit of the presented method is a better approximation of the coefficients of time and cost bounds.
But the method is also able to yield a better asymptotic complexity class, if for example the lower size bound of a variable is approximated to be $0$ instead of the negation of the size bound computed by the method of the KoAT paper \cite{koat}.

The described improvements and extensions of the presented method as well as its performance in the evaluation against other comparable methods makes the presented method a powerful tool for the computation of time and cost bounds of integer programs.
