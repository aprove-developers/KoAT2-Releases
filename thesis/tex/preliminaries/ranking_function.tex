\subsection{Ranking functions}

\begin{definition}[Polynomial Ranking Function] 
	We define $\mathit{Pol}: \LSet \rightarrow \BoundSet_p$ as polynomial ranking function for a transition set $\TSet$ iff there is a nonempty $\TSet_{>} \subseteq \TSet$ such that the following statements hold.
        For all transitions $(\location, \update, \guard, \location') \in \TSet$ and every state $\valuation \in \Valuation$ it holds that
	\[ \guard \Rightarrow \eval{\mathit{Pol}(\location)}{\valuation} \geq \eval{\mathit{Pol}(\location')}{\valuation \circ \update}. \]
        For all transitions $(\location, \update, \guard, \location') \in \TSet_{>}$ and every state $\valuation \in \Valuation$ it holds that        
	\[ \guard \Rightarrow \eval{\mathit{Pol}(\location)}{\valuation} \geq \eval{\mathit{Pol}(\location')}{\valuation \circ \update} + \eval{\cost((\location, \update, \guard, \location'))}{\valuation} \]
        and
	\[ \guard \Rightarrow \eval{\mathit{Pol}(\location)}{\valuation} > 1. \]
        \todo{Boundedness >1?}{}
\end{definition}

\todo{Problem with prf and smt}{}



Farkas lemma specialized for polyranking \cite{bradley2005polyranking}
General farkas lemma \cite{schrijver1998theory}
\begin{definition}[Farkas Lemma]
  Let all atoms $\atom \in \guard$ be of the following linear form with $d_\atom \in \mathbb{Z}$ and $d_{\atom,v} \in \mathbb{Z}$ for each variable $v \in \VSet$.
  \[ d_\atom + \sum_{v \in \VSet} d_{\atom,v} \cdot v \geq 0 \]
  If there exists a $d_\atom$ for each atom $\atom$ and if there exist $d_{\atom,v}$ for each atom $\atom$ and each variable $v$ such that $d_\atom + \sum_{v \in \VSet} d_{\atom,v} \cdot \valuation(v) \geq 0$ is satisfiable for all states $\valuation \in \Valuation$, then it entails a linear inequality $c + \sum_{v \in \VSet} c_v \cdot v \geq 0$ iff there exist real numbers $\lambda_\atom \geq 0$ for each atom $\atom \in \guard$ such that for the constant part it holds that $c \geq \sum_{\atom \in \guard} \lambda_\atom \cdot d_\atom$ and for all variables $v \in \VSet$ it holds that
  \[ c_v = \sum_{\atom \in \guard} \lambda_\atom \cdot d_{\atom,v}. \]
\end{definition}
With this definition of the farkas lemma we can infer from the satisfiability of the existential formula $\exists \lambda_{\atom_1}, \dots, \lambda_{\atom,n}: \exists c_{v_1}, \dots, c_{v_m}: c_v = \sum_{\atom \in \guard} \lambda_\atom \cdot d_{\atom,v}$ that a satisfiable constraint $\guard \in \ConstraintSet$ with only linear atoms entails a linear inequality $c + \sum_{v \in \VSet} c_v \cdot v \geq 0$.
