\subsection{Ranking Functions}

This section introduces a definition of ranking functions.
Ranking functions are well studied \cite{bradley2005polyranking, podelski2004prf, bradley2005linear, bagnara2012new, leike2014ranking, ben2013linear} and widely used in termination and complexity analysis.

The new method uses ranking functions for two different purposes:
For the computation of time bounds and for the computation of cost bounds.
For the computation of time bounds, we use a definition of \cite{koat}.
For the computation of cost bounds, the chapter about cost bounds introduces a specialized definition.

\begin{definition}[Time Ranking Function] 
  We define $\timerank: \LSet \rightarrow \BoundSet_p(\PVSet)$ as time ranking function for a transition set $\TSet$ if and only if there is a nonempty set of strictly decreasing transitions $\TSet_{>} \subseteq \TSet$ such that the following statements hold.
  For all transitions $t = (\location, \update, \guard, \location') \in \TSet$ and every evaluation step $(\location, \valuation) \rightarrow_t (\location', \valuation')$ it holds that
  \[ \exacteval{\guard}{\valuation} \Rightarrow \exacteval{\timerank(\location)}{\valuation} \geq \exacteval{\timerank(\location')}{\valuation'}. \]
  For all transitions $(\location, \update, \guard, \location') \in \TSet_{>}$ and every evaluation step $(\location, \valuation) \rightarrow_t (\location', \valuation')$ it holds that        
  \[ \exacteval{\guard}{\valuation} \Rightarrow \exacteval{\timerank(\location)}{\valuation} - 1 \geq \exacteval{\timerank(\location')}{\valuation'} \]
  and
  \[ \exacteval{\guard}{\valuation} \Rightarrow \exacteval{\timerank(\location)}{\valuation} \geq 1. \]
\end{definition}

A time ranking function $\timerank$ implies that a transition $t \in \TSet_>$ can only be used a limited number of times in an evaluation.
This is ensured since a transition $t \in \TSet_>$ both decreases the measure and is bounded by $1$, as well as other transitions $t \in \TSet$ do not increase the measure.
Therefore, for the transitions $t \in \TSet_>$ the rank $\timerank(\location)$ at a location $\location \in \LSet$ can be used as a bound on the number of occurrences of this transition in any evaluation from this location on.

\begin{example}[Time Ranking Function]
  Consider the program from the introduction in Figure \ref{fig:motivational_example_ranking}.
  \begin{figure}
\centering
\begin{tikzpicture}[->,>=stealth',auto,node distance=5cm,
    thick,
    main node/.style={circle,draw,font=\sffamily\Large\bfseries},
    aligned edge/.style={align=left}]

  \node[main node] (0) {$\location_0$};
  \node[main node] (1) [right of=0] {$\location_1$};

  \path[every node/.style={font=\sffamily\small}]
    (0) edge[aligned edge] node[above=0.2cm] {$t_0$} node[below=0.2cm] {$\update = \text{id}$} (1)
    (1) edge[aligned edge, loop above] node[left=0.2cm] {$t_1$} node[right=0.2cm] {$\update(x) = x - 1$\\$\update(y) = y$\\$\guard = \braced{x > y}$} (1)
    ;
\end{tikzpicture}
\caption{Program for a ranking function computation example}
\label{fig:motivational_example_ranking}
\end{figure}

  The function $\timerank$ with $\timerank(\location_1) = x-y$ can be proven to be a time rank for the transition set $\TSet = \braced{t_1}$.
  Consider a transition set $\TSet_> = \braced{t_1} \subseteq \TSet$.
  For the single transition $t_1$ it holds for every evaluation step $(\location_1,\valuation) \rightarrow_{t_1} (\location_1,\valuation')$ that $\exacteval{x>y}{\valuation} \Rightarrow \exacteval{x-y}{\valuation} > \exacteval{x-y}{\valuation'}$, since $x$ decreases in such an evaluation step.
  Therefore, also $\exacteval{x>y}{\valuation} \Rightarrow \exacteval{x-y}{\valuation} \geq \exacteval{x-y}{\valuation'}$ holds.
  Furthermore, it holds for every evaluation step $(\location_1,\valuation) \rightarrow_{t_1} (\location_1,\valuation')$ that $\exacteval{x>y}{\valuation} \Rightarrow \exacteval{x-y}{\valuation} \geq 1$.

  Note that in this example it would also be possible to consider the transition $t_0$ as part of the set $\TSet$, since it holds for every evaluation step $(\location_1,\valuation) \rightarrow_{t_0} (\location_1,\valuation')$ that $\valuation(x)=\valuation'(x)$ and $\valuation(y)=\valuation'(y)$ and therefore $\exacteval{\emph{true}}{\valuation} \Rightarrow \exacteval{x-y}{\valuation} \geq \exacteval{x-y}{\valuation'}$.
  However, in general an additional transition may increase the measure and therefore violate the time ranking conditions.
\end{example}

For the computation of ranking functions, it is common to use an SMT-Solver.
An SMT-Solver is capable of proving satisfiability and unsatisfiability of existential formulas.
The presented definition of time ranking functions consists of conditions, which need to be fulfilled for all states $\valuation, \valuation' \in \Valuation$ of all evaluation steps $(\location, \valuation) \rightarrow_t (\location', \valuation')$.
Such a universally quantified formula can be proven by showing the unsatisfiability of the negation of the formula with an SMT-Solver.
But for ranking functions, it is also necessary to consider that the coefficients of a time rank polynomial $\timerank(\location)$ are existentially quantified in the conditions of the ranking function definition.
Therefore, a negation of the condition yields universal quantification of the coefficients of a time rank.
Thus, the unsatisfiability of the negation of the condition is not provable by an SMT-Solver.

For such formulas, which contain both existential and universal quantification, it is common to apply a variant of Farkas' lemma \cite{schrijver1998theory}.
There already exist specialized approaches for ranking functions \cite{bradley2005polyranking}.
To utilize these approaches, it is necessary to restrict the domain of time ranks $\timerank(\location)$ to affine polynomials $\BoundSet_a(\PVSet)$.
Then, it is possible to transform the problem of finding a ranking function into a problem solvable with an SMT-Solver.
