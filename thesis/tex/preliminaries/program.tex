\subsection{Programs}

This section defines the term of a program.
In this master's thesis, we consider a program to be an integer transition system.

\begin{definition}[Program] 
  Let $\VSet$ be the set of all occurring program variables.
  A state is a total function $\valuation: \VSet \rightarrow \mathbb{Z}_\bot$ which assigns each program variable $v \in \VSet$ a value $\valuation(v) \in \mathbb{Z}_\bot = \mathbb{Z} \cup \braced{\bot}$.
  We denote with $\valuation(v) = \bot$ that the state does not define a value for the variable $v$.
  We write $\Valuation = \braced{ \valuation \mid \valuation: \VSet \rightarrow \mathbb{Z}_\bot}$ to denote the set of all states.
  Let $\LSet$ be the set of all locations of a program.
  Let $\TSet = \LSet \times (\VSet \rightarrow \BoundSet_p) \times 2^\AtomSet \times \LSet$ be the set of all transitions of the program.
  A program graph is a tuple $(\LSet, \TSet)$ where $\LSet$ is the set of program locations and $\TSet$ is a subset of all possible transitions between them.
  Let $\cost: \TSet \rightarrow \BoundSet$ be a function which maps each transition to its costs with a specific state.
  A program is a tuple $\Program = ((\LSet, \TSet), \location_0, \VSet, c)$.
  We denote with $\location_0$ the unique program location with $\location_0 \in \LSet$ which has no entry transitions in the program.
  We write $t_0$ for an initial transition. That is a transition $(\location_0, \tau, \location) \in \TSet$ where the start location is the initial location $\location_0$.
  We write $\TSet_0 \subseteq \TSet$ to denote the set of all initial transitions.
\end{definition}

In an integer transition system, every variable is considered to be an integer value.
A transition defines a possible move from one location to another location, if certain conditions are met.
In a transition $(\location,\update,\guard,\location') \in \TSet$, the guard $\guard$ represents the condition, which has to be fulfiled to move to the location $\location'$. The update function $\update$ assigns each variable a new value after the transition step.

We consider non-deterministic programs in this master's thesis.
Therefore there is no restriction, that from a location $\location \in \LSet$ there must be only one possible transition, for which the condition is fulfiled.
For example $\Program = (\braced{a,b,c},\braced{(a,\text{id},\text{true},b),(a,\text{id},\text{true},\cost)})$ is a valid program, although it is possible to both move to $b$ and $c$ from location $a$.
Such a program is not representable in common software languages.

For the definition of a run in a program, we define the terms of states and configurations.

\begin{definition}[Configuration] 
  A configuration is a pair $(\location, \valuation) \in \LSet \times \Valuation$ that defines the values of the program variables at a specific location.
\end{definition}

For two states $\valuation_1, \valuation_2 \in \Valuation$ we define $\valuation_1 \leq \valuation_2$ iff for all variables $v \in \VSet$ it holds that $\valuation_1(v) \leq \valuation_2(v)$.
We also define for a state $\valuation \in \Valuation$ the absolute value $\abs{\valuation} \in \Valuation$ iff for all variables $v \in \VSet$ it holds that $\abs{\valuation}(v) = \abs{\valuation(v)}$.

In a last step, we have to define, how a program can be executed.
Therefore we define the term of an evaluation.

\begin{definition}[Evaluation] 
  For a transition $t \in \TSet$ an evaluation step $\rightarrow_t \in (\LSet \times \Sigma) \times (\LSet \times \Sigma)$ is a relation between a previous state at a location and a resulting state at a following location.
  We write $(\location, \valuation) \rightarrow_t (\location', \valuation')$ iff with $t = (\location_t, \update, \guard, \location_t') \in \TSet$ it holds that $\valuation \models \guard$ and $\valuation' = \valuation \circ \update$ as well as $\location = \location_t$ and $\location' = \location_t'$.
  We omit the transition and write $(\location, \valuation) \rightarrow (\location', \valuation')$ iff there exists a transition $t \in \TSet$ such that $(\location, \valuation) \rightarrow_t (\location', \valuation')$ holds.
  We write $(\location, \valuation) \rightarrow^k (\location', \valuation')$ iff there exists a sequence of transitions $t_1, \dots, t_k \in \TSet$ such that $(\location, \valuation) \rightarrow_{t_1} \dots \rightarrow_{t_k} (\location', \valuation')$ holds.
  We write $(\location, \valuation) \rightarrow^* (\location', \valuation')$ iff there exists a $k \in \mathbb{N}$ such that $(\location, \valuation) \rightarrow^k (\location', \valuation')$ holds.
  We write $(\location, \valuation) \rightarrow_{\TSet'} (\location', \valuation')$ iff there exists a transition $t \in \TSet' \subseteq \TSet$ such that $(\location, \valuation) \rightarrow_t (\location', \valuation')$ holds.
\end{definition}
