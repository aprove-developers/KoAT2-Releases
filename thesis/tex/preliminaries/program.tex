\subsection{Programs}

This section defines the term of a program.
In this master's thesis, we consider programs with the following properties.

\begin{itemize}
\item Sequential: All operations are executed one after the other
\item Imperative: Program state is changed by statements
\item Non-recursive: Recursive function calls are not allowed
\item Mathematical integers: The variables take values in the mathematical integer domain $\mathbb{Z}$
\item Non-determinism: From a state we can enter non-deterministically multiple other states and non-deterministic values can be assigned to variables
\item No heap usage: All values are directly accessed and not referenced
\end{itemize}

These properties are common in related work.
Note that using imperative programs is not a restriction, since \todo{Source?}{every functional or logic program can be transformed into an imperative program.}
There also exists related work which extends the analysis to bit-vector arithmetic instead of mathematical integers. \cite{bitvectorarithmetic}

In related work as well as in this thesis, it is common to consider non-deterministic programs.
This is necessary, since the analyzed programs are often abstractions of other programs.
These programs may depend on user input or may use a dynamic memory.

\todo{Example: Cartesian product with filter of (x,x)?}{}

We now formally define the programs used in this thesis.
First, we define a program graph.

\begin{definition}[Program graph] 
  Let $\PVSet$ be the set of all occurring program variables.
  Let $\TVSet$ be a finite set of temporary variables with $\TVSet \cap \PVSet = \emptyset$.
  Let $\LSet$ be the set of all locations of a program.
  A label $w = (\update, \guard)$ consists of the update function $\update: \PVSet \rightarrow \BoundSet_p(\PVSet \cup \TVSet)$ and the guard $\guard \in \ConstraintSet(\PVSet \cup \TVSet)$.
  Let $\mathcal{W}$ be the set of all labels $(\update, \guard)$.
  Let $\TSet \subseteq \LSet \times \mathcal{W} \times \LSet$ be the set of transitions of the program.
  A transition $t = (\location, w, \location')$ consists of the start location $\location$, the target location $\location'$ and the label $w$.
  Then, a program graph is a tuple $(\LSet, \TSet)$.
\end{definition}

We also write $t = (\location, \update, \guard, \location')$ to denote a transition $(\location, (\update, \guard), \location') \in \TSet$.
We define a program as a program graph with an initial location and a function mapping transitions to the cost of their execution.

\begin{definition}[Program] 
  Let $\location_0 \in \LSet$ be the unique program location which has no entry transitions in the program.
  Let $(\LSet, \TSet)$ be a program graph.
  Let $\cost: \TSet \rightarrow \BoundSet(\PVSet)$ be a function which maps each transition to its cost depending on the values of the variables.
  For each transition $t \in \TSet$ and for each assignment $\valuation: \AllVarsSet \rightarrow \mathbb{Z}$ we require that the cost of the transition is positive, i.e. $\exacteval{\cost(t)}{\valuation} > 0$.
  Then, a program is a tuple $\Program = ((\LSet, \TSet), \location_0, \PVSet, \TVSet, c)$.
  Furthermore, we write $t_0$ to denote an initial transition.
  That is a transition $(\location_0, \tau, \location) \in \TSet$ where the start location is the initial location $\location_0$.
  We write $\TSet_0 \subseteq \TSet$ to denote the set of all initial transitions.
\end{definition}

These definitions are based on the definition of programs of the KoAT paper \cite{koat} and a definition which distinguishes between guards and updates \cite{lowerruntime}.

For program graphs we use the well-known definition of strongly connected components of a graph (\cite{sccs}).
We define the strongly connected components of a program $\Program = ((\LSet, \TSet), \location_0, \PVSet, \TVSet, c)$ as the strongly connected components of its program graph $(\LSet, \TSet)$.

From this point on, we also use the abbreviation SCC for a strongly connected component.

In a program, every variable is assumed to take an integer value.
A transition defines a possible move from one location to another location, if certain conditions are met.
In a transition $(\location,\update,\guard,\location') \in \TSet$, the guard $\guard$ represents the condition, which has to be fulfilled to move to the location $\location'$.
The update function $\update$ assigns each program variable a value after the transition step.

We consider non-deterministic programs in this master's thesis.
Non-determinism can occur in a program due to two reasons.
First, non-determinism occurs, when a program state fulfills the guard of multiple succeeding transitions. 
There is no restriction, that only one guard of transitions with the same starting location can be fulfilled.

\begin{figure}
\centering
\begin{tikzpicture}[->,>=stealth',auto,node distance=5cm,
    thick,
    main node/.style={circle,draw,font=\sffamily\Large\bfseries},
    aligned edge/.style={align=left}]

  \node[main node] (0) {$\location_0$};
  \node[main node] (1) [left of=0] {$\location_1$};
  \node[main node] (2) [right of=0] {$\location_2$};

  \path[every node/.style={font=\sffamily\small}]
    (0) edge[aligned edge] node[above=0.2cm] {$t_0$} node[below=0.2cm] {$\update = \text{id}$\\$\guard = \emptyset$} (1)
    (0) edge[aligned edge] node[above=0.2cm] {$t_1$} node[below=0.2cm] {$\update = \text{id}$\\$\guard = \emptyset$} (2)
    ;
\end{tikzpicture}
\caption{Program with a non-deterministic choice}
\label{fig:non_deterministic_choice}
\end{figure}



For example, the program graph in Figure \ref{fig:non_deterministic_choice} is valid, although from the location $\location_0$ it is possible to both move to the locations $\location_1$ and $\location_2$.
Such a program is not representable in common software languages.

The second reason for non-determinism is an update $\update$ which assigns a program variable $v \in \PVSet$ an expression depending on one or multiple temporary variables $\tvar_1, \dots, \tvar_n \in \TVSet$.
The values of these temporary variables are non-deterministic and only restricted by the guard $\guard$.
Therefore, an update of the value of the variable $v \in \PVSet$ might result in different values.
Consider the update $\update$ with $\update(x) = \tvar$, where $x \in \PVSet$ is a program variable and $\tvar \in \TVSet$ is a temporary variable.
Also, consider the guard $\guard = \braced{\tvar > 0, \tvar \leq y}$ with $y \in \PVSet$.
Then, the update of $x$ might result in a range of values $0 < x \leq y$. 
Note that while the set of temporary variables $\TVSet$ is the same for every transition, their value is not carried over to other transitions.
This is reflected in the update $\update: \PVSet \rightarrow \BoundSet_p(\AllVarsSet)$, which only assigns new values to program variables.

We will now formally define the evaluation relation of a program.

\begin{definition}[Evaluation] 
  A state is a function $\valuation: \AllVarsSet \rightarrow \mathbb{Z}$ which assigns each variable $v \in \AllVarsSet$ a value $\valuation(v) \in \mathbb{Z}$.
  We write $\Valuation = \braced{ \valuation \mid \valuation: \AllVarsSet \rightarrow \mathbb{Z}}$ to denote the set of all states.
  A configuration is a pair $(\location, \valuation) \in \LSet \times \Valuation$ that defines the values of all variables at a specific location.
  For a transition $t \in \TSet$ an evaluation step $\rightarrow_t \in (\LSet \times \Sigma) \times (\LSet \times \Sigma)$ is a relation between two configurations.
  We write $(\location, \valuation) \rightarrow_t (\location', \valuation')$ if and only if the transition $t = (\location_t, \update, \guard, \location_t') \in \TSet$ satisfies three conditions.
  First, it holds that the state satisfies the guard $\exacteval{\guard}{\valuation} = \textbf{true}$.
  Second, for each program variable $v \in \PVSet$ the resulting state $\valuation'$ satisfies the equation $\exacteval{\update(v)}{\valuation} = \exacteval{v}{\valuation'}$.
  And third, we have $\location = \location_t$ and $\location' = \location_t'$.
  We omit the transition and write $(\location, \valuation) \rightarrow (\location', \valuation')$ if and only if there exists a transition $t \in \TSet$ such that $(\location, \valuation) \rightarrow_t (\location', \valuation')$ holds.
  We write $(\location, \valuation) \rightarrow^k (\location', \valuation')$ if and only if there exists a sequence of transitions $t_1, \dots, t_k \in \TSet$ such that $(\location, \valuation) \rightarrow_{t_1} \dots \rightarrow_{t_k} (\location', \valuation')$ holds.
  We write $(\location, \valuation) \rightarrow^* (\location', \valuation')$ if and only if there exists a $k \in \mathbb{N}$ such that $(\location, \valuation) \rightarrow^k (\location', \valuation')$ holds.
  We write $(\location, \valuation) \rightarrow_{\TSet'} (\location', \valuation')$ if and only if there exists a transition $t \in \TSet' \subseteq \TSet$ such that $(\location, \valuation) \rightarrow_t (\location', \valuation')$ holds.
\end{definition}

Note that by definition, in an evaluation step $(\location, \valuation) \rightarrow_t (\location', \valuation')$ the value $\valuation'(v)$ of each temporary variable $v \in \TVSet$ is arbitrary.

For two states $\valuation_1, \valuation_2 \in \Valuation$ we define $\valuation_1 \leq \valuation_2$ if and only if for all variables $v \in \AllVarsSet$ it holds that $\valuation_1(v) \leq \valuation_2(v)$.
We also define for a state $\valuation \in \Valuation$ the absolute value $\abs{\valuation} \in \Valuation$ as a state with $\abs{\valuation}(v) = \abs{\valuation(v)}$ for each variable $v \in \AllVarsSet$.

