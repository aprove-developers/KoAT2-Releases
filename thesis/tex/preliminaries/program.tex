\subsection{Programs}

This section defines the term of a program.
In this master's thesis, we consider a program to be an integer transition system.

\begin{definition}[Program] 
  Let $\VSet$ be the set of all occurring program variables.
  Let $\LSet$ be the set of all locations of a program.
  Let $\TSet = \LSet \times F(\VSet) \times \LSet$ be the set of all transitions of the program where \todo{Only allow conjunctions here for PRF?}{$F(\VSet)$} is the set of all quantifier-free formulas over $\VSet$.
  A program is a graph $P = (\LSet, \TSet)$ where $\LSet$ is the set of program locations and $\TSet$ is a subset of all possible transitions between them.
  We write $\location_0$ for the unique program location with $\location_0 \in \LSet$ which has no entry transitions in the program.
  We write $t_0$ for an initial transition. That is a transition $(\location_0, \tau, \location) \in \TSet$ where the start location is the initial location $\location_0$.
  We write $\TSet_0 \subseteq \TSet$ to denote the set of all initial transitions.
\end{definition}

In an integer transition system, every variable is considered to be an integer value.
A transition defines a possible move from one location to another location, if certain conditions are met.
In a transition $(\location,\tau,\location') \in \TSet$, $\tau$ represents the condition, which has to be fulfiled to move to the location $\location'$, as well as the assignment of the variables after the transition step.

We consider non-deterministic programs in this master's thesis.
Therefore there is no restriction, that from a location $\location \in \LSet$ there must be only one possible transition, for which the condition is fulfiled.
For example $P=(\braced{a,b,c},\braced{(a,true,b),(a,true,c)})$ is a valid program, although it is possible to both move to $b$ and $c$ from location $a$.
Such a program is not representable in common software languages.

For the definition of a run in a program, we define the terms of states and configurations.

\begin{definition}[Configuration] 
  A state is a total function $\valuation: \VSet \rightarrow \mathbb{Z}_\bot$ which assigns each program variable $v \in \VSet$ a value $\valuation(v) \in \mathbb{Z}_\bot = \mathbb{Z} \cup \braced{\bot}$.
  We denote with $\valuation(v) = \bot$ that the state does not define a value for the variable $v$.
  We write $\Sigma = \braced{ \valuation \mid \valuation: \VSet \rightarrow \mathbb{Z}_\bot}$ to denote the set of all states.
  A configuration is a pair $(\location, \valuation) \in \LSet \times \Sigma$ that defines the values of the program variables at a specific location.
\end{definition}

In a last step, we have to define, how a program can be executed.
Therefore we define the term of an evaluation.

\begin{definition}[Evaluation] 
  For a transition $t \in \TSet$ an evaluation step $\rightarrow_t \in (\LSet \times \Sigma) \times (\LSet \times \Sigma)$ is a relation between a previous state at a location and a resulting state at a following location.
  We write $(\location, \valuation) \rightarrow_t (\location', \valuation')$ iff with $t = (\location_t, \tau, \location_t') \in \TSet$ it holds that $\valuation \models \tau$ and $\valuation' \models \tau$ as well as $\location = \location_t$ and $\location' = \location_t'$.
  We omit the transition and write $(\location, \valuation) \rightarrow (\location', \valuation')$ iff there exists a transition $t \in \TSet$ such that $(\location, \valuation) \rightarrow_t (\location', \valuation')$ holds.
  We write $(\location, \valuation) \rightarrow^k (\location', \valuation')$ iff there exists a sequence of transitions $t_1, \dots, t_k \in \TSet$ such that $(\location, \valuation) \rightarrow_{t_1} \dots \rightarrow_{t_k} (\location', \valuation')$ holds.
  We write $(\location, \valuation) \rightarrow^* (\location', \valuation')$ iff there exists a $k \in \mathbb{N}$ such that $(\location, \valuation) \rightarrow^k (\location', \valuation')$ holds.
  We write $(\location, \valuation) \rightarrow_{\TSet'} (\location', \valuation')$ iff there exists a transition $t \in \TSet' \subseteq \TSet$ such that $(\location, \valuation) \rightarrow_t (\location', \valuation')$ holds.
\end{definition}
