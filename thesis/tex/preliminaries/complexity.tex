\subsection{Complexity}

In this section we define three different types of complexities.
Runtime complexity describes how many steps an evaluation of a program will take in a worst-case szenario.
The costs of a transition are not considered in this complexity type.
On the other hand cost complexity does consider the costs of transitions and describes the cost of the sequence of transitions taken in a worst-case run.
\todo{Short sentence for size complexity}

\subsubsection{Runtime Complexity}

\begin{definition}[Worst-Case Runtime Complexity]
  We call $\text{rc} \in \BoundSet$ the time complexity of a program if and only if for all $m \in \Valuation$ it holds that
  \[ \text{rc}(m) = \sup \braced{ k \in \mathbb{N} \mid \exists \valuation_0, \location, \valuation: \abs{\valuation_0} \leq m \wedge (\location_0, \valuation_0) \rightarrow^k (\location, \valuation) } \]
\end{definition}

\begin{definition}[Upper Runtime Bound]
  We call $\UTime: \TSet \rightarrow \BoundSet$ a time bound if and only if for all $t \in \TSet$ and all $m \in \Valuation$ it holds that
  \[ \UTime(t)(m) \geq \sup \braced{ k \mid \exists k \in \mathbb{N}, \valuation_0, \location, \valuation: \abs{\valuation_0} \leq m \wedge (\location_0, \valuation_0) (\rightarrow^* \circ \rightarrow_t)^k (\location, \valuation) } \]
\end{definition}

\begin{theorem}[Approximating Runtime Complexity]
	Let $\UTime$ be a runtime approximation for $\TSet$.
	Then it holds that 
	\[ \mathit{rc} \leq \sum_{t \in \TSet}\UTime(t) \]
\end{theorem}

\subsubsection{Cost Complexity}

\begin{definition}[Worst-Case Cost Complexity]
\[ \text{cc}(m) = \sup \braced{ \sum_{0 \leq i \leq k} c(t_i)(\valuation_i) \mid \exists k \geq 1, \valuation_0, \location, \valuation: \abs{\valuation_0} \leq m \wedge
  (\location_0, \valuation_0) \rightarrow_{t_0} (\location_1, \valuation_1) \rightarrow_{t_1} \dots \rightarrow_{t_k} (\location_k, \valuation_k) } \]
\end{definition}

\begin{definition}[Upper Cost Bound]
  \[ \UCost(t)(m) =
  \begin{cases}
    \UTime(t) \cdot \maxO{c(t)} & \text{if } t \text{ is an initial transition} \\
    \UTime(t) \cdot \maxO{\maximum{(c(t)_+)(\USize(\tilde{t})) - (c(t)_-)(\LSize(\tilde{t})) \mid \tilde{t} \in \pre(t)}} & \text{otherwise}
  \end{cases}
  \]
\end{definition}

\begin{theorem}[Approximating Cost Complexity]
	Let $\UCost$ be a cost approximation for $\TSet$.
	Then it holds that 
	\[ \mathit{cc} \leq \sum_{t \in \TSet} \UCost(t) \]
\end{theorem}

\subsubsection{Size complexity}

For the size complexity of a program we distinguish between the upper size complexity and the lower size complexity.


\begin{definition}[Bound set]
  The set $\BoundSet$ of possible bounds is the smallest set with
  \[ \omega \in \BoundSet \]
  \[ k \in \BoundSet \text{ for all } k \in \mathbb{N} \] 
  \[ v \in \BoundSet \text{ for all } v \in \VSet \] 
  \[ -b \in \BoundSet \text{ for all } b \in \BoundSet \] 
  \[ b_1 + b_2 \in \BoundSet \text{ for all } b_1, b_2 \in \BoundSet \] 
  \[ b_1 \cdot b_2 \in \BoundSet \text{ for all } b_1, b_2 \in \BoundSet \] 
  \[ \max(b_1, b_2) \in \BoundSet \text{ for all } b_1, b_2 \in \BoundSet \]
  \[ k^b \in \BoundSet \text{ for all } k \in \mathbb{N}, b \in \BoundSet \]
\end{definition}

\begin{definition}[Bound evaluation]
  For a bound $b \in \BoundSet$ and a state $\valuation \in \VSet \rightarrow \mathbb{Z}_\bot$ we define $\eval{b}{\valuation}$ to denote the evaluation of the bound $b$ to a value in $\mathbb{Z}_\bot$.
  \[ \eval{\omega}{\valuation} = \bot \text{ for } \omega \in \BoundSet \] \todo{Really bot?}
  \[ \eval{k}{\valuation} = k \text{ for all } k \in \mathbb{N} \subset \BoundSet \] 
  \[ \eval{v}{\valuation} = \valuation(v) \text{ for all } v \in \VSet \subset \BoundSet \] 
  \[ \eval{-b}{\valuation} = -\eval{b}{\valuation} \text{ for all } b \in \BoundSet \] 
  \[ \eval{b_1 + b_2}{\valuation} = \eval{b_1}{\valuation} + \eval{b_2}{\valuation} \text{ for all } b_1, b_2 \in \BoundSet \] 
  \[ \eval{b_1 \cdot b_2}{\valuation} = \eval{b_1}{\valuation} \cdot \eval{b_2}{\valuation} \text{ for all } b_1, b_2 \in \BoundSet \] 
  \[ \eval{\max(b_1, b_2)}{\valuation} = \max(\eval{b_1}{\valuation}, \eval{b_2}{\valuation}) \text{ for all } b_1, b_2 \in \BoundSet \]
  \[ \eval{k^b}{\valuation} = k^{\eval{b}{\valuation}} \text{ for all } k \in \mathbb{N} \subset \BoundSet, b \in \BoundSet \]  
  For a bound $b \in \BoundSet$ and a bound assignment $\valuation \in \VSet \rightarrow \BoundSet$ we write $\eval{b}{\valuation}$ to denote the substitution of every variable $v$ of the bound $b$ with the appropriate value $\valuation(v) \in \BoundSet$.
\end{definition}

\todo{Actually define sc}{}

\begin{definition}[Worst-Case Size Complexity]
  We call $\USize: \RV \rightarrow \BoundSet$ an \textbf{upper} size bound if and only if for all $(t, v) \in \RV$ and all \todo{Correct, not to use m here?}{$\valuation \in \Valuation$} it holds that
  \[ \eval{\USize(t, v)}{\valuation_0} \geq \sup \braced{\valuation(v) \mid \exists \location, \valuation: (\location_0, \valuation_0) (\rightarrow^* \circ \rightarrow_t) (\location, \valuation)}. \]
  Furthermore, we call $\LLSB: \RV \rightarrow \BoundSet$ a \textbf{lower} size bound if and only if for all $(t, v) \in \RV$ and all $\valuation$ it holds that
  \[ \eval{\LSize(t, v)}{\valuation_0} \geq \inf \braced{\valuation(v) \mid \exists \location, \valuation: (\location_0, \valuation_0) (\rightarrow^* \circ \rightarrow_t) (\location, \valuation)}. \]
  We call $\Size$ a size bound.
\end{definition}

\begin{definition}[Local Sizebound]
  We call $\ULSB: \RV \rightarrow \BoundSet$ an \textbf{upper} local size bound if and only if for all $(t, v) \in \RV$ and all \todo{Correct, not to use m here?}{$\valuation$} it holds that
  \[ \eval{\ULSB(t, v)}{\valuation} \geq \sup \braced{\valuation'(v) \mid \exists \valuation': \valuation \rightarrow_t \valuation'}. \]
  Furthermore, we call $\LLSB: \RV \rightarrow \BoundSet$ a \textbf{lower} local size bound if and only if for all $(t, v') \in \RV$ and all $\valuation$ it holds that
  \[ \eval{\LLSB(t, v)}{\valuation} \geq \inf \braced{\valuation'(v) \mid \exists \valuation': \valuation \rightarrow_t \valuation'}. \]
  We call $\LSB$ a local size bound.
\end{definition}
