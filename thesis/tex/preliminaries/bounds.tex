\subsection{Bounds}

In our setup, we need a definition of the kind of bounds, we want to compute.

\begin{definition}[Bound set]
  The set $\BoundSet$ of possible bounds is the smallest set with
  \[ \infty \in \BoundSet \]
  \[ \mathbb{N} \subset \BoundSet \] 
  \[ \VSet \subset \BoundSet \] 
  \[ -b \in \BoundSet \text{ for all } b \in \BoundSet \] 
  \[ b_1 + b_2 \in \BoundSet \text{ for all } b_1, b_2 \in \BoundSet \] 
  \[ b_1 \cdot b_2 \in \BoundSet \text{ for all } b_1, b_2 \in \BoundSet \] 
  \[ \max(b_1, b_2) \in \BoundSet \text{ for all } b_1, b_2 \in \BoundSet \]
  \[ k^b \in \BoundSet \text{ for all } k \in \mathbb{N}, b \in \BoundSet \]
\end{definition}

We use the common symbol $\infty$ to represent infinity.
With the defined negation $-$, we can also represent minus infinity by $-\infty$.
This is useful for lower size bounds, where $-\infty$ is a valid lower size bound for every possible evaluation.

We include numbers from $\mathbb{N}$.
All numbers from $\mathbb{Z}$ are also included in the bound set $\BoundSet$, since for each number $n \in \mathbb{N}$ its negation $-n$ is also element of the bound set $\BoundSet$.
As the third base component, we include the program variables $\VSet$ in the bound set $\BoundSet$.

We define the usual addition $+$ and multiplication $\cdot$ between two bounds.
The maximum operator $\max$ is also included with the common meaning.
For the representation of exponential bounds we include $k^b$ for all positive integers $k \in \mathbb{N}$ and all bounds $b \in \BoundSet$.
Therefore, we do not allow arbitrary exponentiation.

In specific scenarios, we only allow a subset of those bounds.
For this purpose, we define with $\BoundSet_p \subseteq \BoundSet$ the subset of all polynomials in $\BoundSet$.
\begin{definition}[Polynomial set]
  The set $\BoundSet_p \subseteq \BoundSet$ of all polynomials is the smallest set with
  \[ \mathbb{N} \subset \BoundSet_p \] 
  \[ \VSet \subset \BoundSet_p \] 
  \[ -b \in \BoundSet_p \text{ for all } b \in \BoundSet_p \] 
  \[ b_1 + b_2 \in \BoundSet_p \text{ for all } b_1, b_2 \in \BoundSet_p \] 
  \[ b_1 \cdot b_2 \in \BoundSet_p \text{ for all } b_1, b_2 \in \BoundSet_p \] 
\end{definition}
Furthermore we define the set of all affine polynomials as $\BoundSet_a \subseteq \BoundSet_p \subseteq \BoundSet$. 
\begin{definition}[Affine polynomial set]
  The set $\BoundSet_a \subseteq \BoundSet_p$ of all affine polynomials is the smallest set with
  \[ a \cdot v \in \BoundSet_a \text{ for all } a \in \mathbb{N}, v \in \VSet \] 
  \[ -a \cdot v \in \BoundSet_a \text{ for all } a \in \mathbb{N}, v \in \VSet \] 
  \[ b_1 + b_2 \in \BoundSet_a \text{ for all } b_1, b_2 \in \BoundSet_p \] 
\end{definition}
Note that we allow some additional notations to simplify the representation of some bounds.
\begin{align*}
  \min(b_1,b_2) &= -\max(-b_1,-b_2) & \text{ for all } b_1, b_2 \in \BoundSet \\
  \max S &= \max(b_1, \max(b_2, \dots \max(\dots, b_n))) & \text{ for all } S = \braced{b_1, \dots, b_n} \subseteq \BoundSet \text{ with } S \neq \emptyset \\
  \max S &= -\infty & \text{ for } S = \emptyset \\
  \maxO{b} &= \max(b,0) & \text{ for all } b \in \BoundSet \\
  \abs{b} &= \maxO{b} + \maxO{-b} & \text{ for all } b \in \BoundSet \\
  b_1 - b_2 &= b_1 + (-b_2) & \text{ for all } b_1, b_2 \in \BoundSet \\
  b^n &= \underbrace{b \cdot b \dots b \cdot b}_\text{n times} & \text{ for all } b \in \BoundSet, n \in \mathbb{N}
\end{align*}
Therefore, the minimum operator $\min$, the subtraction $-$, the absolute value $\abs{\cdot}$ and the exponentiation with a natural number are defined with the expected meaning.
The operator $\maxO{\cdot}$ is used as a abbreviation because of its high usage.

We now present some examples of bounds, which are included in the defined sets $\BoundSet$, $\BoundSet_p$ and $\BoundSet_a$.
\begin{align*}
  0 \cdot x \in \BoundSet_a \subset \BoundSet_p \subset \BoundSet \\
  2 \cdot x - 3 \cdot y \in \BoundSet_a \subset \BoundSet_p \subset \BoundSet \\
  2 \cdot x^2 + 4 \cdot z \in \BoundSet_p \subset \BoundSet \\
  2^{\max(x,3 \cdot y^2)} \cdot 7 \cdot x^3 \in \BoundSet
\end{align*}

After this informal presentation of the meaning of the bounds, we will now formally introduce the formal semantics.
We refer to an assignment of variables $v \in \VSet$ to values from $\mathbb{Z}_\bot$ as state.
We define the set of values $\mathbb{Z}_\bot$ as $\mathbb{Z} \cup \braced{\bot}$ and denote with $\bot$ the undefined value.
\todo{Is it? Multiplication with 0?}{The application of any operator is strict, therefore the result of any application to values, where any value is undefined, is also undefined.}

\begin{definition}[Bound evaluation]
  For a bound $b \in \BoundSet$ and a state $\valuation \in \VSet \rightarrow \mathbb{Z}_\bot$ we define $\eval{b}{\valuation}$ to denote the evaluation of the bound $b$ to a value in $\mathbb{Z}_\bot$.
  \[ \eval{\infty}{\valuation} = \bot \text{ for } \infty \in \BoundSet \]
  \[ \eval{k}{\valuation} = k \text{ for all } k \in \mathbb{N} \subset \BoundSet \] 
  \[ \eval{v}{\valuation} = \valuation(v) \text{ for all } v \in \VSet \subset \BoundSet \] 
  \[ \eval{-b}{\valuation} = -\eval{b}{\valuation} \text{ for all } b \in \BoundSet \] 
  \[ \eval{b_1 + b_2}{\valuation} = \eval{b_1}{\valuation} + \eval{b_2}{\valuation} \text{ for all } b_1, b_2 \in \BoundSet \] 
  \[ \eval{b_1 \cdot b_2}{\valuation} = \eval{b_1}{\valuation} \cdot \eval{b_2}{\valuation} \text{ for all } b_1, b_2 \in \BoundSet \] 
  \[ \eval{\max(b_1, b_2)}{\valuation} = \max(\eval{b_1}{\valuation}, \eval{b_2}{\valuation}) \text{ for all } b_1, b_2 \in \BoundSet \]
  \[ \eval{k^b}{\valuation} = k^{\eval{b}{\valuation}} \text{ for all } k \in \mathbb{N} \subset \BoundSet, b \in \BoundSet \]  
  For a bound $b \in \BoundSet$ and a state $\valuation \in \VSet \rightarrow \BoundSet$, we write $\eval{b}{\valuation}$ to denote the substitution of every variable $v$ of the bound $b$ with the appropriate value $\valuation(v) \in \BoundSet$.
\end{definition}
The mentioned examples can therefore be evaluated as expected.
\begin{align*}
  \eval{0 \cdot x}{\valuation} &= 0 \cdot \valuation(x) \\
  \eval{2 \cdot x - 3 \cdot y}{\valuation} &= 2 \cdot \valuation(x) - 3 \cdot \valuation(y) \\
  \eval{2 \cdot x^2 + 4 \cdot z}{\valuation} &= 2 \cdot \valuation(x)^2 + 4 \cdot \valuation(z) \\
  \eval{2^{\max(x,3 \cdot y^2)} \cdot 7 \cdot x^3}{\valuation} &= 2^{\max(\valuation(x),3 \cdot \valuation(y)^2)} \cdot 7 \cdot \valuation(x)^3
\end{align*}

Additionally, we define the substitution for a subset of the bounds.
Instead of a substitution which replaces variables with a single substitution $\delta: \VSet \rightarrow \BoundSet$, we define a substitution with a case distinction for monotonically increasing and monotonically decreasing variables.
In a substitution $\subst{b}{\delta_1}{\delta_2}$, a monotonically increasing variable is substituted with the substitution $\delta_2: \VSet \rightarrow \BoundSet$ and a monotonically decreasing variable is substituted with the substitution $\delta_1: \VSet \rightarrow \BoundSet$.
If the variable is not monotonous in the bound $b$, then the substitution is undefined.
\begin{definition}[Bound substitution]
  For a bound $b \in \BoundSet$ and two substitutions $\delta_1, \delta_2 \in \VSet \rightarrow \BoundSet$, we define $\subst{b}{\delta_1}{\delta_2}$ to denote the substitution of the variables of the bound $b$ with bounds from the substitutions $\delta_1, \delta_2$.
  \[ \subst{\infty}{\delta_1}{\delta_2} = \infty \text{ for } \infty \in \BoundSet \]
  \[ \subst{k}{\delta_1}{\delta_2} = k \text{ for all } k \in \mathbb{N} \subset \BoundSet \] 
  \[ \subst{v}{\delta_1}{\delta_2} = \delta_2(v) \text{ for all } v \in \VSet \subset \BoundSet \] 
  \[ \subst{-b}{\delta_1}{\delta_2} = -\subst{b}{\delta_2}{\delta_1} \text{ for all } b \in \BoundSet \] 
  \[ \subst{b_1 + b_2}{\delta_1}{\delta_2} = \subst{b_1}{\delta_1}{\delta_2} + \subst{b_2}{\delta_1}{\delta_2} \text{ for all } b_1, b_2 \in \BoundSet \] 
  \[ \subst{k \cdot b}{\delta_1}{\delta_2} = k \cdot \subst{b}{\delta_1}{\delta_2} \text{ for all } b \in \BoundSet \text{ and } k \in \mathbb{N} \] 
  \[ \subst{\max(b_1, b_2)}{\delta_1}{\delta_2} = \max(\subst{b_1}{\delta_1}{\delta_2}, \subst{b_2}{\delta_1}{\delta_2}) \text{ for all } b_1, b_2 \in \BoundSet \]
  For all other $b \in \BoundSet$ the substitution $\subst{b}{\delta_1}{\delta_2}$ is undefined.
  We also write $\subst{b}{\delta}{}$ to denote $\subst{b}{\delta}{\delta}$.
\end{definition}

For all polynomials, we also define polynomial inequalities.

\begin{definition}[Polynomial inequality]
  Let $p_1, p_2 \in \BoundSet_p$ be two polynomials.
  Let $\bowtie \in \braced{\leq, \geq, <, >, =}$ be an arbitrary comparator.
  We define $p_1 \bowtie p_2$ as a polynomial inequality between $p_1$ and $p_2$.
  We define the atom set
  \[ \AtomSet = \braced{p_1 \bowtie p_2 \mid \exists p_1, p_2 \in \BoundSet_p, \exists \bowtie \in \braced{\leq, \geq, <, >, =}} \]
  as the set of all polynomial inequalities.
  Furthermore, we define the constraint set $\ConstraintSet = 2^\AtomSet$ as set of conjunctions between polynomial inequalities.
\end{definition}
Also, we extend the definition of the bound evaluation to polynomial inequalities.

\begin{definition}[Evaluation of polynomial inequalities]
  Let $\valuation$ be a state $\valuation \in \VSet \rightarrow \mathbb{Z}_\bot$.
  For an atom $b_1 \bowtie b_2 \in \AtomSet$, we define $\eval{b_1 \bowtie b_2}{\valuation}$ to denote the evaluation of the atom $b_1 \bowtie b_2$ to a value in $\mathbb{B}_\bot$.
  \[ \eval{b_1 \bowtie b_2}{\valuation} = \eval{b_1}{\valuation} \bowtie \eval{b_2}{\valuation} \]
  For a constraint $\guard \in \ConstraintSet$, we define $\eval{\guard}{\valuation}$ to denote the evaluation of the constraint $\guard$ to a value in $\mathbb{B}_\bot$.
  \[ \eval{\guard}{\valuation} = \bigwedge_{\atom \in \guard} \eval{\atom}{\valuation} \]
\end{definition}
