\subsection{Bounds}

In our setup we need a definition of the kind of bounds we want to compute.

\begin{definition}[Bound set]
  The set $\BoundSet$ of possible bounds is the smallest set with
  \[ \infty \in \BoundSet \]
  \[ \mathbb{N} \subset \BoundSet \] 
  \[ \VSet \subset \BoundSet \] 
  \[ -b \in \BoundSet \text{ for all } b \in \BoundSet \] 
  \[ b_1 + b_2 \in \BoundSet \text{ for all } b_1, b_2 \in \BoundSet \] 
  \[ b_1 \cdot b_2 \in \BoundSet \text{ for all } b_1, b_2 \in \BoundSet \] 
  \[ \max(b_1, b_2) \in \BoundSet \text{ for all } b_1, b_2 \in \BoundSet \]
  \[ k^b \in \BoundSet \text{ for all } k \in \mathbb{N}, b \in \BoundSet \]
\end{definition}

We use $\infty$ to represent infinity.
Together with the defined negation we can also represent minus infinity by $-\infty$.
This is not necessary for time bounds, since time bounds are only allowed to be positive.
But we use this for lower size bounds, where $-\infty$ is a valid lower size bound for every possible evaluation.

We include numbers from $\mathbb{N}$.
All numbers from $\mathbb{Z}$ are also in in the bound set $\BoundSet$, since for each number $n \in \mathbb{N}$ its negation $-n$ is element of the bound set $\BoundSet$.
As the third base component we add the program variables $\VSet$ to the bound set $\BoundSet$.

We define the usual addition $+$ and multiplication $\cdot$ between two bounds.
The maximum operator $\max$ is also included with the common meaning.
Note that the bound $-\max(-b_1,-b_2)$ is also included in the bound set $\BoundSet$ and that we therefore also have a representation of $\min(b_1,b_2)$.
Also note that for a set of bounds $S \subseteq \BoundSet$ we denote with $\max S$ the maximum value of the whole set.
For an empty set of bounds $S = \emptyset$ we define $\max S$ to be $-\infty$.

For the representation of exponential bounds we include $k^b$ for all positive integers $k \in \mathbb{N}$ and all bounds $b \in \BoundSet$.
Therefore we do not allow arbitrary exponentation.

Note that the bound set $\BoundSet$ also includes all polynomials over the variable set $\VSet$ by definition.
\todo{Maybe concrete definition}{We define $\BoundSet_p \subseteq \BoundSet$ as the subset of all polynomials in $\BoundSet$.}

\todo{Examples}{}

We will now formally introduce an evaluation operation on all elements of the bound set.

\begin{definition}[Bound evaluation]
  For a bound $b \in \BoundSet$ and a state $\valuation \in \VSet \rightarrow \mathbb{Z}_\bot$ we define $\eval{b}{\valuation}$ to denote the evaluation of the bound $b$ to a value in $\mathbb{Z}_\bot$.
  \[ \eval{\infty}{\valuation} = \bot \text{ for } \infty \in \BoundSet \] \todo{Really bot?}
  \[ \eval{k}{\valuation} = k \text{ for all } k \in \mathbb{N} \subset \BoundSet \] 
  \[ \eval{v}{\valuation} = \valuation(v) \text{ for all } v \in \VSet \subset \BoundSet \] 
  \[ \eval{-b}{\valuation} = -\eval{b}{\valuation} \text{ for all } b \in \BoundSet \] 
  \[ \eval{b_1 + b_2}{\valuation} = \eval{b_1}{\valuation} + \eval{b_2}{\valuation} \text{ for all } b_1, b_2 \in \BoundSet \] 
  \[ \eval{b_1 \cdot b_2}{\valuation} = \eval{b_1}{\valuation} \cdot \eval{b_2}{\valuation} \text{ for all } b_1, b_2 \in \BoundSet \] 
  \[ \eval{\max(b_1, b_2)}{\valuation} = \max(\eval{b_1}{\valuation}, \eval{b_2}{\valuation}) \text{ for all } b_1, b_2 \in \BoundSet \]
  \[ \eval{k^b}{\valuation} = k^{\eval{b}{\valuation}} \text{ for all } k \in \mathbb{N} \subset \BoundSet, b \in \BoundSet \]  
  For a bound $b \in \BoundSet$ and a bound assignment $\valuation \in \VSet \rightarrow \BoundSet$ we write $\eval{b}{\valuation}$ to denote the substitution of every variable $v$ of the bound $b$ with the appropriate value $\valuation(v) \in \BoundSet$.
\end{definition}
Furthermore, we define the comparison of bounds.

\begin{definition}[Comparison sets]
  We define the atom set $\AtomSet$ as $\braced{p_1 \bowtie p_2 \mid \exists p_1, p_2 \in \BoundSet_p, \exists \bowtie \in \braced{\leq, \geq, <, >, =}}$.
  We define the constraint set $\ConstraintSet$ as $2^\AtomSet$.
\end{definition}
Also, we extend the definition of the bound evaluation to comparison sets.

\begin{definition}[Comparison evaluation]
  Let $\valuation$ be a state $\valuation \in \VSet \rightarrow \mathbb{Z}_\bot$.
  For an atom $b_1 \bowtie b_2 \in \AtomSet$ we define $\eval{b_1 \bowtie b_2}{\valuation}$ to denote the evaluation of the atom $b_1 \bowtie b_2$ to a value in $\mathbb{B}_\bot$.
  \[ \eval{b_1 \bowtie b_2}{\valuation} = \eval{b_1}{\valuation} \bowtie \eval{b_2}{\valuation} \]
  For a constraint $\guard \in \ConstraintSet$ we define $\eval{\guard}{\valuation}$ to denote the evaluation of the constraint $\guard$ to a value in $\mathbb{B}_\bot$.
  \[ \eval{\guard}{\valuation} = \bigwedge_{\atom \in \guard} \eval{\atom}{\valuation} \]
\end{definition}
