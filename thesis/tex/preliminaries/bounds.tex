\subsection{Bounds}

In our setup, we need a definition of the kind of bounds, we want to compute.

\begin{definition}[Bound set]
  Let $\VSet$ be finite set of variables.
  The set $\BoundSet(\VSet)$ of possible bounds is the smallest set with
  \[ \infty \in \BoundSet(\VSet) \]
  \[ \mathbb{N} \subseteq \BoundSet(\VSet) \] 
  \[ \VSet \subseteq \BoundSet(\VSet) \] 
  \[ -b \in \BoundSet(\VSet) \text{ for all } b \in \BoundSet(\VSet) \] 
  \[ b_1 + b_2 \in \BoundSet(\VSet) \text{ for all } b_1, b_2 \in \BoundSet(\VSet) \] 
  \[ b_1 \cdot b_2 \in \BoundSet(\VSet) \text{ for all } b_1, b_2 \in \BoundSet(\VSet) \] 
  \[ \max(b_1, b_2) \in \BoundSet(\VSet) \text{ for all } b_1, b_2 \in \BoundSet(\VSet) \]
  \[ k^b \in \BoundSet(\VSet) \text{ for all } k \in \mathbb{N}, b \in \BoundSet(\VSet) \]
\end{definition}

We use the common symbol $\infty$ to represent infinity.
With the defined negation $-$, we can also represent minus infinity by $-\infty$.
This is useful for lower size bounds, where $-\infty$ is a valid lower size bound for every possible evaluation.

We include numbers from $\mathbb{N}$.
All numbers from $\mathbb{Z}$ are implicitly included in the bound set $\BoundSet(\VSet)$, since for each number $n \in \mathbb{N}$ its negation $-n$ is also element of the bound set $\BoundSet(\VSet)$.
As the third base component, we include the variable set $\VSet$ itself in the bound set $\BoundSet(\VSet)$.

We define the usual addition $+$ and multiplication $\cdot$ between two bounds.
The maximum operator $\max$ is also included with the common meaning.
For the representation of exponential bounds we include $k^b$ for all positive integers $k \in \mathbb{N}$ and all bounds $b \in \BoundSet(\VSet)$.
Therefore, we do not allow arbitrary exponentiation.

In specific scenarios, we only allow a subset of those bounds.
For this purpose, we define with $\BoundSet_p(\VSet) \subseteq \BoundSet(\VSet)$ the subset of all polynomials in $\BoundSet(\VSet)$.
\begin{definition}[Polynomial set]
  Let $\VSet$ be finite set of variables.
  The set $\BoundSet_p(\VSet) \subseteq \BoundSet(\VSet)$ of all polynomials is the smallest set with
  \[ \mathbb{N} \subset \BoundSet_p(\VSet) \] 
  \[ \VSet \subset \BoundSet_p(\VSet) \] 
  \[ -b \in \BoundSet_p(\VSet) \text{ for all } b \in \BoundSet_p(\VSet) \] 
  \[ b_1 + b_2 \in \BoundSet_p(\VSet) \text{ for all } b_1, b_2 \in \BoundSet_p(\VSet) \] 
  \[ b_1 \cdot b_2 \in \BoundSet_p(\VSet) \text{ for all } b_1, b_2 \in \BoundSet_p(\VSet) \] 
\end{definition}
Furthermore we define the set of all affine polynomials as $\BoundSet_a(\VSet) \subseteq \BoundSet_p(\VSet) \subseteq \BoundSet(\VSet)$. 
\begin{definition}[Affine polynomial set]
  Let $\VSet$ be finite set of variables.
  The set $\BoundSet_a(\VSet) \subseteq \BoundSet_p(\VSet)$ of all affine polynomials is the smallest set with
  \[ a \cdot v \in \BoundSet_a(\VSet) \text{ for all } a \in \mathbb{N}, v \in \VSet \] 
  \[ -a \cdot v \in \BoundSet_a(\VSet) \text{ for all } a \in \mathbb{N}, v \in \VSet \] 
  \[ b_1 + b_2 \in \BoundSet_a(\VSet) \text{ for all } b_1, b_2 \in \BoundSet_p(\VSet) \] 
\end{definition}
Note that we allow some additional notations to simplify the representation of some bounds.
\begin{align*}
  \min(b_1,b_2) &= -\max(-b_1,-b_2) & \text{ for all } b_1, b_2 \in \BoundSet(\VSet) \\
  \max S &= \max(b_1, \max(b_2, \dots \max(\dots, b_n))) & \text{ for all } S = \braced{b_1, \dots, b_n} \subseteq \BoundSet(\VSet) \text{ with } S \neq \emptyset \\
  \max S &= -\infty & \text{ for } S = \emptyset \\
  \maxO{b} &= \max(b,0) & \text{ for all } b \in \BoundSet(\VSet) \\
  \abs{b} &= \maxO{b} + \maxO{-b} & \text{ for all } b \in \BoundSet(\VSet) \\
  b_1 - b_2 &= b_1 + (-b_2) & \text{ for all } b_1, b_2 \in \BoundSet(\VSet) \\
  b^n &= \underbrace{b \cdot b \dots b \cdot b}_\text{n times} & \text{ for all } b \in \BoundSet(\VSet), n \in \mathbb{N}
\end{align*}
Therefore, the minimum operator $\min$, the subtraction $-$, the absolute value $\abs{\cdot}$ and the exponentiation with a natural number are defined with the expected meaning.
The operator $\maxO{\cdot}$ is used as a abbreviation because of its high usage.

We now present some examples of bounds, which are included in the defined sets $\BoundSet(\VSet)$, $\BoundSet_p(\VSet)$ and $\BoundSet_a(\VSet)$.
\begin{align*}
  0 \cdot x \in \BoundSet_a(\VSet) \subset \BoundSet_p(\VSet) \subset \BoundSet(\VSet) \\
  2 \cdot x - 3 \cdot y \in \BoundSet_a(\VSet) \subset \BoundSet_p(\VSet) \subset \BoundSet(\VSet) \\
  2 \cdot x^2 + 4 \cdot z \in \BoundSet_p(\VSet) \subset \BoundSet(\VSet) \\
  2^{\max(x,3 \cdot y^2)} \cdot 7 \cdot x^3 \in \BoundSet(\VSet)
\end{align*}

After this informal presentation of the meaning of the bounds, we will now formally introduce the formal semantics.
We refer to an assignment of variables $v \in \VSet$ to values from $\ValueSet$ as state.
We define the set of values $\ValueSet$ as the closure $\mathbb{Z} \cup \braced{\infty, -\infty}$ of $\mathbb{Z}$.
The symbol $\infty$ represents the infinity with $\infty > k$ for every $k \in \mathbb{Z}$.
The symbol $-\infty$ represents the minus infinity with $-\infty < k$ for every $k \in \mathbb{Z}$.

\begin{definition}[Exact evaluation]
  For a bound $b \in \BoundSet(\VSet)$ and a state $\valuation \in \VSet \rightarrow \mathbb{Z}$ we define $\exacteval{b}{\valuation}$ to denote the exact evaluation of the bound $b$ to a value in $\ValueSet$.
  \[ \exacteval{\infty}{\valuation} = \infty \text{ for } \infty \in \BoundSet(\VSet) \]
  \[ \exacteval{k}{\valuation} = k \text{ for all } k \in \mathbb{N} \subset \BoundSet(\VSet) \] 
  \[ \exacteval{v}{\valuation} = \valuation(v) \text{ for all } v \in \VSet \subset \BoundSet(\VSet) \] 
  \[ \exacteval{-b}{\valuation} = -\exacteval{b}{\valuation} \text{ for all } b \in \BoundSet(\VSet) \] 
  \[ \exacteval{b_1 + b_2}{\valuation} = \exacteval{b_1}{\valuation} + \exacteval{b_2}{\valuation} \text{ for all } b_1, b_2 \in \BoundSet(\VSet) \] 
  \[ \exacteval{b_1 \cdot b_2}{\valuation} = \exacteval{b_1}{\valuation} \cdot \exacteval{b_2}{\valuation} \text{ for all } b_1, b_2 \in \BoundSet(\VSet) \] 
  \[ \exacteval{\max(b_1, b_2)}{\valuation} = \max(\exacteval{b_1}{\valuation}, \exacteval{b_2}{\valuation}) \text{ for all } b_1, b_2 \in \BoundSet(\VSet) \]
  \[ \exacteval{k^b}{\valuation} = k^{\exacteval{b}{\valuation}} \text{ for all } k \in \mathbb{N} \subset \BoundSet(\VSet), b \in \BoundSet(\VSet) \]  
\end{definition}
The mentioned examples can therefore be evaluated as expected.
\begin{align*}
  \exacteval{0 \cdot x}{\valuation} &= 0 \cdot \valuation(x) \\
  \exacteval{2 \cdot x - 3 \cdot y}{\valuation} &= 2 \cdot \valuation(x) - 3 \cdot \valuation(y) \\
  \exacteval{2 \cdot x^2 + 4 \cdot z}{\valuation} &= 2 \cdot \valuation(x)^2 + 4 \cdot \valuation(z) \\
  \exacteval{2^{\max(x,3 \cdot y^2)} \cdot 7 \cdot x^3}{\valuation} &= 2^{\max(\valuation(x),3 \cdot \valuation(y)^2)} \cdot 7 \cdot \valuation(x)^3
\end{align*}

Additionally, we define the substitution for a subset of the bounds.
Instead of a substitution which replaces variables with a single substitution $\delta: \VSet \rightarrow \BoundSet(\VSet)$, we define a substitution with a case distinction for monotonically increasing and monotonically decreasing variables.
In a substitution $\apprsubst{b}{\delta_1}{\delta_2}$, a monotonically increasing variable is substituted with the substitution $\delta_2: \VSet \rightarrow \BoundSet(\VSet)$ and a monotonically decreasing variable is substituted with the substitution $\delta_1: \VSet \rightarrow \BoundSet(\VSet)$.
If the variable is not monotonous in the bound $b$, then the substitution is undefined.
\begin{definition}[Overapproximated replacement]
  Let $K$ be an arbitrary sets of elements with an addition $+: K \times K \rightarrow K$, a negation $-: K \rightarrow K$, a multiplication $\cdot: K \times K \rightarrow K$ and a max operator $\max: K \times K \rightarrow K$.
  For a bound $b \in \BoundSet(\VSet)$ and two functions $\lsub, \usub: \VSet \rightarrow K$, we denote with $\oversubst{b}{\lsub}{\usub}$ a function of the type $(\VSet \rightarrow K) \times (\VSet \rightarrow K) \rightarrow K$, which replaces the variables of the bound $b$ with values from the functions $\lsub, \usub$ such that the following equations hold.
  \[ \oversubst{\infty}{\lsub}{\usub} = \infty \text{ for } \infty \in \BoundSet(\VSet) \]
  \[ \oversubst{k}{\lsub}{\usub} = k \text{ for all } k \in \mathbb{N} \subset \BoundSet(\VSet) \] 
  \[ \oversubst{v}{\lsub}{\usub} = \usub(v) \text{ for all } v \in \VSet \subset \BoundSet(\VSet) \] 
  \[ \oversubst{-b}{\lsub}{\usub} = -\undersubst{b}{\lsub}{\usub} \text{ for all } b \in \BoundSet(\VSet) \] 
  \[ \oversubst{b_1 + b_2}{\lsub}{\usub} = \oversubst{b_1}{\lsub}{\usub} + \oversubst{b_2}{\lsub}{\usub} \text{ for all } b_1, b_2 \in \BoundSet(\VSet) \] 
  \[ \oversubst{b_1 \cdot b_2}{\lsub}{\usub} = \maximum{
    \oversubst{b_1}{\lsub}{\usub} \cdot \oversubst{b_2}{\lsub}{\usub},
    \oversubst{b_1}{\lsub}{\usub} \cdot \undersubst{b_2}{\lsub}{\usub},
    \undersubst{b_1}{\lsub}{\usub} \cdot \oversubst{b_2}{\lsub}{\usub},
    \undersubst{b_1}{\lsub}{\usub} \cdot \undersubst{b_2}{\lsub}{\usub}
  } \text{ for all } b_1, b_2 \in \BoundSet(\VSet) \] 
  \[ \oversubst{\max(b_1, b_2)}{\lsub}{\usub} = \maximum{\oversubst{b_1}{\lsub}{\usub}, \oversubst{b_2}{\lsub}{\usub}} \text{ for all } b_1, b_2 \in \BoundSet(\VSet) \]
  \[ \oversubst{k^b}{\lsub}{\usub} = k^{\oversubst{b}{\lsub}{\usub}} \text{ for all } k \in \mathbb{N} \subset \BoundSet(\VSet), b \in \BoundSet(\VSet) \]  
\end{definition}

\begin{definition}[Underapproximated replacement]
  Let $K$ be an arbitrary sets of elements with an addition $+: K \times K \rightarrow K$, a negation $-: K \rightarrow K$, a multiplication $\cdot: K \times K \rightarrow K$ and a max operator $\max: K \times K \rightarrow K$.
  For a bound $b \in \BoundSet(\VSet)$ and two functions $\lsub, \usub: \VSet \rightarrow K$, we denote with $\undersubst{b}{\lsub}{\usub}$ a function of the type $(\VSet \rightarrow K) \times (\VSet \rightarrow K) \rightarrow K$, which replaces the variables of the bound $b$ with values from the functions $\lsub, \usub$ such that the following equations hold.
  \[ \undersubst{\infty}{\lsub}{\usub} = \infty \text{ for } \infty \in \BoundSet(\VSet) \]
  \[ \undersubst{k}{\lsub}{\usub} = k \text{ for all } k \in \mathbb{N} \subset \BoundSet(\VSet) \] 
  \[ \undersubst{v}{\lsub}{\usub} = \lsub(v) \text{ for all } v \in \VSet \subset \BoundSet(\VSet) \] 
  \[ \undersubst{-b}{\lsub}{\usub} = -\oversubst{b}{\lsub}{\usub} \text{ for all } b \in \BoundSet(\VSet) \] 
  \[ \undersubst{b_1 + b_2}{\lsub}{\usub} = \undersubst{b_1}{\lsub}{\usub} + \undersubst{b_2}{\lsub}{\usub} \text{ for all } b_1, b_2 \in \BoundSet(\VSet) \] 
  \[ \undersubst{b_1 \cdot b_2}{\lsub}{\usub} = \minimum{
    \oversubst{b_1}{\lsub}{\usub} \cdot \oversubst{b_2}{\lsub}{\usub},
    \oversubst{b_1}{\lsub}{\usub} \cdot \undersubst{b_2}{\lsub}{\usub},
    \undersubst{b_1}{\lsub}{\usub} \cdot \oversubst{b_2}{\lsub}{\usub},
    \undersubst{b_1}{\lsub}{\usub} \cdot \undersubst{b_2}{\lsub}{\usub}
  } \text{ for all } b_1, b_2 \in \BoundSet(\VSet) \] 
  \[ \undersubst{\max(b_1, b_2)}{\lsub}{\usub} = \maximum{\undersubst{b_1}{\lsub}{\usub}, \undersubst{b_2}{\lsub}{\usub}} \text{ for all } b_1, b_2 \in \BoundSet(\VSet) \]
  \[ \undersubst{k^b}{\lsub}{\usub} = k^{\undersubst{b}{\lsub}{\usub}} \text{ for all } k \in \mathbb{N} \subset \BoundSet(\VSet), b \in \BoundSet(\VSet) \]  
\end{definition}

The bound set $\BoundSet(\VSet)$ meets the criteria of $K$.
Therefore, two functions $\lsub, \usub: \VSet \rightarrow \BoundSet(\VSet)$ qualify for an approximated replacement.
We call an approximated replacement with such functions a substitution.

\todo{Prove, put out of preliminaries}{}

\begin{theorem}[Soundness of approximated substitutions]
  Let $\lsub, \usub: \VSet \rightarrow \BoundSet(\VSet)$ be two substitutions with $\exacteval{\lsub(v)}{\valuation} \leq \exacteval{\usub(v)}{\valuation}$ for each variable $v \in \VSet$ and each state $\valuation \in \Valuation$.
  Let $b \in \BoundSet(\VSet)$ be a bound and $\valuation \in \Valuation$ a state.
  Then, we have
  \[ \exacteval{\undersubst{b}{\lsub}{\usub}}{\valuation} \leq \exacteval{b}{\valuation} \leq \exacteval{\oversubst{b}{\lsub}{\usub}}{\valuation}. \]
\end{theorem}

\begin{theorem}[Soundness of approximated evaluations]
  Let $\lstate, \valuation, \ustate: \VSet \rightarrow \ValueSet$ be three states with $\lstate \leq \valuation \leq \ustate$.
  Let $b \in \BoundSet(\VSet)$ be a bound.
  Then, we have
  \[ \undereval{b}{\lstate}{\ustate} \leq \exacteval{b}{\valuation} \leq \overeval{b}{\lstate}{\ustate}. \]
\end{theorem}
For all polynomials, we also define polynomial inequalities.

\begin{definition}[Polynomial inequality]
  Let $p_1, p_2 \in \BoundSet_p(\VSet)$ be two polynomials.
  Let $\bowtie \in \braced{\leq, \geq, <, >, =}$ be an arbitrary comparator.
  We define $p_1 \bowtie p_2$ as a polynomial inequality between $p_1$ and $p_2$.
  We define the atom set
  \[ \AtomSet = \braced{p_1 \bowtie p_2 \mid \exists p_1, p_2 \in \BoundSet_p(\VSet), \exists \bowtie \in \braced{\leq, \geq, <, >, =}} \]
  as the set of all polynomial inequalities.
  Furthermore, we define the constraint set $\ConstraintSet = 2^\AtomSet$ as set of conjunctions between polynomial inequalities.
\end{definition}
Also, we extend the definition of the bound evaluation to polynomial inequalities.

\begin{definition}[Evaluation of polynomial inequalities]
  Let $\valuation$ be a state $\valuation \in \VSet \rightarrow \ValueSet$.
  For an atom $b_1 \bowtie b_2 \in \AtomSet$, we define $\exacteval{b_1 \bowtie b_2}{\valuation}$ to denote the evaluation of the atom $b_1 \bowtie b_2$ to a value in $\mathbb{B}$.
  \[ \exacteval{b_1 \bowtie b_2}{\valuation} = \exacteval{b_1}{\valuation} \bowtie \exacteval{b_2}{\valuation} \]
  For a constraint $\guard \in \ConstraintSet$, we define $\exacteval{\guard}{\valuation}$ to denote the evaluation of the constraint $\guard$ to a value in $\mathbb{B}$.
  \[ \exacteval{\guard}{\valuation} = \bigwedge_{\atom \in \guard} \exacteval{\atom}{\valuation} \]
\end{definition}
