\section{Bounds}

This section introduces a set which contains the syntactic terms representing the resulting bounds.
A definition of the set as well as a definition of the semantics of the terms in this set is provided.

\begin{definition}[Bound Set]
  Let $\VSet$ be a finite set of variables.
  Let $\infty$, $-$, $+$, $\cdot$ and $\max$ be symbols.
  The set $\BoundSet(\VSet)$ of possible bounds is the smallest set with
  \[ \infty \in \BoundSet(\VSet) \]
  \[ \mathbb{N} \subseteq \BoundSet(\VSet) \] 
  \[ \VSet \subseteq \BoundSet(\VSet) \] 
  \[ -b \in \BoundSet(\VSet) \text{ for all } b \in \BoundSet(\VSet) \] 
  \[ b_1 + b_2 \in \BoundSet(\VSet) \text{ for all } b_1, b_2 \in \BoundSet(\VSet) \] 
  \[ b_1 \cdot b_2 \in \BoundSet(\VSet) \text{ for all } b_1, b_2 \in \BoundSet(\VSet) \] 
  \[ \max(b_1, b_2) \in \BoundSet(\VSet) \text{ for all } b_1, b_2 \in \BoundSet(\VSet) \]
  \[ k^b \in \BoundSet(\VSet) \text{ for all } k \in \mathbb{N}, b \in \BoundSet(\VSet) \]
\end{definition}
This definition provides a set of syntactic terms $\BoundSet(\VSet)$ for an arbitrary set of variables $\VSet$.
We assign each symbol $\infty$, $-$, $+$, $\cdot$ and $\max$ its common semantics.
This is expressed in the following definition.

\begin{definition}[Evaluation of Bounds]
  Let $\ValueSet$ be the closure $\mathbb{Z} \cup \braced{\infty, -\infty}$ of the set of integers $\mathbb{Z}$.
  Let $b \in \BoundSet(\VSet)$ be a bound and let $\valuation: \VSet \rightarrow \mathbb{Z}$ be an assignment from variables to values.
  We define $\exacteval{b}{\valuation}$ to denote the exact evaluation of the bound $b$ to a value in $\ValueSet$, i.e. $\exacteval{\cdot}{\valuation}: \BoundSet(\VSet) \rightarrow \ValueSet$.
  \[ \exacteval{\infty}{\valuation} = \infty \text{ for } \infty \in \BoundSet(\VSet) \]
  \[ \exacteval{k}{\valuation} = k \text{ for all } k \in \mathbb{N} \subseteq \BoundSet(\VSet) \] 
  \[ \exacteval{v}{\valuation} = \valuation(v) \text{ for all } v \in \VSet \subseteq \BoundSet(\VSet) \] 
  \[ \exacteval{-b}{\valuation} = -\exacteval{b}{\valuation} \text{ for all } b \in \BoundSet(\VSet) \] 
  \[ \exacteval{b_1 + b_2}{\valuation} = \exacteval{b_1}{\valuation} + \exacteval{b_2}{\valuation} \text{ for all } b_1, b_2 \in \BoundSet(\VSet) \] 
  \[ \exacteval{b_1 \cdot b_2}{\valuation} = \exacteval{b_1}{\valuation} \cdot \exacteval{b_2}{\valuation} \text{ for all } b_1, b_2 \in \BoundSet(\VSet) \] 
  \[ \exacteval{\max(b_1, b_2)}{\valuation} = \max(\exacteval{b_1}{\valuation}, \exacteval{b_2}{\valuation}) \text{ for all } b_1, b_2 \in \BoundSet(\VSet) \]
  \[ \exacteval{k^b}{\valuation} = k^{\exacteval{b}{\valuation}} \text{ for all } k \in \mathbb{N} \subset \BoundSet(\VSet), b \in \BoundSet(\VSet) \]  
\end{definition}
The symbol $\infty$ represents infinity.
With the defined negation $-$, minus infinity is represented by $-\infty$.
This is useful for lower size bounds, where $-\infty$ is a valid lower size bound for every possible evaluation.

Also, the bound set $\BoundSet(\VSet)$ includes all natural numbers $\mathbb{N}$.
Integers from $\mathbb{Z}$ are implicitly included in the bound set $\BoundSet(\VSet)$ since for each number $n \in \mathbb{N}$ its negation $-n$ is also an element of the bound set $\BoundSet(\VSet)$.
The variable set $\VSet$ itself is included in the bound set $\BoundSet(\VSet)$ as the third base component.

We define the usual addition $+$ and multiplication $\cdot$ between two bounds.
The maximum operator $\max$ is also included with its common semantics.
For the representation of exponential bounds, we include $k^b$ for all positive integers $k \in \mathbb{N}$ and all bounds $b \in \BoundSet(\VSet)$.
Arbitrary exponentiation is not included (e.g. $\forall x \in \VSet, b \in \BoundSet(\VSet): x^b \notin \BoundSet(\VSet)$).

Note that we allow some additional notations to simplify the representation of some bounds.
\begin{align*}
  \min(b_1,b_2) &\coloneqq -\max(-b_1,-b_2) & \text{ for all } b_1, b_2 \in \BoundSet(\VSet) \\
  \max S &\coloneqq \max(b_1, \max(b_2, \dots \max(\dots, b_n))) & \text{ for } S = \braced{b_1, \dots, b_n} \subseteq \BoundSet(\VSet) \text{ with } S \neq \emptyset \\
  \max \emptyset &\coloneqq -\infty \\
  \maxO{b} &\coloneqq \max(b,0) & \text{ for all } b \in \BoundSet(\VSet) \\
  \abs{b} &\coloneqq \maxO{b} + \maxO{-b} & \text{ for all } b \in \BoundSet(\VSet) \\
  b_1 - b_2 &\coloneqq b_1 + (-b_2) & \text{ for all } b_1, b_2 \in \BoundSet(\VSet) \\
  b^n &\coloneqq \underbrace{b \cdot b \dots b \cdot b}_\text{n times} & \text{ for all } b \in \BoundSet(\VSet), n \in \mathbb{N}
\end{align*}
The minimum operator $\min$, the subtraction $-$, the absolute value $\abs{\cdot}$ and the exponentiation with a natural number have the expected semantics.
The operator $\maxO{\cdot}$ is used as an abbreviation due to its high usage.

In specific scenarios, we only allow a subset of these bounds.
For this purpose, we define by $\BoundSet_p(\VSet) \subseteq \BoundSet(\VSet)$ the subset of all polynomials in $\BoundSet(\VSet)$.
This subset can be understood as an equivalent representation of the polynomial ring $\mathbb{Z}[\VSet]$.
\begin{definition}[Polynomial Set]
  Let $\VSet$ be a finite set of variables.
  The set $\BoundSet_p(\VSet) \subseteq \BoundSet(\VSet)$ of all polynomials is the smallest set with
  \[ \mathbb{N} \subseteq \BoundSet_p(\VSet) \] 
  \[ \VSet \subseteq \BoundSet_p(\VSet) \] 
  \[ -b \in \BoundSet_p(\VSet) \text{ for all } b \in \BoundSet_p(\VSet) \] 
  \[ b_1 + b_2 \in \BoundSet_p(\VSet) \text{ for all } b_1, b_2 \in \BoundSet_p(\VSet) \] 
  \[ b_1 \cdot b_2 \in \BoundSet_p(\VSet) \text{ for all } b_1, b_2 \in \BoundSet_p(\VSet) \] 
\end{definition}
Furthermore, we define the set of all affine polynomials as $\BoundSet_a(\VSet) \subseteq \BoundSet_p(\VSet) \subseteq \BoundSet(\VSet)$. 
\begin{definition}[Affine Polynomial Set]
  Let $\VSet$ be a finite set of variables.
  The set $\BoundSet_a(\VSet) \subseteq \BoundSet_p(\VSet)$ of all affine polynomials is the smallest set with
  \[ a \cdot v \in \BoundSet_a(\VSet) \text{ for all } a \in \mathbb{N}, v \in \VSet \] 
  \[ -a \cdot v \in \BoundSet_a(\VSet) \text{ for all } a \in \mathbb{N}, v \in \VSet \] 
  \[ b_1 + b_2 \in \BoundSet_a(\VSet) \text{ for all } b_1, b_2 \in \BoundSet_p(\VSet) \] 
\end{definition}

To clarify the differences between these sets, the following lines provide some examples of bounds.
\begin{align*}
  0 \cdot x \in \BoundSet_a(\VSet) \subseteq \BoundSet_p(\VSet) \subseteq \BoundSet(\VSet) \\
  2 \cdot x - 3 \cdot y \in \BoundSet_a(\VSet) \subseteq \BoundSet_p(\VSet) \subseteq \BoundSet(\VSet) \\
  2 \cdot x^2 + 4 \cdot z \in \BoundSet_p(\VSet) \subseteq \BoundSet(\VSet) \\
  2^{\max(x,3 \cdot y^2)} \cdot 7 \cdot x^3 \in \BoundSet(\VSet)
\end{align*}
The evaluation of the examples yields the expected results.
\begin{align*}
  \exacteval{0 \cdot x}{\valuation} &= 0 \cdot \valuation(x) = 0 \\
  \exacteval{2 \cdot x - 3 \cdot y}{\valuation} &= 2 \cdot \valuation(x) - 3 \cdot \valuation(y) \\
  \exacteval{2 \cdot x^2 + 4 \cdot z}{\valuation} &= 2 \cdot \valuation(x)^2 + 4 \cdot \valuation(z) \\
  \exacteval{2^{\max(x,3 \cdot y^2)} \cdot 7 \cdot x^3}{\valuation} &= 2^{\max(\valuation(x),3 \cdot \valuation(y)^2)} \cdot 7 \cdot \valuation(x)^3
\end{align*}
For the subset of polynomials, we define an atom set $\AtomSet(\VSet)$ to express polynomial inequalities and a constraint set $\ConstraintSet(\VSet)$ to express conjunctions of polynomial inequalities.

\begin{definition}[Atom Set and Constraint Set]
  Let $\VSet$ be a finite set of variables.
  Let $p_1, p_2 \in \BoundSet_p(\VSet)$ be two polynomials.
  Let $\bowtie \in \braced{\leq, \geq, <, >, =}$ be an arbitrary comparator.
  We define $p_1 \bowtie p_2$ as a polynomial inequality between $p_1$ and $p_2$.
  We define the atom set
  \[ \AtomSet(\VSet) = \left\{ p_1 \bowtie p_2 \mid p_1, p_2 \in \BoundSet_p(\VSet), \bowtie \in \braced{\leq, \geq, <, >, =} \right\} \]
  as the set of all polynomial inequalities.
  Furthermore, we define the constraint set $\ConstraintSet(\VSet) = 2^{\AtomSet(\VSet)}$ as the set of conjunctions of polynomial inequalities.
\end{definition}
Also, we extend the definition of the bound evaluation to polynomial inequalities.

\begin{definition}[Evaluation of Atoms and Constraints]
  Let $\valuation: \VSet \rightarrow \mathbb{Z}$ be an assignment from variables to values.
  For an atom $b_1 \bowtie b_2 \in \AtomSet(\VSet)$, we define $\exacteval{b_1 \bowtie b_2}{\valuation}$ to denote the evaluation of the atom $b_1 \bowtie b_2$ to a boolean value in $\braced{\emph{true}, \emph{false}}$.
  \[ \exacteval{b_1 \bowtie b_2}{\valuation} = \exacteval{b_1}{\valuation} \bowtie \exacteval{b_2}{\valuation} \]
  For a constraint $\guard \in \ConstraintSet(\VSet)$, we define $\exacteval{\guard}{\valuation}$ to denote the evaluation of the constraint $\guard$ to a boolean value in $\braced{\emph{true}, \emph{false}}$.
  \[ \exacteval{\guard}{\valuation} = \bigwedge_{\atom \in \guard} \exacteval{\atom}{\valuation} \]
\end{definition}
