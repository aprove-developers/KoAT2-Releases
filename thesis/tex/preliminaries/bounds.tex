\subsection{Bounds}

In this section, we introduce a set which contains the syntactic terms that represent the resulting bounds.
Additionally, we formally define the semantics of those terms.

\begin{definition}[Bound set]
  Let $\VSet$ be a finite set of variables.
  Let $\infty$, $-$, $+$, $\cdot$ and $\max$ be symbols.
  The set $\BoundSet(\VSet)$ of possible bounds is the smallest set with
  \[ \infty \in \BoundSet(\VSet) \]
  \[ \mathbb{N} \subseteq \BoundSet(\VSet) \] 
  \[ \VSet \subseteq \BoundSet(\VSet) \] 
  \[ -b \in \BoundSet(\VSet) \text{ for all } b \in \BoundSet(\VSet) \] 
  \[ b_1 + b_2 \in \BoundSet(\VSet) \text{ for all } b_1, b_2 \in \BoundSet(\VSet) \] 
  \[ b_1 \cdot b_2 \in \BoundSet(\VSet) \text{ for all } b_1, b_2 \in \BoundSet(\VSet) \] 
  \[ \max(b_1, b_2) \in \BoundSet(\VSet) \text{ for all } b_1, b_2 \in \BoundSet(\VSet) \]
  \[ k^b \in \BoundSet(\VSet) \text{ for all } k \in \mathbb{N}, b \in \BoundSet(\VSet) \]
\end{definition}
This definition provides a set of syntactic terms $\BoundSet(\VSet)$ for an arbitrary set of variables $\VSet$.
We assign each symbol $\infty$, $-$, $+$, $\cdot$ and $\max$ its common semantics.
This is expressed in the following definition.

\begin{definition}[Evaluation of bounds]
  Let $\ValueSet$ be the closure $\mathbb{Z} \cup \braced{\infty, -\infty}$ of the set of integers $\mathbb{Z}$.
  Let $b \in \BoundSet(\VSet)$ be a bound and let $\valuation \in \VSet \rightarrow \mathbb{Z}$ be an assignment from variables to values.
  We define $\exacteval{b}{\valuation}$ to denote the exact evaluation of the bound $b$ to a value in $\ValueSet$.
  \[ \exacteval{\infty}{\valuation} = \infty \text{ for } \infty \in \BoundSet(\VSet) \]
  \[ \exacteval{k}{\valuation} = k \text{ for all } k \in \mathbb{N} \subset \BoundSet(\VSet) \] 
  \[ \exacteval{v}{\valuation} = \valuation(v) \text{ for all } v \in \VSet \subset \BoundSet(\VSet) \] 
  \[ \exacteval{-b}{\valuation} = -\exacteval{b}{\valuation} \text{ for all } b \in \BoundSet(\VSet) \] 
  \[ \exacteval{b_1 + b_2}{\valuation} = \exacteval{b_1}{\valuation} + \exacteval{b_2}{\valuation} \text{ for all } b_1, b_2 \in \BoundSet(\VSet) \] 
  \[ \exacteval{b_1 \cdot b_2}{\valuation} = \exacteval{b_1}{\valuation} \cdot \exacteval{b_2}{\valuation} \text{ for all } b_1, b_2 \in \BoundSet(\VSet) \] 
  \[ \exacteval{\max(b_1, b_2)}{\valuation} = \max(\exacteval{b_1}{\valuation}, \exacteval{b_2}{\valuation}) \text{ for all } b_1, b_2 \in \BoundSet(\VSet) \]
  \[ \exacteval{k^b}{\valuation} = k^{\exacteval{b}{\valuation}} \text{ for all } k \in \mathbb{N} \subset \BoundSet(\VSet), b \in \BoundSet(\VSet) \]  
\end{definition}
The symbol $\infty$ is used to represent its common meaning infinity.
With the defined negation $-$, we can also represent minus infinity by $-\infty$.
This is useful for lower size bounds, where $-\infty$ is a valid lower size bound for every possible evaluation.

Also, the bound set $\BoundSet$ includes all natural numbers $\mathbb{N}$.
Integers from $\mathbb{Z}$ are implicitly included in the bound set $\BoundSet(\VSet)$, since for each number $n \in \mathbb{N}$ its negation $-n$ is also element of the bound set $\BoundSet(\VSet)$.
The variable set $\VSet$ itself is included in the bound set $\BoundSet(\VSet)$ as the third base component.

We define the usual addition $+$ and multiplication $\cdot$ between two bounds.
The maximum operator $\max$ is also included with the common meaning.
For the representation of exponential bounds we include $k^b$ for all positive integers $k \in \mathbb{N}$ and all bounds $b \in \BoundSet(\VSet)$.
Arbitrary exponentiation is not included (e.g. $x^y \notin \BoundSet$).

Note that we allow some additional notations to simplify the representation of some bounds.
\begin{align*}
  \min(b_1,b_2) &:= -\max(-b_1,-b_2) & \text{ for all } b_1, b_2 \in \BoundSet(\VSet) \\
  \max S &:= \max(b_1, \max(b_2, \dots \max(\dots, b_n))) & \text{ for } S = \braced{b_1, \dots, b_n} \subseteq \BoundSet(\VSet) \text{ with } S \neq \emptyset \\
  \max S &:= -\infty & \text{ for } S = \emptyset \\
  \maxO{b} &:= \max(b,0) & \text{ for all } b \in \BoundSet(\VSet) \\
  \abs{b} &:= \maxO{b} + \maxO{-b} & \text{ for all } b \in \BoundSet(\VSet) \\
  b_1 - b_2 &:= b_1 + (-b_2) & \text{ for all } b_1, b_2 \in \BoundSet(\VSet) \\
  b^n &:= \underbrace{b \cdot b \dots b \cdot b}_\text{n times} & \text{ for all } b \in \BoundSet(\VSet), n \in \mathbb{N}
\end{align*}
The minimum operator $\min$, the subtraction $-$, the absolute value $\abs{\cdot}$ and the exponentiation with a natural number are defined with the expected meaning.
The operator $\maxO{\cdot}$ is used as a abbreviation due to its high usage.

In specific scenarios, we only allow a subset of those bounds.
For this purpose, we define with $\BoundSet_p(\VSet) \subseteq \BoundSet(\VSet)$ the subset of all polynomials in $\BoundSet(\VSet)$.
This subset can be understood as different representation for the polynomial ring $\mathbb{Z}[\VSet]$.
\begin{definition}[Polynomial set]
  Let $\VSet$ be a finite set of variables.
  The set $\BoundSet_p(\VSet) \subseteq \BoundSet(\VSet)$ of all polynomials is the smallest set with
  \[ \mathbb{N} \subset \BoundSet_p(\VSet) \] 
  \[ \VSet \subset \BoundSet_p(\VSet) \] 
  \[ -b \in \BoundSet_p(\VSet) \text{ for all } b \in \BoundSet_p(\VSet) \] 
  \[ b_1 + b_2 \in \BoundSet_p(\VSet) \text{ for all } b_1, b_2 \in \BoundSet_p(\VSet) \] 
  \[ b_1 \cdot b_2 \in \BoundSet_p(\VSet) \text{ for all } b_1, b_2 \in \BoundSet_p(\VSet) \] 
\end{definition}
Furthermore, we define the set of all affine polynomials as $\BoundSet_a(\VSet) \subseteq \BoundSet_p(\VSet) \subseteq \BoundSet(\VSet)$. 
\begin{definition}[Affine polynomial set]
  Let $\VSet$ be a finite set of variables.
  The set $\BoundSet_a(\VSet) \subseteq \BoundSet_p(\VSet)$ of all affine polynomials is the smallest set with
  \[ a \cdot v \in \BoundSet_a(\VSet) \text{ for all } a \in \mathbb{N}, v \in \VSet \] 
  \[ -a \cdot v \in \BoundSet_a(\VSet) \text{ for all } a \in \mathbb{N}, v \in \VSet \] 
  \[ b_1 + b_2 \in \BoundSet_a(\VSet) \text{ for all } b_1, b_2 \in \BoundSet_p(\VSet) \] 
\end{definition}

For the clarification of the differences between those sets, the following lines provide some examples of bounds, which are included in the defined sets $\BoundSet(\VSet)$, $\BoundSet_p(\VSet)$ and $\BoundSet_a(\VSet)$.
\begin{align*}
  0 \cdot x \in \BoundSet_a(\VSet) \subset \BoundSet_p(\VSet) \subset \BoundSet(\VSet) \\
  2 \cdot x - 3 \cdot y \in \BoundSet_a(\VSet) \subset \BoundSet_p(\VSet) \subset \BoundSet(\VSet) \\
  2 \cdot x^2 + 4 \cdot z \in \BoundSet_p(\VSet) \subset \BoundSet(\VSet) \\
  2^{\max(x,3 \cdot y^2)} \cdot 7 \cdot x^3 \in \BoundSet(\VSet)
\end{align*}
The evaluation of the examples yields the expected results.
\begin{align*}
  \exacteval{0 \cdot x}{\valuation} &= 0 \cdot \valuation(x) \\
  \exacteval{2 \cdot x - 3 \cdot y}{\valuation} &= 2 \cdot \valuation(x) - 3 \cdot \valuation(y) \\
  \exacteval{2 \cdot x^2 + 4 \cdot z}{\valuation} &= 2 \cdot \valuation(x)^2 + 4 \cdot \valuation(z) \\
  \exacteval{2^{\max(x,3 \cdot y^2)} \cdot 7 \cdot x^3}{\valuation} &= 2^{\max(\valuation(x),3 \cdot \valuation(y)^2)} \cdot 7 \cdot \valuation(x)^3
\end{align*}
For the subset of polynomials, we define an atom set $\AtomSet$ to express polynomial inequalities and a constraint set $\ConstraintSet$ to express conjunctions of polynomial inequalities.

\begin{definition}[Atom set and constraint set]
  Let $p_1, p_2 \in \BoundSet_p(\VSet)$ be two polynomials.
  Let $\bowtie \in \braced{\leq, \geq, <, >, =}$ be an arbitrary comparator.
  We define $p_1 \bowtie p_2$ as a polynomial inequality between $p_1$ and $p_2$.
  We define the atom set
  \[ \AtomSet = \braced{p_1 \bowtie p_2 \mid \exists p_1, p_2 \in \BoundSet_p(\VSet), \exists \bowtie \in \braced{\leq, \geq, <, >, =}} \]
  as the set of all polynomial inequalities.
  Furthermore, we define the constraint set $\ConstraintSet = 2^\AtomSet$ as set of conjunctions between polynomial inequalities.
\end{definition}
Also, we extend the definition of the bound evaluation to polynomial inequalities.

\begin{definition}[Evaluation of atoms and constraints]
  Let $\valuation \in \VSet \rightarrow \mathbb{Z}$ be an assignment from variables to values.
  For an atom $b_1 \bowtie b_2 \in \AtomSet$, we define $\exacteval{b_1 \bowtie b_2}{\valuation}$ to denote the evaluation of the atom $b_1 \bowtie b_2$ to a value in $\mathbb{B}$.
  \[ \exacteval{b_1 \bowtie b_2}{\valuation} = \exacteval{b_1}{\valuation} \bowtie \exacteval{b_2}{\valuation} \]
  For a constraint $\guard \in \ConstraintSet$, we define $\exacteval{\guard}{\valuation}$ to denote the evaluation of the constraint $\guard$ to a value in $\mathbb{B}$.
  \[ \exacteval{\guard}{\valuation} = \bigwedge_{\atom \in \guard} \exacteval{\atom}{\valuation} \]
\end{definition}
