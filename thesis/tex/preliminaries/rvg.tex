\subsection{Result Variable Graph}

For the analysis of size bounds, it is helpful to consider a graph which contains information about which result variables affect each other.
Such a graph was presented by the original KoAT paper \cite{koat} as a result variable graph.
Before we can define the actual result variable graph, we need to define the terms of pre-transitions and active variables.
The pre-transitions of a transition $t \in \TSet$ are all transitions, which may occur immediately before an evaluation step with the transition $t$.

\begin{definition}[Pre-transitions] 
  We define $\pre: \TSet \rightarrow 2^\TSet$ as
  \[\pre(t) = \braced{\pret \in \TSet \mid \exists \valuation_0, \location, \valuation: (\location_0, \valuation_0) (\rightarrow^* \circ \rightarrow_\pret \circ \rightarrow_{t}) (\location, \valuation)}\]
  to denote the set of all transitions that may precede the transition $t$ in an evaluation.	
\end{definition}

If $\pre(t)$ represents the set of all transitions that may precede the transition $t$ in an evaluation, then $\pre_{\mathit{trivial}}((\location_1, \update_1, \guard_1, \location_1')) = \braced{(\location_2, \update_2, \guard_2, \location_2') \in \TSet \mid \location_2' = \location_1}$ is a valid over approximation of $\pre(t)$.

\begin{figure}
\centering
\begin{tikzpicture}[->,>=stealth',auto,node distance=5cm,
    thick,
    main node/.style={circle,draw,font=\sffamily\Large\bfseries},
    aligned edge/.style={align=left}]

  \node[main node] (0) {$\location_0$};
  \node[main node] (1) [right of=0] {$\location_1$};
  \node[main node] (2) [right of=1] {$\location_2$};

  \path[every node/.style={font=\sffamily\small}]
    (0) edge[aligned edge] node[above=0.2cm] {$t_0$} node[below=0.2cm] {$\update = \text{id}$} (1)
    (1) edge[aligned edge, loop above] node[above left=0.1cm] {$t_1$} node[above right=0.1cm] {$\update(x) = x$\\$\update(y) = x + y + z$\\$\update(z) = y - z$} (1)
    (1) edge[aligned edge] node[above=0.2cm] {$t_2$} node[below=0.2cm] {$\update(x) = 5$\\$\update(y) = z$\\$\update(z) = y$} (2)
    ;
\end{tikzpicture}
\caption{Program for the illustration of RVGs}
\label{fig:rvg_example_program}
\end{figure}


As an example consider the program in Figure \ref{fig:rvg_example_program}.
The set of pre-transitions for $t_0$ is empty, since no transition can occur before an initial transition $t_0 \in \TSet_0$.
The set of pre-transitions for $t_1$ is $\pre(t_1) = \braced{t_0, t_1}$, since both $t_0$ and $t_1$ may occur directly before the occurrence of the transition $t_1$.
Also, the set of pre-transitions for $t_2$ is $\pre(t_2) = \braced{t_0, t_1}$.

Additionally to pre-transitions, we define the active variables of bounds. 

\begin{definition}[Active variables]
  Let $\VSet$ be a finite set of variables.  
  A variable $v \in \VSet$ is active in a bound $b \in \BoundSet(\VSet)$ if and only if there are two states $\valuation, \valuation' \in \Valuation$ with $\valuation(v') = \valuation'(v')$ for each $v' \in \VSet \setminus \braced{v}$ and $\valuation(v) \neq \valuation'(v)$ such that $\exacteval{b}{\valuation} \neq \exacteval{b}{\valuation'}$.
  Then, the function $\actV: \BoundSet(\VSet) \rightarrow 2^\VSet$ describes with $\actV(b)$ all variables, which are active in the bound $b \in \BoundSet(\VSet)$.
\end{definition}

Again consider the program in Figure \ref{fig:rvg_example_program}.
The update $\update_1(y)$ has the active variables $\actV(\update_1(y)) = \braced{x,y,z}$, while the update $\update_2(x)$ does not have active variables (i.e. $\actV(\update_2(x)) = \emptyset$).

A valid overapproximation of the active variables of a bound $b \in \BoundSet(\VSet)$ is the set of all occurring variables.
\begin{definition}[Overapproximation of active variables]
  For a bound $b \in \BoundSet(\VSet)$ we define the active variables $\actV(b)$.
  \[ \actV(\infty) = \emptyset \text{ for } \infty \in \BoundSet(\VSet) \]
  \[ \actV(k) = \emptyset \text{ for all } k \in \mathbb{N} \subset \BoundSet(\VSet) \] 
  \[ \actV(v) = \braced{v} \text{ for all } v \in \VSet \subset \BoundSet(\VSet) \] 
  \[ \actV(-b) = \actV(b) \text{ for all } b \in \BoundSet(\VSet) \] 
  \[ \actV(b_1 + b_2) = \actV(b_1) \cup \actV(b_2) \text{ for all } b_1, b_2 \in \BoundSet(\VSet) \] 
  \[ \actV(b_1 \cdot b_2) = \actV(b_1) \cup \actV(b_2) \text{ for all } b_1, b_2 \in \BoundSet(\VSet) \] 
  \[ \actV(\max(b_1, b_2)) = \actV(b_1) \cup \actV(b_2) \text{ for all } b_1, b_2 \in \BoundSet(\VSet) \]
  \[ \actV(k^b) = \actV(b) \text{ for all } k \in \mathbb{N} \subset \BoundSet(\VSet), b \in \BoundSet(\VSet) \]  
\end{definition}

Note that this overapproximation recognizes the variable $x$ as active in a bound $x - x$.
For our purposes this is sufficient, but in some cases, the implementation is also able to detect such properties due to the simplification of bounds.
The bound $x - x$ is then simplified to the bound $0$ and as a consequence, the variable $x$ would not be considered as active.

Additionally to these definitions, the definition of a result variable graph is based on local size bounds.
A local size bound $\LSB$ approximates the value of a variable $v \in \VSet$ after a single evaluation step.
While the lower local size bound $\LLSB(t,v)$ yields an underapproximation of the actual value of $v$ after an evaluation step with the transition $t$, the upper local size bound $\ULSB$ yields an overapproximation.
Local size bounds are discussed in detail later in this thesis.

With those definitions, it is possible to define the result variable graph.

\begin{definition}[Result variable graph]
  Let $\LSB$ be a local size bound.
  We define 
  \[ \text{RVG} = (\TSet \times \VSet, \braced{((t, v), (t', v')) \mid t \in \pre(t'), v \in \actV(\LLSB(t',v')) \cup \actV(\ULSB(t',v'))}) \]
  to denote the result variable graph.
\end{definition}

Note that since pre-transitions and active variables are both defined as overapproximations, a result variable graph is an overapproximation of the dependencies between result variables as well.

\begin{figure}
\centering

\begin{tikzpicture}[->,>=stealth',auto,node distance=5cm,
    thick,
    main node/.style={circle,draw,font=\sffamily\Large\bfseries},
    aligned edge/.style={align=left}]

  \node[main node] (0) {$l_0$};
  \node[main node] (1) [right of=0] {$l_1$};
  \node[main node] (2) [right of=1] {$l_2$};

  \path[every node/.style={font=\sffamily\small}]
    (0) edge[aligned edge] node[above=0.2cm] {$t_0$} node[below=0.2cm] {$\update(x) = x$\\$\update(y) = y$\\$\update(z) = z$} (1)
    (1) edge[aligned edge, loop above] node[left=0.2cm] {$t_1$} node[right=0.2cm] {$\update(x) = x$\\$\update(y) = x + y + z$\\$\update(z) = y - z$} (1)
    (1) edge[aligned edge] node[above=0.2cm] {$t_2$} node[below=0.2cm] {$\update(x) = 5$\\$\update(y) = z$\\$\update(z) = y$} (2)
    ;
\end{tikzpicture}

\begin{tikzpicture}[->,>=stealth',auto,node distance=1.5cm]

  \node (0x) {$t_0,x$};
  \node (0y) [below of=0x] {$t_0,y$};
  \node (0z) [below of=0y] {$t_0,z$};
  \node (1x) [right of=0x] {$t_1,x$};
  \node (1y) [below of=1x] {$t_1,y$};
  \node (1z) [below of=1y] {$t_1,z$};
  \node (2x) [right of=1x] {$t_2,x$};
  \node (2y) [below of=2x] {$t_2,y$};
  \node (2z) [below of=2y] {$t_2,z$};

  \path
    (0x) edge (1x)
    (0x) edge (1y)
    (0y) edge (1y)
    (0y) edge (1z)
    (0z) edge (1y)
    (0z) edge (1z)
    (1x) edge[loop above] (1x)
    (1x) edge (1y)
    (1y) edge[loop right] (1y)
    (1y) edge (1z)
    (1y) edge (2z)
    (1z) edge (1y)
    (1z) edge[loop below] (1z)
    (1z) edge (2y)
    ;
\end{tikzpicture}

\caption{Program with only trivial SCCs}
\label{fig:trivial_sizebound_example}
\end{figure}


Consider the result variable graph in Figure \ref{fig:rvg_example}.
This RVG corresponds to the program in Figure \ref{fig:rvg_example_program}.
The program consists of an initial transition $t_0$, a loop $t_1$ and an exit transition $t_2$.
The value of the variable $x$ is completely unaffected by the other variables of the program and is set to a constant in the exit transition $t_2$.
While this information is not directly available in the program graph, it is shown in the result variable graph, since no edge is leading from a result variable with $y$ or $z$ to a result variable with $x$.
On the other hand, the variables $y$ and $z$ affect each other in the loop $t_1$.
This is directly reflected by the fact, that $(t_1,y)$ and $(t_1,z)$ form an SCC in the result variable graph.
The analysis of this master's thesis, as well as the original KoAT \cite{koat}, use that information about result variables forming an SCC to determine size bounds.
