In this section we present a template for local size bounds.
If we require the local size bounds to be a of that form, we are able to use those characteristics in the definition of the methods for global size bounds.
Therefore we define two subsets $\BoundSet^\sqcap_l, \BoundSet^\sqcup_l \subseteq \BoundSet$.

\begin{definition}[Local bound sets]
  We define $\BoundSet^\sqcap_l \in \BoundSet$ as 
  \[ \BoundSet^\sqcap_l = \braced{
    s \cdot \left(
        e
      + \sum_{v \in P_1} v
      - \sum_{v \in N_1} v
      + \sum_{v \in P_2} \maxO{v}
      + \sum_{v \in N_2} \maxO{-v}
      \right) \mid \exists s \in \mathbb{N} \text{ with } s \geq 1, e \in \mathbb{Z}, P_1, P_2, N_1, N_2 \subseteq \VSet }\]
  Also we define $\BoundSet^\sqcup_l \in \BoundSet$ as 
  \[ \BoundSet^\sqcup_l = \braced{
    s \cdot \left(
        e
      + \sum_{v \in P_1} v
      - \sum_{v \in N_1} v
      - \sum_{v \in P_2} \maxO{-v}
      - \sum_{v \in N_2} \maxO{v}
      \right) \mid \exists s \in \mathbb{N} \text{ with } s \geq 1, e \in \mathbb{Z}, P_1, P_2, N_1, N_2 \subseteq \VSet }\]
\end{definition}

We now use these local bound sets for the definition of the scaled sum.

\begin{definition}[Scaled sum]
  We say that $\alpha \in \RV$ is bounded by an \textbf{upper} scaled sum if and only if there is a bound $b \in \BoundSet^\sqcap_l$ such that it holds for all states $\valuation$ that
  \[ \eval{\ULSB(\alpha)}{\valuation} = \eval{b}{\valuation} \]
  We then denote with $s^\sqcap_\alpha$ the scaling factor $s$, with $e^\sqcap_\alpha$ the constant $e$ and with the sets $P_{\alpha,1}^\sqcap, P_{\alpha,2}^\sqcap, N_{\alpha,1}^\sqcap, N_{\alpha,2}^\sqcap$ the four variable sets $P_1$, $P_2$, $N_1$ and $N_2$.
  We say that $\alpha \in \RV$ is bounded by a \textbf{lower} scaled sum if and only if there is a bound $b \in \BoundSet^\sqcup_l$ such that it holds for all states $\valuation$ that
  \[ \eval{\LLSB(\alpha)}{\valuation} = \eval{b}{\valuation} \]
  We then denote with $s^\sqcup_\alpha$ the scaling factor $s$, with $e^\sqcup_\alpha$ the constant $e$ and with the sets $P_{\alpha,1}^\sqcup, P_{\alpha,2}^\sqcup, N_{\alpha,1}^\sqcup, N_{\alpha,2}^\sqcup$ the four variable sets $P_1$, $P_2$, $N_1$ and $N_2$.
\end{definition}

We show the possible bounds with this definition by some examples.
Let $t = (\location,\tau,\location') \in \TSet$ be a transition for which we want to find a local size bound $b$.
If $\tau = (x' = x)$, we can represent the bound with $b = 1 \cdot (0 + x)$.
If $\tau = (x' = -x)$, we can represent the bound with $b = 1 \cdot (0 - x)$.
This would not be possible, if $N_1$ would not be a part of the template.
If $\tau = (x' = 2 \cdot x + 2 \cdot y + 4)$, we have a possible bound $b = 2 \cdot (2 + x + y)$.

For the previous examples, it was sufficient to have $P_2 = \emptyset$ and $N_2 = \emptyset$.
If $\tau = (x' = 2 \cdot x + 3 \cdot y)$, one could suggest the bound $b = 3 \cdot (x + y)$.
But for a state $\valuation$ with $\eval{x}{\valuation} < 0$ and $\eval{y}{\valuation} = k$ this bound is not sound, since with $\eval{b}{\valuation}$ we would underapproximate the actual value $\eval{2 \cdot x + 3 \cdot y}{\valuation}$.
Therefore the elements of the local bound set provide the sets $P_2$ and $N_2$.
For $\tau = (x' = 2 \cdot x + 3 \cdot y)$ we can then infer the bound $b = 3 \cdot (\maxO{x} + y)$.
With the defined state $\valuation$ we then have $\eval{b}{\valuation} = k$, which is a sound overapproximation of the actual value. 

For the previous case we used the set $P_2$.
We provide the set $N_2$ for the negative counterpart.
If $\tau = (x' = -(2 \cdot x) + -(3 \cdot y))$, we can infer the bound $b = 3 \cdot (\maxO{-x} - y)$.
With this construction we only consider negative values for $x$ and therefore ensure a sound overapproximation.

We will now present the algorithm for the computation of sound, but minimal local size bounds $b^\sqcap \in \BoundSet^\sqcap_l$ and $b^\sqcup \in \BoundSet^\sqcup_l$.
We first consider the upper case.
\todo{As mentioned before}{Fix with usage of tau}, a local size bound $b = \ULSB(t,v)$ is a sound overapproximation of the effect of a transition $t$, iff for all states $\valuation$ it holds that
\[ \eval{\ULSB(t, v)}{\valuation} \geq \sup \braced{\valuation'(v) \mid \exists \valuation': \valuation \rightarrow_t \valuation'}. \]
Therefore we are looking for a model of the following formula.
\[ \exists s \in \mathbb{N} \text{ with } s \geq 1:
\exists e \in \mathbb{Z}:
\exists P_1, P_2, N_1, N_2 \subseteq \VSet:
\forall \valuation \in \Valuation:
\eval{b}{\valuation} \geq \sup \braced{\valuation'(v) \mid \exists \valuation': \valuation \rightarrow_t \valuation'} \]
\todo{Cite}{An SMT-Solver is only able to find models for existential formulas} $\exists v_0: \exists v_1: \dots \exists v_k: \phi(v_0, \dots, v_k)$.
Therefore we use an approach, where we fix the parameters $s$, $e$ and $P_1, N_1, P_2, N_2$ and use the SMT-Solver to prove unsatisfiability of the negation of the formula.
That is, for fixed $s \in \mathbb{N}$ with $s \geq 1$, $e \in \mathbb{Z}$ and $P_1, N_1, P_2, N_2 \subseteq \VSet$ we search for a model for the following formula.
\[ \exists \valuation \in \Valuation: \eval{b}{\valuation} < \sup \braced{\valuation'(v) \mid \exists \valuation': \valuation \rightarrow_t \valuation'} \]
If the SMT-Solver does not find a model, the formula is unsatisfiable and therefore the fixed parameters form a sound overapproximation.

It is still open, how we choose the parameters.
Therefore we first need to define a measure for the quality of local bounds.
We say that $b_1 \in \BoundSet^\sqcap_l$ is a better bound than $b_2 \in \BoundSet^\sqcap_l$ iff for all states $\valuation$ it holds that $\eval{b_1}{\valuation} \leq \eval{b_2}{\valuation}$.


\todo{Provide the algorithm how we infer those sizebounds}{}

\todo{Proof that the algorithm always finds a valid local sizebound}{}
