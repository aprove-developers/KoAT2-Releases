\section{Introduction}

In this master's thesis we will present a method to find timebounds for arbitrary integer programs.
There exist complexity analysis tools which already address the same task.
Those tools use different approaches for the search for timebounds.
The tool Loopus \cite{loopus1} \cite{loopus2} \todo{Do and check papers}{...}
The tool CoFloCo \cite{cofloco1} \cite{cofloco2} \cite{cofloco3} \cite{cofloco4} \todo{Do and check papers}{...}
The tool KoAT \cite{koat} \todo{Do}{...}

The method of this master's thesis is based on the work of KoAT \cite{koat}.
It changes the methods of KoAT, such that the restriction for non-monotonic bounds is no longer necessary.
As a result the method is now able to infer non-monotonic time bounds.

We now take a look at an example to see the benefit of non-monotonic bounds.
Figure \ref{fig:motivational_example} shows a program, which takes two variables $x$ and $y$ as input and runs a loop, which decrements $x$ in each step with $t_1$ until $x$ is not greater than 0 anymore.
We can easily see, that the transition $t_1$ will be used at most $\max \braced{0, x-y}$ times.
However, since the previous method only used monotonic bounds, it only infers the bound $\abs{x}+\abs{y}$ for $t_1$.

\todo{Take a look at why lower time bounds are not an option}{}

\subsection{Motivation}

\begin{figure}
\centering
\begin{tikzpicture}[->,>=stealth',auto,node distance=5cm,
    thick,
    main node/.style={circle,draw,font=\sffamily\Large\bfseries},
    aligned edge/.style={align=left}]

  \node[main node] (0) {$l_0$};
  \node[main node] (1) [right of=0] {$l_1$};

  \path[every node/.style={font=\sffamily\small}]
    (0) edge[aligned edge] node {$t_0: x' = x \wedge y' = y$} (1)
    (1) edge[aligned edge, loop right] node {$t_1: x > y \wedge x' = x - 1 \wedge y' = y$} (1)
    ;
\end{tikzpicture}
\caption{Motivational program for non-monotonic bounds}
\label{fig:motivational_example}
\end{figure}


\begin{figure}
\centering
\begin{tikzpicture}[->,>=stealth',auto,node distance=5cm,
    thick,
    main node/.style={circle,draw,font=\sffamily\Large\bfseries},
    aligned edge/.style={align=left}]

  \node[main node] (0) {$l_0$};
  \node[main node] (1) [right of=0] {$l_1$};

  \path[every node/.style={font=\sffamily\small}]
    (0) edge[aligned edge] node {$t_0: x' = x \wedge y' = y$} (1)
    (1) edge[aligned edge, loop right] node {$t_1: x > 0 \wedge x' = x - 1 \wedge y' = -2 \cdot y$} (1)
    ;
\end{tikzpicture}
\caption{Program, where non-monotonic bounds does not yield a benefit}
\label{fig:non_beneficial_example}
\end{figure}

