The last section introduced local size bounds in the form of scaled sums.
This and the following section present a method to lift these local size bounds to global size bounds.

First, we provide global size bounds for trivial SCCs of the result variable graph.
Note that a node in the result variable graph is a result variable $\rv = (t,v) \in \RV$.
An SCC with a single node without an edge to itself is called trivial.

We have to distinguish two cases.
If the transition $t$ of the result variable $\rv$ is an initial transition $t \in \TSet_0$, then the local size bound already depends on the start values.
Thus, the local size bound for the result variable $\rv$ is also a global size bound.
If the transition $t$ is not an initial transition (i.e. $t \notin \TSet_0$), then we have to substitute the variables of the local size bound by the global size bounds of the predecessors of $t$.
The method of \cite{koat} uses monotonically increasing bounds as local size bounds.
Therefore, it is able to substitute every variable with a bound of the maximum absolute value.
The new method uses a definition of scaled sums which is non-monotonic.
Therefore, additional logic is necessary to ensure that the substituted bounds are valid approximations.
This additional logic is realized by the presented approximated substitutions.
For lower local size bounds it is valid to use the underapproximated substitution, for upper local size bounds it is valid to use the overapproximated substitution.
Then, the only step left is to select the minimal underapproximated substitution for lower bounds and the maximal overapproximated substitution for upper bounds.
The following theorem formally introduces the computation of size bounds for trivial SCCs.

With the definition of local size bounds in the form of scaled sums, it is now possible to lift those local effects to a global level.
Instead of expressing the effect depending on the values immediately before a transition, we will now express the effect depending on the start values.

We first provide those global size bounds for trivial SCCs of the RVG.
Note that a node in the RVG is a result variable $\alpha = (t,v) \in \RV$.
Trivial SCCs consist of a single node and they have the property that every evaluation passes the SCC only once.
In the graph this is recognizable through the fact, that no sequence of edges is leading back to the single node.

Again, we have to distinguish two cases.
If the transition $t$ of the result variable $\alpha$ of the node is an initial transition $t \in \TSet_0$, then the effect of the local size bound already depends on the start values.
Thus, the local size bound for the result variable $\alpha$ is also a global size bound.
If the transition $t$ is not an initial transition $t \notin \TSet_0$, then we have to substitute the variables of the local size bound by the global size bounds before entering the SCC.
The old KoAT tool uses monotonically increasing bounds for the local size bounds.
Therefore it is able to substitute every variable with its upper size bound obtained so far.
Since the defined scaled sums for local size bounds of KoAT2 are non-monotonic, it is necessary to split additional cases.
For scaled sums we have monotonically increasing and monotonically decreasing components.
Let $\alpha = (t,v) \in \RV$ be a result variable.
Then, the upper local size bound $\ULSB(\alpha)$ and the lower local size bound $\LLSB(\alpha)$ are monotonically increasing for each variable $v \in P_{\alpha,1} \cup P_{\alpha,2}$.
On the other hand for each variable $v \in N_{\alpha,1} \cup N_{\alpha,2}$ the upper local size bound $\ULSB(\alpha)$ and the lower local size bound $\LLSB(\alpha)$ are monotonically decreasing.

The redefined upper size bound ${\USize}'$ must be a sound overapproximation of the values of the variables.
Since upper size bounds $\USize$ overapproximate the value of variables and lower size bounds $\LSize$ underapproximate the value, it is sound to use the upper size bound $\USize$ for the substitution with the monotonically increasing variables and the lower size bound $\LSize$ for the substitution with the monotonically decreasing variables.

The redefined lower size bound ${\LSize}'$ must be a sound underapproximation of the values of the variables.
Therefore the upper size bound $\USize$ must be used for the substitution with the monotonically decreasing variables and the lower size bound $\LSize$ must be used for the substitution with the monotonically increasing variables.
The following definition formally introduces the computation of size bounds for trivial SCCs.

\begin{theorem}[SizeBounds for trivial SCCs]
  Let $(\UTime, \Size)$ be a complexity approximation.
  Let $\braced{\alpha} \subseteq \RV$ be a trivial SCC of the RVG.
  Let $\LSB$ be a local size bound in the form of a scaled sum.
  We define $\USize'(\alpha') = \USize(\alpha')$ and $\LSize'(\alpha') = \LSize(\alpha')$ for all $\alpha' \neq \alpha$.
  For all $\alpha = (t, v')$ with an initial transition $t$ we define
  \[ \LSize'(\alpha) = \LLSB(\alpha) \text{ and } \USize'(\alpha) = \ULSB(\alpha). \]
  For all $\alpha = (t, v')$ where $t$ is not an initial transition we define the following.
  We define $\LSize'(\alpha) = \LSize(\alpha)$, if $\alpha$ is not bounded by a lower scaled sum.
  Similar, we define $\USize'(\alpha) = \USize(\alpha)$, if $\alpha$ is not bounded by an upper scaled sum.
  For all $\alpha$ which are bounded by an upper scaled sum and by a lower scaled sum we define 
  \begin{equation}
    \begin{split}
      \USize'(\alpha) = \max \braced{ s^\sqcap_\alpha \cdot ( e^\sqcap_\alpha & + \sum_{v \in P_{\alpha,1}^\sqcap} \USize(\tilde{t}, v) - \sum_{v \in N_{\alpha,1}^\sqcap} \LSize(\tilde{t}, v) \\
        & + \sum_{v \in P_{\alpha,2}^\sqcap} \maxO{\USize(\tilde{t}, v)} + \sum_{v \in N_{\alpha,2}^\sqcap} \maxO{-\LSize(\tilde{t}, v)} ) \mid \tilde{t} \in \pre(t)}
    \end{split}
  \end{equation}
  \begin{equation}
    \begin{split}
      \LSize'(\alpha) = \min \braced{ s^\sqcup_\alpha \cdot ( e^\sqcup_\alpha & + \sum_{v \in P_{\alpha,1}^\sqcup} \LSize(\tilde{t}, v) - \sum_{v \in N_{\alpha,1}^\sqcup} \USize(\tilde{t}, v) \\
        & - \sum_{v \in P_{\alpha,2}^\sqcup} \maxO{-\LSize(\tilde{t}, v)} - \sum_{v \in N_{\alpha,2}^\sqcup} \maxO{\USize(\tilde{t}, v)} ) \mid \tilde{t} \in \pre(t)}
    \end{split}
  \end{equation}
  Then, $\mathit{SizeBounds}(\UTime, \Size, \braced{\alpha}) = (\LSize', \USize')$ is also a size approximation.
\end{theorem}

\todo{Example}{}


We explain the presented method with an example program.

\begin{figure}
\centering

\begin{tikzpicture}[->,>=stealth',auto,node distance=5cm,
    thick,
    main node/.style={circle,draw,font=\sffamily\Large\bfseries},
    aligned edge/.style={align=left}]

  \node[main node] (0) {$l_0$};
  \node[main node] (1) [right of=0] {$l_1$};
  \node[main node] (2) [right of=1] {$l_2$};

  \path[every node/.style={font=\sffamily\small}]
    (0) edge[aligned edge] node[above=0.2cm] {$t_0$} node[below=0.2cm] {$\update(x) = 2 \cdot x - 3 \cdot y + 1$\\$\update(y) = y$\\$\guard = \braced{y \leq 0}$} (1)
    (1) edge[aligned edge] node[above=0.2cm] {$t_1$} node[below=0.2cm] {$\update(x) = -2 \cdot x + 3 \cdot y - 1$\\$\update(y) = y$} (2)
    ;
\end{tikzpicture}

\begin{tikzpicture}[->,>=stealth',auto,node distance=1.5cm]

  \node (0) {$t_0,x$};
  \node (1) [right of=0] {$t_1,x$};
  \node (2) [below of=0] {$t_0,y$};
  \node (3) [right of=2] {$t_1,y$};

  \path
    (0) edge (1)
    (2) edge (1)
    (2) edge (3)
    ;
\end{tikzpicture}

\caption{A program with its result variable graph}
\label{fig:rvg_example}
\end{figure}


Consider the program in Figure \ref{fig:trivial_sizebound_example}.
The program takes an arbitrary integer $x$ and an integer $y \leq 0$ and returns the variable $y$ unchanged, while it assigns the variable $x$ in both transitions a value depending on the incoming variables $x$ and $y$.
The result variable graph contains four trivial SCCs, each result variable forming its own SCC.
We now inspect the SCCs $\braced{(t_0,x)}$ and $\braced{(t_1,x)}$.
  
\begin{example}[Size Bounds for Trivial SCCs with an Initial Transition]
  For the SCC $\braced{(t_0,x)}$, the transition $t_0$ is an initial transition.
  We can determine a scaled sum $3 \cdot (1 + \maxO{x} - y)$ as upper local size bound $\ULSB(t_0,x)$ and a scaled sum $2 \cdot x$ as lower local size bound $\LLSB(t_0,x)$.
  Since $t_0$ is an initial transition, we have $\USize'(t_0,x) = \ULSB(t_0,x)$ and $\LSize'(t_0,x) = \LLSB(t_0,x)$ and therefore the determined local size bounds are also global size bounds.
\end{example}
  
\begin{example}[Size Bounds for Trivial SCCs with a Non-Initial Transition]
  For the SCC $\braced{(t_1,x)}$ the transition $t_1$ is not an initial transition.
  Therefore, the local size bound expresses the values of the variables in terms of their value immediately before the execution of $t_1$.
  For the computation of global size bounds, we need to substitute these variables with the global size bound obtained so far until the execution of $t_1$.
  We already inferred these size bounds $\USize(t_0,x)$ and $\LSize(t_0,x)$ for the single transition $t_0$ leading to $t_1$.
  Additionally, we can trivially infer the size bounds $\USize(t_0,y) = y$ and $\LSize(t_0,y) = y$.
  We can also determine a scaled sum $3 \cdot (\maxO{-x} + y)$ as upper local size bound $\ULSB(t_1,x)$ and a scaled sum $3 \cdot (-1 - \maxO{x} + y)$ as lower local size bound $\LLSB(t_1,x)$.
  For the result variable $\rv = (t_0,x)$, the parameters of the upper local size bound are $s^\sqcap_\rv = 3$, $e^\sqcap_\rv = 0$, $N^\sqcap_{\rv,2} = \braced{x}$, $P^\sqcap_{\rv,1} = \braced{y}$ and $P^\sqcap_{\rv,2} = N^\sqcap_{\rv,1} = \emptyset$.
  The parameters of the lower local size bound are $s^\sqcup_\rv = 3$, $e^\sqcup_\rv = -1$, $N^\sqcup_{\rv,2} = \braced{x}$, $P^\sqcup_{\rv,1} = \braced{y}$ and $P^\sqcup_{\rv,2} = N^\sqcup_{\rv,1} = \emptyset$.
  For a sound approximation, we need to use the underapproximated substitution and the overapproximated substitution respectively.
  Therefore, the resulting global upper size bound $\USize'(t_1,x)$ is 
  \begin{align*}
    && \usubst{3 \cdot \left( \maxO{-x} + y \right) }{\LSize(t_0)}{\USize(t_0)} \\
    & = & 3 \cdot \left( \maxO{-\LSize(t_0, x)} + \USize(t_0, y) \right) \\
    & = & 3 \cdot \left( \maxO{-2 \cdot x} + y \right)
  \end{align*}
  The resulting global lower size bound $\LSize'(t_1,x)$ we can determine as
  \begin{align*}
    && \lsubst{3 \cdot \left( -1 - \maxO{x} + y \right) }{\LSize(t_0)}{\USize(t_0)} \\
    & = & 3 \cdot \left( -1 - \maxO{\USize(t_0, x)} + \LSize(t_0, y) \right) \\
    & = & 3 \cdot \left( -1 - \maxO{3 \cdot (1 + \maxO{x} - y)} + y \right)
  \end{align*}
\end{example}
