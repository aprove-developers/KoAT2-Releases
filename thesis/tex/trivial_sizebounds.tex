In the last section local size bounds in the form of scaled sums were introduced.
This and the following section present a method to lift those local size bounds to global size bounds.

We first provide global size bounds for trivial SCCs of the RVG.
Note that a node in the RVG is a result variable $\rv = (t,v) \in \RV$.
Trivial SCCs consist of a single node without an edge to itself.

We have to distinguish two cases.
If the transition $t$ of the result variable $\rv$ is an initial transition $t \in \TSet_0$, then the effect of the local size bound already depends on the start values.
Thus, the local size bound for the result variable $\rv$ is also a global size bound.
If the transition $t$ is not an initial transition (e.g. $t \notin \TSet_0$), then we have to substitute the variables of the local size bound by the global size bounds before entering the SCC.
The original KoAT uses monotonically increasing bounds for the local size bounds.
Therefore, it is able to substitute every variable with a bound of the maximum absolute value obtained so far.
The new method uses a definition of scaled sums which is non-monotonic.
Therefore, additional logic is necessary to ensure that the substituted bounds are valid approximations.
This additional logic is realized by the presented approximated substitutions.
For lower local size bounds it is valid to use the underapproximated substitution, for upper local size bounds it is valid to use the overapproximated substitution.
Then, the only step left is to select the minimal underapproximated substitution for lower bounds and the maximal overapproximated substitution for upper bounds.
The following definition formally introduces the computation of size bounds for trivial SCCs.

\begin{theorem}[Size Bounds for Trivial SCCs]
  Let $\UTime$ be a time bound and let $\Size$ be a size bound.
  Let $\SCC = \braced{(t,v)} \subseteq \RV$ be a trivial SCC of the result variable graph.
  Let $\LSB$ be a local size bound.
  We define $\USize'(\outrv) = \USize(\outrv)$ and $\LSize'(\outrv) = \LSize(\outrv)$ for all $\outrv \notin \SCC$.
  Otherwise, we define $\USize'(t,v)$ as
  \[ \USize'(t,v) =
  \begin{cases}
    \ULSB(t, v) & \text{ if } t \in \TSet_0 \\
    \USize(t, v) & \text{ if } t \in \TSet \setminus \TSet_0 \text{ and } \ULSB(t,v) \notin \BoundSet^\sqcap_l \\
    \maximum{ \usubst{\ULSB(t,v)}{\LSize(\pret)}{\USize(\pret)} \mid \pret \in \pre(t)} & \text{ if } t \in \TSet \setminus \TSet_0 \text{ and } \ULSB(t,v) \in \BoundSet^\sqcap_l \\
  \end{cases}
  \]
  Similarly, we define $\LSize'(t,v)$ as
  \[ \LSize'(t,v) =
  \begin{cases}
    \LLSB(t, v) & \text{ if } t \in \TSet_0 \\
    \LSize(t, v) & \text{ if } t \in \TSet \setminus \TSet_0 \text{ and } \LLSB(t,v) \notin \BoundSet^\sqcup_l \\
    \minimum{ \lsubst{\LLSB(t,v)}{\LSize(\pret)}{\USize(\pret)} \mid \pret \in \pre(t)} & \text{ if } t \in \TSet \setminus \TSet_0 \text{ and } \LLSB(t,v) \in \BoundSet^\sqcup_l \\
  \end{cases}
  \]
  Then, $\text{SizeBounds}(\UTime, \Size, \braced{(t,v)}) = (\LSize', \USize')$ is also a size bound.
\end{theorem}


We explain the presented method with an example.
Consider the program in figure \ref{fig:trivial_sizebound_example}.

\begin{figure}
\centering

\begin{tikzpicture}[->,>=stealth',auto,node distance=5cm,
    thick,
    main node/.style={circle,draw,font=\sffamily\Large\bfseries},
    aligned edge/.style={align=left}]

  \node[main node] (0) {$l_0$};
  \node[main node] (1) [right of=0] {$l_1$};
  \node[main node] (2) [right of=1] {$l_2$};

  \path[every node/.style={font=\sffamily\small}]
    (0) edge[aligned edge] node[above=0.2cm] {$t_0$} node[below=0.2cm] {$\update(x) = 2 \cdot x - 3 \cdot y + 1$\\$\update(y) = y$\\$\guard = \braced{y \leq 0}$} (1)
    (1) edge[aligned edge] node[above=0.2cm] {$t_1$} node[below=0.2cm] {$\update(x) = -2 \cdot x + 3 \cdot y - 1$\\$\update(y) = y$} (2)
    ;
\end{tikzpicture}

\begin{tikzpicture}[->,>=stealth',auto,node distance=1.5cm]

  \node (0) {$t_0,x$};
  \node (1) [right of=0] {$t_1,x$};
  \node (2) [below of=0] {$t_0,y$};
  \node (3) [right of=2] {$t_1,y$};

  \path
    (0) edge (1)
    (2) edge (1)
    (2) edge (3)
    ;
\end{tikzpicture}

\caption{Program with only trivial SCCs}
\label{fig:trivial_sizebound_example}
\end{figure}


The program takes an arbitrary integer $x$ and an integer $y \leq 0$ and returns the variable $y$ unchanged, while it assigns the variable $x$ in both transitions a value depending on the incoming variables $x$ and $y$.
The result variable graph contains four trivial SCCs, each result variable forming an own SCC.
We now inspect the SCCs $\braced{(t_0,x)}$ and $\braced{(t_1,x)}$.

For the SCC $\braced{(t_0,x)}$ the transition $t_0$ is an initial transition.
We can determine a scaled sum $3 \cdot (1 + \maxO{x} - y)$ as upper local size bound $\ULSB(t_0,x)$ and a scaled sum $2 \cdot x$ as lower local size bound $\LLSB(t_0,x)$.
Since we have $\USize'(t_0,x) = \ULSB(t_0,x)$ and $\LSize'(t_0,x) = \LLSB(t_0,x)$, the determined local size bounds are also global size bounds.

For the SCC $\braced{(t_1,x)}$ the transition $t_1$ is not an initial transition.
Therefore, the local size bound expresses the values of the variables in terms of their value immediately before the execution of $t_1$.
For the computation of the global size bounds, we need to substitute these variables with the global size bound obtained so far until the execution of $t_1$.
We already inferred those size bounds $\USize(t_0,x)$ and $\LSize(t_0,x)$ for the single transition $t_0$ leading to $t_1$.
Additionally, we can trivially infer the size bounds $\USize(t_0,y) = y$ and $\LSize(t_0,y) = y$.
We can also determine a scaled sum $3 \cdot (\maxO{-x} + y)$ as upper local size bound $\ULSB(t_1,x)$ and a scaled sum $3 \cdot (-1 - \maxO{x} + y)$ as lower local size bound $\LLSB(t_1,x)$.
For the result variable $\rv = (t_0,x)$ the parameters of the upper local size bound are $s^\sqcap_\rv = 3$, $e^\sqcap_\rv = 0$, $N^\sqcap_{\rv,2} = \braced{x}$, $P^\sqcap_{\rv,1} = \braced{y}$ and $P^\sqcap_{\rv,2} = N^\sqcap_{\rv,1} = \emptyset$ and the parameters of the lower local size bound are $s^\sqcup_\rv = 3$, $e^\sqcup_\rv = -1$, $N^\sqcup_{\rv,2} = \braced{x}$, $P^\sqcup_{\rv,1} = \braced{y}$ and $P^\sqcup_{\rv,2} = N^\sqcup_{\rv,1} = \emptyset$.
For a sound approximation, we need to use the underapproximated substitution and the overapproximated substitution respectively.
The resulting global upper size bound $\USize'(t_1,x)$ therefore is 
\begin{align*}
  && \usubst{3 \cdot \left( \maxO{-x} + y \right) }{\LSize(t_0)}{\USize(t_0)} \\
  & = & 3 \cdot \left( \maxO{-\LSize(t_0, x)} + \USize(t_0, y) \right) \\
  & = & 3 \cdot \left( \maxO{-2 \cdot x} + y \right)
\end{align*}
The resulting global lower size bound $\LSize'(t_1,x)$ we can determine as
\begin{align*}
  && \lsubst{3 \cdot \left( -1 - \maxO{x} + y \right) }{\LSize(t_0)}{\USize(t_0)} \\
  & = & 3 \cdot \left( -1 - \maxO{\USize(t_0, x)} + \LSize(t_0, y) \right) \\
  & = & 3 \cdot \left( -1 - \maxO{3 \cdot (1 + \maxO{x} - y)} + y \right)
\end{align*}
