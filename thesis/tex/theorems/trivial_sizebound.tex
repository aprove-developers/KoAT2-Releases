With the definition of local size bounds in the form of scaled sums, it is now possible to lift those local effects to a global level.
Instead of expressing the effect depending on the values immediately before a transition, we will now express the effect depending on the start values.

We first provide those global size bounds for trivial SCCs of the RVG.
Note that a node in the RVG is a result variable $\alpha = (t,v) \in \RV$.
Trivial SCCs consist of a single node and they have the property that every evaluation passes the SCC only once.
In the graph this is recognizable through the fact, that no sequence of edges is leading back to the single node.

Again, we have to distinguish two cases.
If the transition $t$ of the result variable $\alpha$ of the node is an initial transition $t \in \TSet_0$, then the effect of the local size bound already depends on the start values.
Thus, the local size bound for the result variable $\alpha$ is also a global size bound.
If the transition $t$ is not an initial transition $t \notin \TSet_0$, then we have to substitute the variables of the local size bound by the global size bounds before entering the SCC.
The old KoAT tool uses monotonically increasing bounds for the local size bounds.
Therefore it is able to substitute every variable with its upper size bound obtained so far.
Since the defined scaled sums for local size bounds of KoAT2 are non-monotonic, it is necessary to split additional cases.
For scaled sums we have monotonically increasing and monotonically decreasing components.
Let $\alpha = (t,v) \in \RV$ be a result variable.
Then, the upper local size bound $\ULSB(\alpha)$ and the lower local size bound $\LLSB(\alpha)$ are monotonically increasing for each variable $v \in P_{\alpha,1} \cup P_{\alpha,2}$.
On the other hand for each variable $v \in N_{\alpha,1} \cup N_{\alpha,2}$ the upper local size bound $\ULSB(\alpha)$ and the lower local size bound $\LLSB(\alpha)$ are monotonically decreasing.

The redefined upper size bound ${\USize}'$ must be a sound overapproximation of the values of the variables.
Since upper size bounds $\USize$ overapproximate the value of variables and lower size bounds $\LSize$ underapproximate the value, it is sound to use the upper size bound $\USize$ for the substitution with the monotonically increasing variables and the lower size bound $\LSize$ for the substitution with the monotonically decreasing variables.

The redefined lower size bound ${\LSize}'$ must be a sound underapproximation of the values of the variables.
Therefore the upper size bound $\USize$ must be used for the substitution with the monotonically decreasing variables and the lower size bound $\LSize$ must be used for the substitution with the monotonically increasing variables.
The following definition formally introduces the computation of size bounds for trivial SCCs.

\begin{theorem}[SizeBounds for trivial SCCs]
  Let $(\UTime, \Size)$ be a complexity approximation.
  Let $\braced{\alpha} \subseteq \RV$ be a trivial SCC of the RVG.
  Let $\LSB$ be a local size bound in the form of a scaled sum.
  We define $\USize'(\alpha') = \USize(\alpha')$ and $\LSize'(\alpha') = \LSize(\alpha')$ for all $\alpha' \neq \alpha$.
  For all $\alpha = (t, v')$ with an initial transition $t$ we define
  \[ \LSize'(\alpha) = \LLSB(\alpha) \text{ and } \USize'(\alpha) = \ULSB(\alpha). \]
  For all $\alpha = (t, v')$ where $t$ is not an initial transition we define the following.
  We define $\LSize'(\alpha) = \LSize(\alpha)$, if $\alpha$ is not bounded by a lower scaled sum.
  Similar, we define $\USize'(\alpha) = \USize(\alpha)$, if $\alpha$ is not bounded by an upper scaled sum.
  For all $\alpha$ which are bounded by an upper scaled sum and by a lower scaled sum we define 
  \begin{equation}
    \begin{split}
      \USize'(\alpha) = \max \braced{ s^\sqcap_\alpha \cdot ( e^\sqcap_\alpha & + \sum_{v \in P_{\alpha,1}^\sqcap} \USize(\tilde{t}, v) - \sum_{v \in N_{\alpha,1}^\sqcap} \LSize(\tilde{t}, v) \\
        & + \sum_{v \in P_{\alpha,2}^\sqcap} \maxO{\USize(\tilde{t}, v)} + \sum_{v \in N_{\alpha,2}^\sqcap} \maxO{-\LSize(\tilde{t}, v)} ) \mid \tilde{t} \in \pre(t)}
    \end{split}
  \end{equation}
  \begin{equation}
    \begin{split}
      \LSize'(\alpha) = \min \braced{ s^\sqcup_\alpha \cdot ( e^\sqcup_\alpha & + \sum_{v \in P_{\alpha,1}^\sqcup} \LSize(\tilde{t}, v) - \sum_{v \in N_{\alpha,1}^\sqcup} \USize(\tilde{t}, v) \\
        & - \sum_{v \in P_{\alpha,2}^\sqcup} \maxO{-\LSize(\tilde{t}, v)} - \sum_{v \in N_{\alpha,2}^\sqcup} \maxO{\USize(\tilde{t}, v)} ) \mid \tilde{t} \in \pre(t)}
    \end{split}
  \end{equation}
  Then, $\mathit{SizeBounds}(\UTime, \Size, \braced{\alpha}) = (\LSize', \USize')$ is also a size approximation.
\end{theorem}

\todo{Example}{}
