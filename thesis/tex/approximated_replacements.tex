\section{Approximated replacements}

In the preliminaries, a definition for the exact evaluation of bounds from a bound set $\BoundSet(V)$ was given.
With such an exact evaluation it is possible to evaluate a bound $b \in \BoundSet$ with a state $\valuation \in \Valuation$ to a value $\exacteval{b}{\valuation} \in \ValueSet$.
But in general, we want to be able to evaluate bounds with arbitrary states $\valuation \in \Valuation$ between a lower bound $\lstate \leq \valuation$ and an upper bound $\ustate \geq \valuation$.

Therefore, we introduce approximated evaluations.
In contrast to an exact evaluation, which replaces variables with a single state $\valuation \in \Valuation$, an approximated evaluation considers two input states $\lstate, \ustate \in \Valuation$.
Then, the bound is evaluated in a way, that the result is greater than any exact evaluation with a state $\valuation$ with $\lstate \leq \valuation \ustate$.

We define a generalized principle, which is also applicable to substitutions.

\begin{definition}[Approximated replacement]
  Let $K$ be an arbitrary sets of elements with an addition $+: K \times K \rightarrow K$, a negation $-: K \rightarrow K$, a multiplication $\cdot: K \times K \rightarrow K$ and a max operator $\max: K \times K \rightarrow K$.
  For a bound $b \in \BoundSet(\VSet)$ and two functions $\lsub, \usub: \VSet \rightarrow K$, we denote with $\lsubst{b}{\lsub}{\usub}$ and $\usubst{b}{\lsub}{\usub}$ functions of the type $(\VSet \rightarrow K) \times (\VSet \rightarrow K) \rightarrow K$.
  These functions replace the variables of the bound $b$ with values from the functions $\lsub, \usub$ such that the following equations hold.
  \[ \lsubst{\infty}{\lsub}{\usub} = \usubst{\infty}{\lsub}{\usub} = \infty \text{ for } \infty \in \BoundSet(\VSet) \]
  \[ \lsubst{k}{\lsub}{\usub} = \usubst{k}{\lsub}{\usub} = k \text{ for all } k \in \mathbb{N} \subset \BoundSet(\VSet) \] 
  \[ \lsubst{v}{\lsub}{\usub} = \lsub(v) \text{ for all } v \in \VSet \subset \BoundSet(\VSet) \] 
  \[ \usubst{v}{\lsub}{\usub} = \usub(v) \text{ for all } v \in \VSet \subset \BoundSet(\VSet) \] 
  \[ \lsubst{-b}{\lsub}{\usub} = -\usubst{b}{\lsub}{\usub} \text{ for all } b \in \BoundSet(\VSet) \] 
  \[ \usubst{-b}{\lsub}{\usub} = -\lsubst{b}{\lsub}{\usub} \text{ for all } b \in \BoundSet(\VSet) \] 
  \[ \lsubst{b_1 + b_2}{\lsub}{\usub} = \lsubst{b_1}{\lsub}{\usub} + \lsubst{b_2}{\lsub}{\usub} \text{ for all } b_1, b_2 \in \BoundSet(\VSet) \] 
  \[ \usubst{b_1 + b_2}{\lsub}{\usub} = \usubst{b_1}{\lsub}{\usub} + \usubst{b_2}{\lsub}{\usub} \text{ for all } b_1, b_2 \in \BoundSet(\VSet) \] 
  \[ \lsubst{b_1 \cdot b_2}{\lsub}{\usub} = \minimum{
    \usubst{b_1}{\lsub}{\usub} \cdot \usubst{b_2}{\lsub}{\usub},
    \usubst{b_1}{\lsub}{\usub} \cdot \lsubst{b_2}{\lsub}{\usub},
    \lsubst{b_1}{\lsub}{\usub} \cdot \usubst{b_2}{\lsub}{\usub},
    \lsubst{b_1}{\lsub}{\usub} \cdot \lsubst{b_2}{\lsub}{\usub}
  } \text{ for all } b_1, b_2 \in \BoundSet(\VSet) \] 
  \[ \usubst{b_1 \cdot b_2}{\lsub}{\usub} = \maximum{
    \usubst{b_1}{\lsub}{\usub} \cdot \usubst{b_2}{\lsub}{\usub},
    \usubst{b_1}{\lsub}{\usub} \cdot \lsubst{b_2}{\lsub}{\usub},
    \lsubst{b_1}{\lsub}{\usub} \cdot \usubst{b_2}{\lsub}{\usub},
    \lsubst{b_1}{\lsub}{\usub} \cdot \lsubst{b_2}{\lsub}{\usub}
  } \text{ for all } b_1, b_2 \in \BoundSet(\VSet) \] 
  \[ \lsubst{\max(b_1, b_2)}{\lsub}{\usub} = \maximum{\lsubst{b_1}{\lsub}{\usub}, \lsubst{b_2}{\lsub}{\usub}} \text{ for all } b_1, b_2 \in \BoundSet(\VSet) \]
  \[ \usubst{\max(b_1, b_2)}{\lsub}{\usub} = \maximum{\usubst{b_1}{\lsub}{\usub}, \usubst{b_2}{\lsub}{\usub}} \text{ for all } b_1, b_2 \in \BoundSet(\VSet) \]
  \[ \lsubst{k^b}{\lsub}{\usub} = k^{\lsubst{b}{\lsub}{\usub}} \text{ for all } k \in \mathbb{N} \subset \BoundSet(\VSet), b \in \BoundSet(\VSet) \]  
  \[ \usubst{k^b}{\lsub}{\usub} = k^{\usubst{b}{\lsub}{\usub}} \text{ for all } k \in \mathbb{N} \subset \BoundSet(\VSet), b \in \BoundSet(\VSet) \]
  We call an expression $\lsubst{b}{\lsub}{\usub}$ an underapproximated replacement and $\usubst{b}{\lsub}{\usub}$ an overapproximated replacement.
\end{definition}

The bound set $\BoundSet(\VSet)$ meets the criteria of $K$.
Therefore, two functions $\lsub, \usub: \VSet \rightarrow \BoundSet(\VSet)$ qualify for an approximated replacement.
We call an approximated replacement with such functions an approximated substitution.

\todo{Prove}{}

\begin{theorem}[Soundness of approximated substitutions]
  Let $\lsub, \usub: \VSet \rightarrow \BoundSet(\VSet)$ be two substitutions with $\exacteval{\lsub(v)}{\valuation} \leq \exacteval{\usub(v)}{\valuation}$ for each variable $v \in \VSet$ and each state $\valuation \in \Valuation$.
  Let $b \in \BoundSet(\VSet)$ be a bound and $\valuation \in \Valuation$ a state.
  Then, we have
  \[ \exacteval{\lsubst{b}{\lsub}{\usub}}{\valuation} \leq \exacteval{b}{\valuation} \leq \exacteval{\usubst{b}{\lsub}{\usub}}{\valuation}. \]
\end{theorem}

Also, the two functions $\lstate, \ustate: \VSet \rightarrow K$ qualify for an approximated replacement.
We call an approximated replacement with such functions an approximated evaluation.

\begin{theorem}[Soundness of approximated evaluations]
  Let $\lstate, \valuation, \ustate: \VSet \rightarrow \ValueSet$ be three states with $\lstate \leq \valuation \leq \ustate$.
  Let $b \in \BoundSet(\VSet)$ be a bound.
  Then, it holds that
  \[ \leval{b}{\lstate}{\ustate} \leq \exacteval{b}{\valuation} \leq \ueval{b}{\lstate}{\ustate}. \]
\end{theorem}

